%% -*- coding:utf-8 -*-
%%%%%%%%%%%%%%%%%%%%%%%%%%%%%%%%%%%%%%%%%%%%%%%%%%%%%%%%%
%%   $RCSfile: hpsg-handout.tex,v $
%%  $Revision: 1.3 $
%%      $Date: 2006/05/17 12:19:27 $
%%     Author: Stefan Mueller (DFKI)
%%    Purpose: 
%%   Language: LaTeX
%%%%%%%%%%%%%%%%%%%%%%%%%%%%%%%%%%%%%%%%%%%%%%%%%%%%%%%%%
%% $Log: hpsg-handout.tex,v $
%% Revision 1.3  2006/05/17 12:19:27  stefan
%% *** empty log message ***
%%
%% Revision 1.2  2004/08/14 15:44:44  stefan
%% konstituentenreihenfolge
%%
%% Revision 1.1  2004/06/21 19:14:48  stefan
%% alte Version vor LaTeX-Beamer
%%
%% Revision 1.1  2002/01/09 20:06:23  stefan
%% Initial revision
%%
%% Revision 1.1  2001/10/21 17:01:35  stefan
%% Initial revision
%%
%%%%%%%%%%%%%%%%%%%%%%%%%%%%%%%%%%%%%%%%%%%%%%%%%%%%%%%%%

\documentclass[xcolor=dvipsnames,blackandwhite,slidestop,handout,10pt]{beamer}

%% -*- coding:utf-8 -*-

\usepackage{styles/Beamer/hu-beamer-includes-pdflatex} %,beamer-movement}



% Checkmark
%\usepackage{tikz} % Examples and documentation: http://www.texample.net/tikz
%\usepackage{bbding}
\makeatletter
\newcommand{\dingfamily}{\fontencoding{U}\fontfamily{ding}\selectfont}
\newcommand{\@chooseSymbol}[1]{{\dingfamily\symbol{#1}}}
\newcommand{\Checkmark}{\@chooseSymbol{'041}}
\makeatother

%\fuberlinlogon{0.89cm}

% creates an unggly bibliography
%\usepackage[T1]{fontenc}


\usepackage{jambox}

%\usepackage{epsfig}
\usepackage{tabularx}

%\usepackage{tikz-grid}

%\renewcommand{\trace}{\raisebox{0.2ex}{\_}\rule{0cm}{0.7em}}


%\usepackage[T4,T1]{fontenc}

% die machen \! kaputt
%\usepackage{tipa,tipx,phonetic}

% % \TH, \th, \dh, \DH
% \usepackage{wasysym}

% \let\TH=\Thorn
% \let\th=\thorn

% % Das ʒ muss aus tipa.sty muss auf textyogh umgebogen werden, da war \textezh drin.


% \let\textezh=\textyogh
% \let\textoopen=\textopeno
% %\let\texteopen=\textopene gibts nicht
% \let\texteopen=\textepsilon
% \let\textgammalatinsmall=\textbabygamma
% \let\ng=\engma
% \let\textvhook=\textscriptv

%\DeclareUnicodeCharacter {720}{\textlengthmark }
%\DeclareUnicodeCharacter {635}{\textturnrrtail\,}

%% \nodemargin5pt%\treelinewidth2pt\arrowwidth6pt\arrowlength10pt
%\selectlanguage{german}
%% \psset{nodesep=5pt} %,linewidth=0.8pt,arrowscale=2}
%% \psset{linewidth=0.5pt}

%\usepackage[customcolors]{hf-tikz}

% for subnodes in trees
% is in hu-beamer-...
%% \usepackage{tcolorbox}
%% \tcbuselibrary{skins}
%% \newtcbox{\mybox}[1][]{empty,shrink tight,nobeforeafter,on line,before upper=\vphantom{gM},remember as=#1,top=2pt,bottom=2pt}


%\beamertemplateemptyfootbar

%\hypersetup{pdfauthor={Stefan Müller (unter Benutzung von Material von Peter Gallmann, FSU Jena)}}
%\hypersetup{pdftitle={Head-Driven Phrase Structure Grammar für das Deutsche}}


%\subject{Generative Grammatik für das Deutsche}


%\beamerdefaultoverlayspecification{<+->}


%\mode<beamer>{\beamertemplatebackfindforwardnavigationsymbolshorizontal}

%% -*- coding:utf-8 -*-
\newcommand{\danish}{\jambox{(\ili{Danish})}}
\newcommand{\dutch}{\jambox{(\ili{Dutch})}}
\newcommand{\english}{\jambox{(\ili{English})}}
\newcommand{\german}{\jambox{(\ili{German})}}
\newcommand{\yiddish}{\jambox{(\ili{Yiddish})}}
\newcommand{\icelandic}{\jambox{(\ili{Islandic})}}


\let\mc=\multicolumn


\newcommand{\sigle}[1]{\tiny{#1}}

\newcommand{\ili}[1]{#1}


% Dateien in geteilte-Folien werden in verschiedenen Veranstaltungen benutzt.
% Was genau angezeigt wird, wird über toggles gesteuert.
\newtoggle{psgbegriffe}\togglefalse{psgbegriffe}

\newtoggle{hpsgvorlesung}\togglefalse{hpsgvorlesung}
\newtoggle{gb-intro}\togglefalse{gb-intro}
%\newtoggle{syntaxvorlesungen}\toggletrue{syntaxvorlesungen}

%\newtoggle{konstituentenprobleme}\toggletrue{konstituentenprobleme}
%\newtoggle{konstituentenprobleme-hinweis}\togglefalse{konstituentenprobleme-hinweis}
\newtoggle{einfsprachwiss-exclude}\toggletrue{einfsprachwiss-exclude}
\newtoggle{einfsprachwiss-include}\togglefalse{einfsprachwiss-include}



% To center and ignore the bar at the node in trees. I use the \rlap directly to avoid confusion
% with the AVM \1.
%\newcommand\1{\rlap{$'$}}


\title{Strukturen der germanischen Sprachen}

%\includeonly{organisatorisches-germanisch-vl}
%\includeonly{germanisch-phaenomene}
%\includeonly{germanisch-valenz-scrambling}
%\includeonly{germanisch-verbalkomplex}
%\includeonly{germanisch-verbstellung}
%\includeonly{germanisch-extraction}
%\includeonly{germanisch-mehr-vf}
%\includeonly{germanisch-subjekt}
%\includeonly{germanisch-passiv}
%\includeonly{germanisch-expletives}

\begin{document} 
\hypersetup{bookmarksopen=false}

\exewidth{(35)}

%%% -*- coding:utf-8 -*-
\section{Organisatorisches}

\frame{
\frametitle{Organisatorisches}


\begin{itemize}
%% \item alle Teilnehmer bitte Mail in Maillisten eintragen\\
%%       zugänglich von \url{http://www.cl.uni-bremen.de/~stefan/Lehre/HPSG/}
%%   \begin{itemize}
%%   \item Mail-Liste für alle Linguistik-Studierenden
%%   \item Mail-Liste für alle Teilnehmer dieser Veranstaltung
%%   \end{itemize}
\item Bitte bei moodle anmelden (gibt kein Passwort)
\pause
\item Telefon und Sprechzeiten siehe: \url{https://hpsg.hu-berlin.de/~stefan/}
\pause
\item Beschwerden, Verbesserungsvorschläge:
      \begin{itemize}
      \item mündlich
      \item per Mail oder 
      \item anonym über das Web:\\
            \url{https://hpsg.hu-berlin.de/~stefan/Lehre/}
      \end{itemize}
\item Bitte unbedingt Mail-Regeln beachten!\\
\url{https://hpsg.hu-berlin.de/~stefan/Lehre/mailregeln.html}
\end{itemize}
}

\frame{
\frametitle{Materialien}


\begin{itemize}

\item Information zur Vorlesung:\\
\url{https://hpsg.hu-berlin.de/~stefan/Lehre/Germanisch/}

\item Wiederholung/Grundlagen: \citew[Kapitel~1--2]{MuellerGTBuch2},\\
      \citew{GrundkursReader}
\item Einführungsbuch zur HPSG: \citew{MuellerLehrbuch3}
    
\end{itemize}
}

\frame{
\frametitle{Vorgehen}


\begin{itemize}
\item Handouts ausdrucken, immer mitbringen und persönliche Anmerkungen einarbeiten
\item Veranstaltungen vorbereiten
\item Veranstaltungen unbedingt nacharbeiten!
      \begin{itemize}
      \item Kontrollfragen
      \item Übungsaufgaben
      \end{itemize}
\item Fragen!
\end{itemize}
}


\section{Leistungen}
\frame{
\frametitle{Leistungen}

BA Ling: Modul 3: Grammatik II: Der Satz\\
BA Deutsch: Modul 6 Wort und Satz

\begin{itemize}
      \item Studiengang BA Germanistische Linguistik: Klausur in der letzten Woche im Vorlesungsraum
        bzw. dann im zweiten Prüfungszeitraum
      \item für alle (freiwillig): 
        \begin{itemize}
        \item kleine Tests zur Vertiefung
        \item zwei Fragen zum zu lesenden Text
        \end{itemize}
%Hausarbeit (\url{https://hpsg.hu-berlin.de/~stefan/Lehre/hausarbeiten.html})
\end{itemize}

Ideale Zeitaufteilung:

\begin{tabular}{@{}lr@{~}l}
Präsenzstudium Vorlesung  & 25 h \\
Vor- und Nachbereitung    & 35 h & (35/15 = 2 h 20 min für jede Sitzung)\\

Klausurvorbereitung       & 90 h & (90h/15 = 6h)
\end{tabular}

Das Modul entspricht 9 bzw.\ 8 Leistungspunkten.


}


%\input ../../plagiat.tex

\section{Ziele der Veranstaltung}


\frame{
\frametitle{Ziele}


% \begin{itemize}[<+->]
% \item Vermittlung grundlegender Vorstellungen über Grammatik
% \item Vorstellung zweier Grammatiktheorien und deren Herangehensweisen
% \item Vergleich der Sprachen Deutsch, Dänisch, Niederländisch
% \end{itemize}

\begin{itemize}
\item Überblick über die germanischen Sprachen
\item Detailliertere sprachvergleichende Diskussion ausgewählter syntaktischer Phänomene
\end{itemize}

Lehramtsrelevante Teilziele
\begin{itemize}
\item Wiederholung und Festigung grundlegender Begriffe:

\begin{itemize}
\item Wortarten
\item Kasus und andere morphosyntaktische Merkmale
\item Grammatische Funktionen
\item syntaktische Struktur des Deutschen und der Nachbarsprachen
      
      Hauptsätze/Nebensätze/Fragen

\item Valenz (Unterscheidung Argument/Adjunkt)
\item Aktiv/Passiv
\end{itemize}
\item Ansonsten: Blick über den Schul-Tellerrand
\end{itemize}


}

%\include{germanisch-ueberblick}
 \include{germanisch-phaenomene}
% \include{germanisch-psg}
% \include{germanisch-valenz-scrambling}
% \include{germanisch-adjunkte}
% \include{germanisch-verbalkomplex}
% \include{germanisch-verbstellung}
% \include{germanisch-extraction}
% %% \include{germanisch-mehr-vf}

% \include{germanisch-passiv}
% \include{germanisch-expletives}

%\include{germanisch-syntax}



\appendix
% muß immer geladen werden, wegen Referenzen
%\section<presentation>*{Literatur}

\section{Literaturverzeichnis}

%% \frame{
%%   \frametitle<presentation>{Literatur}
  
%%   \beamertemplatebookbibitems

%% \bibliography{biblio}
%% \bibliographystyle{natbib.myfullname}
%% }
%\beamertemplatebookbibitems

% there seems to be a bug. These things are only set on the first literature slide
%
% The bug is still there, but the fix does not work any longer. 
%\iflanguage{german}{\renewcommand{\refname}{Literaturverzeichnis}} % should be set automatically ???
\iflanguage{german}{\def\insertsectionhead{Literaturverzeichnis}}{\def\insertsectionhead{\refname}}
%\def\insertsectionhead{\refname}
%\def\insertsubsectionhead{}

\huberlinjustbarfootline

\ifpdf
\else
\ifxetex
\else
\let\url=\burl
\fi
\fi
\begin{multicols}{2}
{\renewcommand*{\bibfont}{\tiny}

%\beamertemplatearticlebibitems

%\bibliography{bib-abbr,biblio,crossrefs}
%\bibliographystyle{unified}

% biblatex

%\addbibresource{bib-abbr.bib,biblio.bib}

% no book icon please
\setbeamertemplate{bibliography item}{}

%\printbibliography
\printbibliography[heading=bibliography,notkeyword=this] 

}
\end{multicols}




\end{document}


% Local variables:
% mode: lazy-lock
% End:




Fragen:

1. Kann die Definition für strukturellen und lexikalischen Kasus für andere Sprachen, auch außerhalb der germanischen Sprachen, übernommen werden oder müsste man sie von Grund auf neu definieren? 

2. Wie könnte man feststellen, ob ein Kasus einer Sprache, die noch nicht sehr gut beschrieben ist, ein struktureller oder lexikalischer Kasus ist? 

3. Welche Schritte sind notwendig um das Kasussystem einer Sprache zu definieren. Wie geht man vor? 

4. Wie geht man als Student damit um, wenn bezüglich eines Merkmals oder linguistischen Phänomens eine Kontroverse Meinung herrscht, wie z. B. bei der Einordnung des Dativs (lexikalisch/strukturell).




1.	Im Satz (85) ist ein Subjekt Haraldi oder Sigga?
Laut der Regel in Isländisch steht Subjekt direkt nach dem finiten Verb, egal in welchem Kasus (Haraldi in Dat.). Aber in diesem Satz gibt es auch eine Nominalgruppe  (Sigga).
2.	Wird Objektskasus nur in den Sätzen mit Passiv beobachtet?
  



1. Wenn die HPSG-Theorie eine lexikonbasierte Theorie ist und die
Syntaxstruktur vom Verb abhängen sollte, warum sollte man unterscheiden
zwischen strukturellem und lexikalischem Kasus?

2. Sind die semantische Kasus hier eine Subkategorie wie struktureller und
lexikalischer Kasus oder eine Tiefstruktur unter den syntaktischen Kasus?
Gibt es Position oder Beschreibung für semantische Kasus in Fillmores Case
Grammar wie agentive, objective, benefactive, locative, usw. in HPSG? 3.
Da ich Deutsch als Fremdsprache lernen, bedeutet die Grammatik für mich
die Regeln zu befolgen. Eine der Regeln ist, dass Objektive in der
folgenden Reihenfolge angeordnet werden: Akkusativpronomen < Dativpronomen
< Dativnomen < Akkusativnomen.
Also ist das Scrambling auf Seite 40 eine ungrammatische Ausnahme der
Regel oder ein häufiges Phänomen, oder ist die Regel nur eine
Standardvariante? Was ist das Ziel von HPSG und die andere
Grammatiktheorien speziell? Wie unterscheidet sich die Grammatik für
formale Beschreibung von Sprachlehre?











