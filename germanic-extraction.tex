%% -*- coding:utf-8 -*-
%\subsection{Verbzweitstellung und Voranstellung im Englischen}
\subsection{Verb second and Verb first position in English}

\frame{
%\frametitle{Extraktion}
\frametitle{Extraction}

\begin{itemize}
%\item Auch in Sprachen mit relativ fester Konstituentenstellung ist es mitunter möglich, Konstituenten umzustellen:
\item Even in languages with a relatively fixed constituent position, it is sometimes possible to rearrange constituents:

\eal
\ex This book, I read yesterday.
\ex Yesterday, I read this book.
\zl

\end{itemize}

}

\frame{
	%\frametitle{Extraktion}
	\frametitle{Extraction}

\smallframe

\begin{itemize}
%\item Die germanischen (V2-)Sprachen stellen irgendeine Konstituente vor das finite Verb:
\item The Germanic (V2) languages place some constituent before the finite verb:

\eal
\ex 
\gll \alert{Ich} habe das Buch gestern gelesen.\\
		I have the book yesterday read\\\german
\glt `I have read the book yesterday.'
\ex 
\gll \alert{Das} \alert{Buch} habe ich gestern gelesen. \\
		the book have I yesterday read\\
\ex 
\gll \alert{Gestern} habe ich das Buch gelesen. \\
		 yesterday have I the book read\\
\ex 
\gll \alert{Gelesen} habe ich das Buch gestern,
    gekauft hatte ich es aber schon vor einem Monat. \\
    	read have I the book yesterday bought had I it but yet before a month\\
\glt `I read the book yesterday, but I bought it last month already.'
\ex 
\gll \alert{Das} \alert{Buch} \alert{gelesen} habe ich gestern. \\
		the book read    have I yesterday\\
\zl

\end{itemize}

}


\frame{
%\frametitle{Extraktion ist nicht satzgebunden}
\frametitle{Extraction is not clause-bounded}

\begin{itemize}
%\item Extraktion kann über Satzgrenzen hinweggehen:
\item Extraction may cross one or several clause boundaries:
\eal
\ex Chris, David saw.
\ex Chris, we think that David saw.
\ex Chris, we think Anna claims that David saw.
\zl

\end{itemize}
}

\frame[shrink=20]{
	%\frametitle{Extraktion ist nicht satzgebunden}
	\frametitle{Extraction is not clause-bounded}

\begin{itemize}

%\item Im Deutschen wohl eher in den süddeutschen Varietäten, aber:
\item In German probably more in the southern German varieties, but:

\eal
\label{bsp-Fernabhaengigkeit}
\ex \label{bsp-um-zwei-millionen}
\gll {}[Um zwei Millionen Mark]$_i$ soll er versucht haben,
{}[eine Versicherung \_$_i$ zu betrügen]. \\ %\footnote{taz, 04.05.2001, S.\,20.} \\
         \spacebr{}around two million Deutsche.Marks should he tried have \spacebr{}an insurance.company {} to deceive \\
\glt `He apparently tried to cheat an insurance company out of two million Deutsche Marks.' \hfill {\footnotesize[taz, 04.05.2001, S.\,20]}

\ex
\gll "`Wer$_i$, glaubt\iw{glauben} er, daß er \_$_i$ ist?"' erregte sich ein Politiker vom Nil. \\ %\footnote{Spiegel, 8/1999, S.\,18.} \\
		 \hphantom{"`}who believes he that he {} is was.upset \textsc{refl} a politician from.the Nile\\
\glt `A politician from the Nile was upset: ``Who does he believe he is?''.' \hfill {\footnotesize[Spiegel, 8/1999, S.\,18]}

\ex\label{ex-wen-glaubst-du-dass}
\gll 	Wen$_i$ glaubst du, daß ich \_$_i$ gesehen habe. \\ %\footnote{\citew[\page84]{Scherpenisse86a}.} \\
			who believes you that I {} seen have\\
\glt `Who do you believe that I have seen?'  \hfill {\footnotesize [\citew[\page84]{Scherpenisse86a}]}

\ex 
\gll {}[Gegen ihn]$_i$ falle es den Republikanern hingegen schwerer,
    {}[[Angriffe \_$_i$] zu lancieren]. \\ %\footnote{taz, 08.02.2008, S.\,9.} \\
    	 {}\spacebr{}against him fall it the Republicans however more.difficult \hphantom{[[}attacks {} to launch\\
\glt `It is, however, more difficult for the Republicans to launch attacks against him.'  \hfill {\footnotesize [taz, 08.02.2008, S.\,9]}
\zl

\end{itemize}
}

\frame{
%\frametitle{Weitergabe von Information im Baum}
\frametitle{Passing on information in the tree}


%% \centerfit{\begin{tikzpicture}[
%% level 1+/.style={level distance=3\baselineskip},
%% frontier/.style={distance from root=12\baselineskip},
%% connect/.style={semithick,<->,color=green}]
%% %\draw (-3,-5) to[grid with coordinates] (4,0);
%% \Tree[.S
%%         [.\node (NP) {NP}; Chris ]
%%         [.\node (S/NP) {S/NP};
%%           [.NP David ] 
%%           [.\node (VP/NP) {VP/NP};  
%%             [.V saw ]
%%             [.\node (NP/NP) {NP/NP}; \trace{} ] ] ] ]
%% \draw[connect] (NP/NP.north east) [bend right] to (VP/NP.south east);
%% \draw[connect] (VP/NP.north east) [bend right] to (S/NP.south east);
%% \draw[connect] (S/NP.north east) [bend right] to (NP);
%% \end{tikzpicture}}

\medskip
\centerline{%
\begin{forest}
sm edges
[S
  [\gruen<3>{NP} [Chris] ]
  [S/\gruen<2>{NP} 
    [NP [David] ] 
    [VP/\gruen<2>{NP}  
      [V [saw] ]
      [NP/\gruen<1>{NP} [\trace] ] ] ] ]
%% todo to do
%% \draw[connect] (NP/NP.north east) [bend right] to (VP/NP.south east);
%% \draw[connect] (VP/NP.north east) [bend right] to (S/NP.south east);
%% \draw[connect] (S/NP.north east) [bend right] to (NP);
\end{forest}}

\begin{itemize}
%\item Hinten fehlt das Objekt (Lücke, Spur): NP/NP
\item The object is missing in the back (gap, trace): NP/NP
\pause
%\item Information über das fehlende Objekt wird an VP- und S-Knoten repräsentiert.
\item Information about the missing object is represented at VP and S nodes
\pause
%\item Fehlende NP steht vorn. (der so genannte Füller)
\item Missing NP is fronted (the so-called filler)
\end{itemize}

}


\frame{
%\frametitle{Weitergabe im Baum über größere Entfernungen}
\frametitle{Passing on in the tree over longer distances}

%% \centerfit{\scalebox{.7}{%
%% \begin{tikzpicture}[
%% level 1+/.style={level distance=3\baselineskip},
%% frontier/.style={distance from root=21\baselineskip},
%% connect/.style={semithick,<->,color=green}]
%% %\draw (-3,-5) to[grid with coordinates] (4,0);
%% \Tree[.S
%%         [.\node (NP) {NP}; Chris ]
%%         [.\node (S/NP1) {S/NP};
%%           [.NP we ] 
%%           [.\node (VP/NP1) {VP/NP};  
%%             [.V think ]
%%               [.\node (CP/NP) {CP/NP};
%%                 [.C that ]
%%                 [.\node (S/NP) {S/NP};
%%                   [.NP David ] 
%%                   [.\node (VP/NP) {VP/NP};  
%%                     [.V saw ]
%%                     [.\node (NP/NP) {NP/NP}; \trace{} ] ] ] ] ] ] ]
%% \draw[connect] (NP/NP.north east)  [bend right] to (VP/NP.south east);
%% \draw[connect] (VP/NP.north east)  [bend right] to (S/NP.south east);
%% \draw[connect] (S/NP.north east)   [bend right] to (CP/NP.south east);
%% \draw[connect] (CP/NP.north east)  [bend right] to (VP/NP1.south east);
%% \draw[connect] (VP/NP1.north east) [bend right] to (S/NP1.south east);
%% \draw[connect] (S/NP1.north east)  [bend right] to (NP);

%% \end{tikzpicture}}}


\centerfit{\scalebox{.8}{%
\begin{forest}
sm edges
[S
  [\gruen{NP} [Chris] ]
  [S/\gruen{NP}
    [NP [we] ] 
    [VP/\gruen{NP}  
       [V [think] ]
       [CP/\gruen{NP}
         [C [that] ]
         [S/\gruen{NP}
            [NP [David] ] 
            [VP/\gruen{NP}  
               [V [saw] ]
               [NP/\gruen{NP} [\trace ] ] ] ] ] ] ] ]
%% \draw[connect] (NP/NP.north east)  [bend right] to (VP/NP.south east);
%% \draw[connect] (VP/NP.north east)  [bend right] to (S/NP.south east);
%% \draw[connect] (S/NP.north east)   [bend right] to (CP/NP.south east);
%% \draw[connect] (CP/NP.north east)  [bend right] to (VP/NP1.south east);
%% \draw[connect] (VP/NP1.north east) [bend right] to (S/NP1.south east);
%% \draw[connect] (S/NP1.north east)  [bend right] to (NP);
\end{forest}}}


}

\frame{
%\frametitle{Schema für Füller-Kopf-Strukturen}
\frametitle{Filler-Head Schema}

\vfill
\centerline{
\begin{forest}
[{H}
  [\ibox{1}]
  [H/\ibox{1}]]
\end{forest}
}

\vfill


}


\frame[shrink]{
%\frametitle{Extraktion + Verbumstellung = V2: Dänisch (SVO)}
\frametitle{Extraction + verb movement = V2: German (SOV)}


%% \centerfit{\begin{tikzpicture}[
%% level 1+/.style={level distance=3\baselineskip},
%% frontier/.style={distance from root=15\baselineskip},
%% connect/.style={semithick,<->,color=green}]
%% %\draw (-3,-5) to[grid with coordinates] (4,0);
%% \Tree[.S
%%         [.\node (NP) {NP}; \edge[roof]; {[das Buch]$_i$} ]
%%         [.\node (S/NP) {S/NP};
%%           [.{V \sliste{ S/\!/V }} 
%%             [.V liest$_k$ ] ]
%%            [.\node (S//V/NP) {S$/\!/$V/NP};
%%              [.\node (NP/NP) {NP/NP}; \trace$_i${} ]
%%              [.{V$'$$\!/\!/$V}
%%                [.NP Conny ]
%%                [.{V$\!/\!/$V} \_$_k$ ] ] ] ] ] ]
%% \draw[connect] (NP/NP) [bend right] to (S//V/NP.south east);
%% \draw[connect] (S//V/NP.north east) [bend right] to (S/NP.east);
%% \draw[connect] (S/NP.north east) [bend right] to (NP);
%% \end{tikzpicture}}


\centerfit{%
\begin{forest}
sm edges
[S
  [\gruen<2>{NP$_i$} [das Buch;the book, roof] ]
  [S/\gruen<2>{NP}
     [V \sliste{ S\gruen<1>{$/\!/$V} } 
        [\gruen<1>{V} [liest$_j$;reads] ] ]
     [S\gruen<1>{$/\!/$V}\!\gruen<2>{/NP}
        [NP\gruen<2>{/NP} [\trace$_i$] ]
        [V$'$\gruen<1>{$\!/\!/$V}
           [NP [Conny;Conny] ]
           [V\gruen<1>{$\!/\!/$V} [\_$_j$] ] ] ] ] ] ]
%% \draw[connect] (NP/NP) [bend right] to (S//V\!/NP.south east);
%% \draw[connect] (S//V\!/NP.north east) [bend right] to (S/NP.east);
%% \draw[connect] (S/NP.north east) [bend right] to (NP);
\end{forest}}


}




\frame{
%\frametitle{Extraktion + Verbumstellung = V2: Dänisch (SVO)}
\frametitle{Extraction + verb movement = V2: Danish (SVO)}


\centerfit{\scalebox{.9}{
\begin{forest}
sm edges
[S
   [NP$_i$ [bogen;book.\textsc{def} ] ]
      [S/NP
         [V \sliste{ S$/\!/$V }
           [V [læser$_j$;reads] ] ]
           [S$/\!/$V\!/NP
             [NP [Conny;Conny] ]
             [VP$\!/\!/$V$\!/$NP
               [V$\!/\!/$V  [\_$_j$] ]
               [NP/NP [\trace$_i$ ] ] ] ] ] ] 
%% \draw[connect] (NP/NP.north east) [bend right] to (V/V.south east);
%% \draw[connect] (V/V.north east) [bend right] to (S//V\!/NP.south east);
%% \draw[connect] (S//V\!/NP.north east) [bend right] to (S/NP.east);
%% \draw[connect] (S/NP.north east) [bend right] to (NP);
\end{forest}
}}

%Das Deutsche unterscheidet sich vom Dänischen in der OV/VO-Stellung und der damit verbundenen
%VP-Bildung, sonst ist bzgl.\ V2 alles gleich.
German differs from Danish in the OV/VO position and the associated
VP formation, otherwise everything is the same with regard to \ V2.

}





%% \frame{
%% \frametitle{Weitergabe von Information im Baum}

%% \centerfit{\begin{tikzpicture}[
%% level 1+/.style={level distance=3\baselineskip},
%% frontier/.style={distance from root=15\baselineskip},
%% connect/.style={semithick,<->,color=green}]
%% %\draw (-3,-5) to[grid with coordinates] (4,0);
%% \Tree[.S
%%         [.\node (NP) {NP}; wen ]
%%         [.\node (S/NP) {S/NP};
%%           [.{V \sliste{ S/\!/V }} 
%%             [.V glaubst$_k$ ] ]
%%            [.\node (S//V\!/NP) {S$/\!/$V\!/NP};
%%              [.\node (NP/NP) {NP/NP}; \trace{} ]
%%              [.{V$'$$\!/\!/$V}
%%                [.NP Conny ]
%%                [.{V$\!/\!/$V} \_$_k$ ] ] ] ] ] ]
%% \draw[connect] (NP/NP) [bend right] to (S//V\!/NP.south east);
%% \draw[connect] (S//V\!/NP.north east) [bend right] to (S/NP.east);
%% \draw[connect] (S/NP.north east) [bend right] to (NP);
%% \end{tikzpicture}}

%% }

%\subsubsection{Vergleich Englisch, Dänisch, Deutsch}
\subsubsection{Comparison English, Danish, German}


\frame{
%\frametitle{Vergleich Englisch, Dänisch, Deutsch}
\frametitle{Comparison English, Danish, German}

%Die Sätze sehen gleich aus, haben aber eine ganz andere Struktur:
The sentences look the same, but have a completely different structure:

\eal
\ex Conny reads a book.
\ex Conny læser en bog.
\ex Conny liest ein Buch.
\zl

}


\frame{
%\frametitle{Strukturen für Englisch, Dänisch, Deutsch}
\frametitle{Structures for English, Danish, German}


~\vfill
\scalebox{.8}{%
\begin{forest}
sm edges
[S
  [NP [Conny]]
  [VP
    [V [reads]]
    [NP [a book,baseline,roof]]]]
\end{forest}}
\hfill
\visible<2->{\scalebox{.8}{%
\begin{forest}
sm edges
[S
   [NP$_i$ [Conny;Conny] ]
      [S/NP
         [V \sliste{ S$/\!/$V }
           [V [læser$_j$;reads] ] ]
           [S$/\!/$V\!/NP
             [NP/NP [\trace$_i$ ] ]
             [VP$\!/\!/$V
               [V$\!/\!/$V  [\_$_j$] ]
               [NP [en bog;a book,baseline,roof] ] ] ] ] ] 
%% \draw[connect] (NP/NP.north east) [bend right] to (V/V.south east);
%% \draw[connect] (V/V.north east) [bend right] to (S//V\!/NP.south east);
%% \draw[connect] (S//V\!/NP.north east) [bend right] to (S/NP.east);
%% \draw[connect] (S/NP.north east) [bend right] to (NP);
\end{forest}}}
\hfill
\visible<3>{\scalebox{.8}{%
\begin{forest}
sm edges
[S
   [NP$_i$ [Conny;Conny] ]
      [S/NP
         [V \sliste{ S$/\!/$V }
           [V [liest$_j$;reads] ] ]
           [S$/\!/$V\!/NP
             [NP/NP [\trace$_i$ ] ]
             [V$'$$\!/\!/$V
               [NP [ein Buch;a book,baseline,roof] ] 
               [V$\!/\!/$V  [\_$_j$] ] ] ] ] ] 
\end{forest}}}

\vfill

}


\frame{
%\frametitle{V2 in den germanischen Sprachen}
\frametitle{V2 in the Germanic languages}

\smallexamples
\smallframe
 

\begin{itemize}
%\item Alle germanischen Sprachen (außer Englisch) haben V2-Sätze mit Verbvoranstellung +
%  Voranstellung einer Konstituente.
\item All Germanic languages (except English) have V2 sentences with verb fronting +
fronting of a constituent.
\pause
%\item Sätze immer gleich analysiert, auch wenn Subjekt das Vorfeld füllt.
\item Sentences are always analyzed in the same way, even if the subject fills the antecedent.
\pause
%\item V1-Sätze sind nicht unbedingt Fragen:
\item V1 sentences are not necessarily questions:

\eal
\ex 
\gll Gibt er ihm das Buch?\\
		gives he him the book\\
\glt `Does he give him the book?'

\ex 
\gll Gib mir das Buch!\\
		give me the book\\
\zl

\end{itemize}

}

\frame{
	%\frametitle{V2 in den germanischen Sprachen}
	\frametitle{V2 in the Germanic languages}
	
	\smallexamples
	\smallframe

\begin{itemize}
%\item V2-Sätze nicht unbedingt Aussagen:
\item V2 sentences not necessarily statements:

\eal
\ex 
\gll Wem gibt er das Buch?\\
		who gives he the book\\
\glt `Whom does he give the book to?'

\ex 
\gll Jetzt gib ihm das Buch!\\
		now give him the book\\
\glt `Give him the book now!'

\ex 
\gll Jetzt gibt er ihm das Buch.\\
		now gives he him the book\\
\glt `He gives him the book now.'
\zl

\end{itemize}

%\pause\pause\pause

}


%      <!-- Local IspellDict: en_US-w_accents -->
