%% -*- coding:utf-8 -*-

%%%%%%%%%%%%%%%%%%%%%%%%%%%%%%%%%%%%%%%%%%%%%%%%%%%%%%%%%
%%   $RCSfile: goeteborg2004-include.tex,v $
%%  $Revision: 1.4 $
%%      $Date: 2004/08/25 22:04:18 $
%%     Author: Stefan Mueller (DFKI)
%%    Purpose: 
%%   Language: LaTeX
%%%%%%%%%%%%%%%%%%%%%%%%%%%%%%%%%%%%%%%%%%%%%%%%%%%%%%%%%
%% $Log: goeteborg2004-include.tex,v $
%% Revision 1.4  2004/08/25 22:04:18  stefan
%% erste goeteborg-Version
%%
%% Revision 1.3  2004/07/20 12:41:35  stefan
%% *** empty log message ***
%%
%% Revision 1.2  2004/05/11 21:23:07  stefan
%% lange Version
%%
%% Revision 1.1  2004/05/09 11:51:09  stefan
%% Version wie Leipzig
%%
%% Revision 1.2  2004/05/09 11:43:57  stefan
%% endgültige Fassung
%%
%% Revision 1.1  2004/05/07 15:01:41  stefan
%% Leipziger Version mit Fix einer falschen Lexikonregel
%%
%% Revision 1.1  2003/09/17 10:19:57  stefan
%% Iniitial Version
%%
%%%%%%%%%%%%%%%%%%%%%%%%%%%%%%%%%%%%%%%%%%%%%%%%%%%%%%%%%





\lecture{Das Phänomen}{phänomen}


\subsection{Exkurs: Verbstellung und scheinbar mehrfache Vorfeldbesetzung}

\subsubsection{Subjekt und Adverbiale}
\frame{
\frametitle{Subjekt und Adverbiale}

\eal
\label{bsp-mehrfach-vf-subjekt}%
\ex {}[\blau{Richtig}] [\blau{Geld}] wird aber nur im Briefgeschäft verdient.\label{bsp-richtig-geld}\footnote{
        taz, 28./29.10.2000, S.\,5
}
\ex {}[\blau{Alle Träume}] [\blau{gleichzeitig}]  lassen sich nur  selten verwirklichen.\footnote{Broschüre der Berliner Sparkasse, 1/1999}

\ex {}[\blau{Weiterhin}] [\blau{Hochbetrieb}] herrscht am Innsbrucker Eisoval.\footnote{
 COSMAS. \sigle{I00/JAN.00911}\\[-.5\baselineskip]
}
\zl


}

\subsubsection{Akkusativobjekt und PP}

\exewidth{(44)}

\frame{
\DATaframe
\parskip0pt
\frametitle{Akkusativobjekt und PP}
\savespace\smallexamples
\exewidth{(10)}
\ea
{} [\blau{Nichts}] [\blau{mit  derartigen     Entstehungstheorien}] hat es natürlich zu tun, wenn \ldots (K. Fleischmann, nach \citew{vdVelde78a})
\z
%% \ea {}[\blau{Wenig}] [\blau{mit Sprachgeschichte}] hat der dritte Beitrag in dieser Rubrik zu tun, [\ldots] (Zeitschrift für Dialektologie und Linguistik, LXIX, 3/2002)
%% \z
%% \ea {}[Gar nichts mehr] [mit dem Tabakkonzern] hat Jan Philipp Reemtsma zu tun, der das Unternehmen 1978
%%     im Alter von 28 Jahren erbte und 1980 für 300 Millionen Mark (nach Steuern) an Herz verkaufte. (taz, 16.01.2003)
%% \z
%% \ea Produktiv ist auch das Modell mit komplexer Basis (meist Kompositum), 
%%       die Zugehörigkeit von Personen zu einem Betrieb o.\,ä.\ bezeichnend [\ldots]
%%       {}[\blau{Personen}] [\blau{nach der Zugehörigkeit}] bezeichnen auch \emph{Gesellschafter},
%%       \emph{Gewerkschafter} [\ldots]. (Im Haupttext von \citep[S.\,155]{FB95a})
%% \z
%% \ea {}[\blau{Großes Gewicht}] [\blau{für die Geschworenen}] hatte ein auf"|gezeichnetes Telefongespräch des Scheichs
%%       mit den Bombenlegern des World Trade Centers (WTC). (taz, 04.10.1995, S.\,8)
%% \z
%\exewidth{(8)}
\ea {}[\blau{Zum zweiten Mal}] [\blau{die Weltmeisterschaft}]  errang Clark 1965. % \ldots
\citep{Benes71}%
\z
\ea
{}[\blau{Die Kinder}] [\blau{nach Stuttgart}] sollst du bringen. \citep[S.\,81]{Engel70a}
\z
}

\subsubsection{Dativobjekt und PP}

\frame{
\frametitle{Dativobjekt und PP}

\ea
{}[\blau{Der Universität}] [\blau{zum Jubiläum}] gratulierte auch Bundesminister Dorothee Wilms, die in den fünfziger Jahren in
Köln studiert hatte. (Kölner Universitätsjournal, 1988, S.\,36, zitiert nach \citew[S.\,87]{Duerscheid89a})

\z

}

\subsubsection{Adjektiv und lokale oder direktionale PPen}

\frame{
\frametitle{Adjektiv und lokale oder direktionale PPen}

\ea
\blau{Einsam auf dem kleinen Bahnhof       im     Moor}\\
blieb  der lächelnde Junge zu\-rück. (Heinrich Böll, nach \citew{Benes71})
\z

\pause
%% \ex
%% {}[\blau{Einsam}] [\blau{am Eingang}] steht ein blitzblank neues Infoterminal und will über das neue Preissystem
%%     Auskunft geben. (taz berlin, 02./03.2002, S.\,33)
%% \pause
\ea 
{}[\blau{Trocken}] [\blau{durch   die Stadt}] kommt man am     Wochenende\\
auch mit  der BVG. (taz, 10.07.1998, S.\,22)
\z


\pause

Noch warm (Spiegel 4/2011, S.\,116): Adjektiv und Präpositionalobjekt:
\ea
{}[\blau{Offen}] [\blau{über ihren Burnout}] sprachen in jüngerer Zeit der Skispringer Sven
Hannawald, der Fernehkoch Tim Mälzer, die Publizistin Miriam Meckel und der Politiker Matthias Platzeck.
\z


}



\subsubsection{Temporale Präpositionalphrasen und Instrumentpräpositionalphrase}

\handoutframe{
\frametitle{Temporale PP und Instrument-PP}

\ea
{}[\blau{Zum letzten Mal}] [\blau{mit der Kurbel}] wurden gestern\\
die Bahnschranken an zwei Übergängen im Oberbergischen
Ründeroth geschlossen.\\
(Kölner Stadtanzeiger, 26.4.88, S.\,28, nach \citew[S.\,107]{Duerscheid89a})
\label{zum-letzten-mal}
\z

}

\subsubsection{Andere Muster}

\frame{

\frametitle{usw.}

Es gibt Belege für:
\begin{itemize}
%\item Instrumentpräpositionalphrase und direktionale Präpositionalphrasen
\item Adverbial gebrauchte Adjektive und direktionale/lokale PPen
\item Nominalphrasen in Kopulakonstruktionen und Adverbiale
\item Präpositionalphrasen in Kopulakonstruktionen und Adverbien
\item Prädikative Konjunktionalphrasen und Adverb
\item Direktionale Präpositionalphrasen und Adverbien
\item Lokale Präpositionalphrasen und Adverbien
\end{itemize}
in \citew{Mueller2003b} und Datensammlung auf:\\
\url{http://hpsg.fu-berlin.de/~stefan/Pub/mehr-vf-ds.html}


}

\frame{
\frametitlefit{Empirische Grundlagen (DFG MU 2822/1-1 und SFB 632, A6)}

\mode<beamer>{~}%
\mode<handout>{\medskip}
\vfill

\centerline{
\includegraphics[totalheight=0.6\textheight]{DFG-Projekt/empirie-tabelle}
}

\vfill

\begin{itemize}
\item Datensammlung (Felix Bildhauer):\\
      gegenwärtig $\textgreater$ 3200 Instanzen von MF mit Kontext
\item Annotiert mit syntaktischer Kategorie/Funktion und Informationsstruktur
\end{itemize}

\vfill\vfill

%\mode<beamer>{
~\\
%}

}

\lecture{Beschränkungen}{Beschränkungen}
\subsubsection{Beschränkungen}


\frame{

\frametitle{Beschränkungen}

\savespace
\smallexamples\small
\begin{itemize}
\item \citet{Fanselow93a}:
Vorangestellte Konstituenten müssen Satzgenossen sein.
\eal
\ex[]{
Ich glaube \gruen{dem Linguisten} nicht,
\rot{einen Nobelpreis} gewonnen zu haben.
}
\ex[*]{
\gruen{Dem Linguisten} \rot{einen Nobelpreis} glaube ich nicht gewonnen zu haben.
}
\pause
\ex[]{
Ich habe \gruen{den Mann} gebeten, den Brief \rot{in den Kasten} zu werfen.
}
\ex[*]{
\gruen{Den Mann} \rot{in den Kasten} habe ich gebeten, den Brief zu werfen.
}
\zl
\pause
\item Erklärung:\\ Konstituenten im \vf hängen von leerem Kopf ab oder modifizieren ihn.\\
Der leere Kopf steht in Beziehung zu einem Verb (\citealt{Fanselow93a};\\
\citealt{Hoberg97a})

\pause
\item empirische Überprüfung der Vorhersagen/Behauptungen:
\begin{itemize}
\item Phrasen im VF hängen vom selben Verb ab. \pause {\large\gruen \Checkmark}
\pause
\item Anordnung der Konstituenten wie im Mittelfeld \pause {\large\gruen \Checkmark}
\end{itemize}

\end{itemize}

}

\lecture{Bisherige Ansätze}{Bisherige Ansätze}

\subsubsection{Bisherige Ansätze}


\outline{

\begin{itemize}
\item Das Phänomen
\item \blau{Bisherige Ansätze}
\item Die Analyse
%\item Informationsstruktur
\item Zusammenfassung und zukünftige Arbeiten
\end{itemize}

}



% \subsubsection{Lötscher 1985, Eisenberg 1994, Hoberg 1997}
%
% \frame{
% \frametitle{\citew*{Loetscher85}, \citew{Eisenberg94a} und \citew{Hoberg97a}}
%
% \begin{itemize}
% \item<+-> \citew[S.\,412--413]{Eisenberg94a}, \citew[S.\,1634]{Hoberg97a}: nicht formalisiert
% \item<+-> \citew{Loetscher85}:
%       \begin{itemize}
%       \item Transformationen, erzeugen Abfolgen im Mittelfeld
%       \item<+-> Teile des Mittelfelds, die adjazent zum Verbalkomplex sind,\\
%             können allein oder mit Teilen des Verbalkomplexes vorangestellt werden.
%       \end{itemize}
% \item<+-> Ansatz mit gegenwärtigen Annahmen nicht kompatibel\\
%           zu aktuellen, ähnlichen Ansätzen \hyperlink{fanselow-gmueller}{später}
% \end{itemize}
% }

\subsubsubsection{Wunderlich (1984) und Dürscheid (1989)}

\frame{

\frametitle{\citew{Wunderlich84} und \citew{Duerscheid89a}}

\gblabelsep{.5em}

\begin{itemize}
\item<+-> \citet*{Wunderlich84}: PPen bilden eine komplexe PP im \vf. \nocite{Riemsdijk78a,Dowty79a}
\eal
\ex 
{}[\sub{PP} [\sub{PP} Zu ihren Eltern] [\sub{PP} nach Stuttgart]] ist sie gefahren.\label{ex-zu-ihren-eltern-nach}

\ex
{}[\sub{PP} [\sub{PP} Von München]    [\sub{PP} nach Hamburg]]   sind es 900 km.
\ex 
{}[\sub{PP} [\sub{PP} Durch den Park]  [\sub{PP} zum Bahnhof]]    sind sie gefahren.
\zl
\item<+-> zweite PP modifiziert die erste\\
      möglich, wenn PPs dieselbe semantische Rolle füllen\\
      PPs in (\mex{0}a) sind {\sc goal} einer Bewegung
\item<+-> Wunderlich subsumiert verschiedene thematische Rollen der PPen in (\mex{0}b--c) (Source, Path und Goal einer Bewegung)
unter eine, nämlich die Lokalisierung einer Bewegung
\item<+-> Dürscheid nimmt solch eine Analyse auch für hier diskutierte Fälle an.
\end{itemize}
}

\frame{
\frametitle{Probleme des Wunderlichschen Ansatzes}

\begin{itemize}
\item Lässt sich nicht für andere Fälle in \fromto{\ref{bsp-mehrfach-vf-subjekt}}{\ref{zum-letzten-mal}} verwenden:
\ea
{}[\blau{Alle Träume}] [\blau{gleichzeitig}]  lassen sich nur  selten verwirklichen.\footnote{Broschüre der Berliner Sparkasse, 1/1999}
\z
\pause
\begin{itemize}
\item Was ist die Kategorie der Projektion im \Vf?
\pause
\item Warum sollten semantische Rollen verschiedener Konstituenten zusammengefaßt werden?
\pause
\item Was ist mit Adjunkten? Adjunkte bekommen keine Rollen.
\end{itemize}
\end{itemize}
}






\subsubsubsection{Müller (2000)}

%% \renewcommand{\node}[2]{\pgfnodebox{#1}[virtual]{\pgfxy(3,0.5)}{#2}{\nodemargin}{\nodemargin}
%% \let\nodeconnect\pgfnodeconnline

\frame{

\frametitlefit{Mehrere Konstituenten $\stackrel{?}{=}$ mehrere Extraktionen \citep{Mueller2000d}}


\begin{itemize}
\item Idee: Beide Konstituenten sind Teile von Fernabhängigkeiten.
\item Die erste wird als Füller in einer Füller-Lücke-Konstruktion abgebunden.
\item Die zweite wird oberhalb dieser Konstruktion ebenfalls in einer Füller-Lücke-Konstruktion abgebunden.

\bigskip

\nodemargin5pt
\begin{tabular}[t]{llll}
                 &&&\node{s}{S[{\sc slash} \nliste{}]}\\[4ex]
\node{c1}{C$_1$}   &&&\node{s-c1}{S[{\sc slash} \nliste{C$_1$}]}\\[4ex]
&\node{c2}{C$_2$}  &&\node{s-c1-c2}{S[{\sc slash} \nliste{C$_1$, C$_2$}]}\\[4ex]
%&&\node{c3}{C$_3$} &\node{s-c1-c2-c3}{S[{\sc slash} \liste{C$_1$, C$_2$, C$_3$}]}\\
\end{tabular}
\nodeconnect{s}{c1}\nodeconnect{s}{s-c1}%
\nodeconnect{s-c1}{c2}\nodeconnect{s-c1}{s-c1-c2}%
%}

% \step{
% \item In a linearization domain approach both can be seralized in the same domain
%       and treated as members of the \vf.
% \item Thematic conditions on the multiple constituents in the \vf rule out ungrammatical
%       combinations in the \vf.
% }
\end{itemize}


}

\frame{

\frametitle{Probleme mit der Monotonie und mit Idioms}

\smallframe
\begin{itemize}
\item einige Teile von manchen Idiomen können nur zusammen vorangestellt werden:
\eal
\ex[]{
{}[\blau{Öl}] [\blau{ins Feuer}]  goß   gestern   das Rote-Khmer-Radio: [\ldots]\footnote{taz, 18.06.1997, S.\,8}
}
\ex[*]{
{}[ins Feuer]  goß   gestern   das Rote-Khmer-Radio Öl: [\ldots]
}
\zl
\pause
\eal
\ex[]{
{}[\blau{Das Tüpfel}] [\blau{aufs i}]    setze der Bürgermeister von Miami, [\ldots]\footnote{taz, 25.04.2000, S.\,3}
}
\ex[*]{
{}[aufs i]    setze der Bürgermeister von Miami das Tüpfel, [\ldots]
}
\zl
\pause
\eal
\ex[]{
{}[\blau{Ihr Fett}] [\blau{weg}] bekamen natürlich auch alte und neue {Regierung [\ldots]}\footnote{Mannheimer Morgen, 10.3.1999}
}
\ex[*]{
{}[Weg] bekamen natürlich auch alte und neue Regierung ihr Fett
}
\zl
\pause
\item Um die einzelnen Extraktionen in den b-Beispielen auszuschließen,\\
      müßte man Beschränkungen formulieren, die Extraktion nur dann zulassen,\\
      wenn die anderen Elemente auch extrahiert werden.
\end{itemize}

}

% Reserve
% \subsubsection{G.\ Müller, 1998 und Fanselow, 1993, 2002}

% {
% \gotobuttonright{partikelvoranstellung}{Partikelverben}

% \frame[label=fanselow-gmueller]{
% \frametitle{G.\ \citew{GMueller98a} und \citew{Fanselow93a,Fanselow2002a}}

% \savespace
% \footnotesizeexamples
% \begin{itemize}
% \item<+-> Fanselow und G.\ Müller schlagen Analysen vor, die meiner ähneln:

% \ea
% {}[\ssub{VP}\,[Zum zweiten Mal] [die Weltmeisterschaft] \_\ssub{V}\,]$_i$ errang$_j$ Clark 1965 \_$_i$ \_$_j$.
% \z

% \item<+-> \citet{Fanselow93a}: \_\sub{V} ist eine Verbspur, wie sie beim Gapping vorkommt
% \item<+-> G.\ Müller nimmt eine Restbewegungsanalye\nocite{WdB87a,Thiersch86a} an:
%       \begin{itemize}
%       \item Aus VP werden Elemente hinausbewegt, und der Rest wird vorangestellt.
%       \item (\mex{0}) ist der extreme Fall, in dem das Verb aus der VP herausbewegt wurde.
%       \end{itemize}
% \item<+-> \citet{Fanselow2002a} schließt sich G.\ Müllers Restbewegungsanalyse\\
% für scheinbar mehrfache Vorfeldbesetzungen an.
% \bigskip
% \item<+-> \citet[S.\,281]{Haider93a}, \citet{deKuthy2002a}, \citet{dKM2001a} und \citet{Fanselow2002a}:\\
% Restbewegungsanalysen haben \hyperlink{restbewegung-probleme}{empirische Probleme}.
% \end{itemize}
% }
% }
\nocite{GMueller98a}
\nocite{Haider93a,deKuthy2002a,dKM2001a,Fanselow2002a}


\lecture{Die Analyse}{Die Analyse}


\subsubsection{Die Analyse}

\outline{

\begin{itemize}
\item Das Phänomen
\item Bisherige Ansätze
\item \blau{Die Analyse}
      \begin{itemize}
      \item Grundannahmen
      \item Details der Analyse
%      \item Mögliche Einwände
      \end{itemize}
%\item Informationsstruktur
\item Zusammenfassung und zukünftige Arbeiten
\end{itemize}

}
%\fi


\subsubsubsection{Grundannahmen}

%\psset{linewidth=0.8pt,arrowscale=2}%
\psset{nodesep=4pt}

\frame{

\frametitle{Grundannahmen: Konstituentenstellung und Valenz}

\vfill
\hfill%
%\resizebox{!}{0.8\textheight}{
%\small
\psset{xunit=1cm,yunit=5.4mm}%
%
% node labels for moving elements will be typeset by the \tmove command
% here we have to provide invisible boxes to get the line drawing right.
\begin{pspicture}(4.8,1)(14.4,7.6)

%\rput[B](1,1){\rnode{speccp}{\visible<1->{diesen Mann$_i$}}}
\rput[B](5,1){\rnode{jeder}{jeder}}
\rput[B](7,1){\rnode{ihnmf}{ihn}}
\rput[B](9,1){\rnode{kennt}{kennt}}

\rput[B](7,3){\rnode{np1}{NP[\textit{acc}]}}
\rput[B](9,3){\rnode{v}{V}}\nput[labelsep=2pt]{0}{v}{\only<2->{\nliste{NP[\textit{nom}], NP[\textit{acc}]}}}

\rput[B](5,5){\rnode{np2}{NP[\textit{nom}]}}
\rput[B](8,5){\rnode{vs1}{V$'$}}\nput[labelsep=2pt]{0}{vs1}{\only<3->{\nliste{NP[\textit{nom}]}}}

\rput[B](6.5,7){\rnode{vp}{VP}}\nput[labelsep=2pt]{0}{vp}{\only<3->{\nliste{ }}}



\psset{angleA=-90,angleB=90,arm=0pt}

\ncdiag{v}{kennt}
\ncdiag{vs1}{np1}\ncdiag{vs1}{v}
\ncdiag{vs2}{np2}\ncdiag{vs2}{vs1}
\ncdiag{vp}{vs2}

%\ncdiag{np3}{t1}

\ncdiag{i}{t2}
\ncdiag{is}{i}\ncdiag{is}{vp}
\ncdiag{vp}{np2}\ncdiag{ip}{is}

\ncdiag{np2}{jeder}
\ncdiag{np1}{ihnmf}
\ncdiag{vp}{vs1}

%\psgrid

\end{pspicture}%
\hfill\hfill\mbox{}
\vfill
\pause
\begin{itemize}
\item Valenzanforderung ist in einer Liste repräsentiert
\item Ein beliebiges Element der Liste kann mit Kopf kombiniert werden.
\pause
\item Liste mit restlichen Elementen wird nach oben gegeben.
\pause
\item Beliebige Reihenfolge der Abbindung $\to$ \\auch Abfolge Acc $<$ Nom analysierbar.
\end{itemize}

}

\subsubsubsection{Verbbewegung in HPSG}

%\subsubsubsubsection{Schematische Darstellung}

\frame{

\frametitle{\large Repräsentationen und Lexikonregeln: Verbbewegung}

\parskip0pt\itemsep1pt
\small

\vfill
\hfill%
\scalebox{0.8}{%
%\small
\psset{xunit=1cm,yunit=5.4mm}%
\psset{nodesep=4pt}
%
% node labels for moving elements will be typeset by the \tmove command
% here we have to provide invisible boxes to get the line drawing right.
\begin{pspicture}(2.6,1)(9.4,10)
%\psgrid

\only<4->{\pscurve[%showpoints=true,%
linecolor=green,arrows=<->](8.7,2.8)(8.9,2.4)(9.2,3.2)(8.3,5.2)(6.9,7.2)(5,8.2)(3.6,7.2)(3.1,5.5)}

%\rput[B](1,1){\rnode{speccp}{\visible<1->{diesen Mann$_i$}}}
\rput[B](3,1){\rnode{cleer}{kennt$_k$}}
\rput[B](5,1){\rnode{jeder}{jeder}}
\rput[B](7,1){\rnode{ihnmf}{ihn}}
\rput[B](9,1){\rnode{kennt}{[ \_ ]$_k$}}

\rput[B](7,3){\rnode{np1}{NP}}
\rput[B](9,3){\rnode{v}{V\only<4->{\!/\!/V}}}

\rput[B](5,5){\rnode{np2}{NP}}
\rput[B](8,5){\rnode{vs1}{V$'$\only<4->{/\!/V}}}

\rput[B](6.5,7){\rnode{vp}{VP\only<4->{/\!/V}}}


\rput[B](3,5){\rnode{vlex}{V}}

\rput[B](3,7){\rnode{c}{V \nliste{ VP\only<4->{/\!/V} }}}

%\rput[B](1,9){\rnode{np3}{NP}}
\rput[B](4.75,9){\rnode{cs}{VP}}


%\rput[B](3.0625,11){\rnode{cp}{VP}}




\psset{angleA=-90,angleB=90,arm=0pt}

\ncdiag{v}{kennt}
\ncdiag{vs1}{np1}\ncdiag{vs1}{v}
\ncdiag{vs2}{np2}\ncdiag{vs2}{vs1}
\ncdiag{vp}{vs2}

%\ncdiag{np3}{t1}

\ncdiag{i}{t2}
\ncdiag{is}{i}\ncdiag{is}{vp}
\ncdiag{vp}{np2}\ncdiag{ip}{is}

\ncdiag{np3}{speccp}
\ncdiag{np2}{jeder}
\ncdiag{np1}{ihnmf}
\ncdiag{vp}{vs1}
\only<-2,4->{
\ncdiag{c}{vlex}
}
\ncdiag{vlex}{cleer}
\ncdiag{cs}{c}\ncdiag{cs}{vp}
\ncdiag{cp}{np3}
\ncdiag{cp}{cs}

\only<beamer| beamer:3>{
\ncline[arrows=->,linecolor=red]{vlex}{c}
}



\end{pspicture}%
}\hfill\hfill\mbox{}
\vfill

\begin{itemize}[<+->]
\item In Verberstsätzen steht in der Verbletztposition eine Spur.
\item In Verberststellung steht eine besondere Form des Verbs,\\
      die eine Projektion der Verbspur selegiert.
\item Dieser spezielle Lexikoneintrag ist durch eine Lexikonregel lizenziert.
\item Verbindung Verb/Spur durch Informationsweitergabe im Baum
\end{itemize}
\nocite{Jacobs86a,Jacobs91a}
}

\subsubsubsection{Konstituentenbewegung}

\frame{

%\frametitle{Repräsentationen und Lexikonregeln: 
\frametitle{Konstituentenbewegung}


\vfill
\hfill%
\scalebox{0.7}{
%\small
\psset{xunit=1cm,yunit=5.4mm}
%
% node labels for moving elements will be typeset by the \tmove command
% here we have to provide invisible boxes to get the line drawing right.
\begin{pspicture}(-.1,0.8)(9.4,12)
%
%\psgrid
%
\only<2->{\pscurve[%showpoints=true,%
%arrows=<->](6.6,2.8)(6.7,2.4)(7.0,2.4)(7.4,3.2)(8.3,5.2)(6.8,7.3)(5.1,9.2)(3,10.2)(1.3,9.6)
linecolor=green,arrows=<->](4.6,4.8)(4.7,4.4)(5.0,4.4)(5.3,5.2)(6.8,7.3)(5.1,9.2)(3,10.2)(1.3,9.6)
}
%
\rput[B](1,1){\rnode{speccp}{{diesen Mann$_i$}}}
\rput[B](3,1){\rnode{cleer}{{kennt}}}
\rput[B](5,1){\rnode{jeder}{{[ \_ ]$_i$}}}
\rput[B](7,1){\rnode{ihnmf}{jeder}}
\rput[B](9,1){\rnode{kennt}{{[ \_ ]$_k$}}}
%
\rput[B](7,3){\rnode{np1}{NP}}
\rput[B](9,3){\rnode{v}{V}}
%
\rput[B](5,5){\rnode{np2}{NP/NP}}
\rput[B](8,5){\rnode{vs1}{V$'$}}
%
\rput[B](6.5,7){\rnode{vp}{VP/NP}}
%
%
\rput[B](3,5){\rnode{vlex}{V}}
%
\rput[B](3,7){\rnode{c}{V}}
%
\rput[B](1,9){\rnode{np3}{NP}}
\rput[B](4.75,9){\rnode{cs}{VP/NP}}
%
%
\rput[B](3.0625,11){\rnode{cp}{VP}}
%
%
%
%
\psset{angleA=-90,angleB=90,arm=0pt}
%
\ncdiag{v}{kennt}
\ncdiag{vs1}{np1}\ncdiag{vs1}{v}
\ncdiag{vs2}{np2}\ncdiag{vs2}{vs1}
\ncdiag{vp}{vs2}
%
\ncdiag{np3}{t1}
%
\ncdiag{i}{t2}
\ncdiag{is}{i}\ncdiag{is}{vp}
\ncdiag{vp}{np2}\ncdiag{ip}{is}
%
\pstriangle(1,1.6)(2.0,7.2)
\ncdiag{np2}{jeder}
\ncdiag{np1}{ihnmf}
\ncdiag{vp}{vs1}
\ncdiag{c}{vlex}
\ncdiag{vlex}{cleer}
\ncdiag{cs}{c}\ncdiag{cs}{vp}
\ncdiag{cp}{np3}
\ncdiag{cp}{cs}
%
\end{pspicture}
}
\hfill\hfill\mbox{}
\vfill
\begin{itemize}[<+->]
\item Wie bei Verbbewegung: Spur an ursprünglicher "`normaler"' Position.
\item Weiterreichen der Information im Baum
\item Konstituentenbewegung ist nicht lokal, Verbbewegung ist lokal\\
      mit zwei verschiedenen Merkmalen modelliert ({\sc slash} vs.\ {\sc dsl})
\end{itemize}

}

\subsubsubsection{Der Verbalkomplex}

\frame{

\frametitle{Der Verbalkomplex}


\hfill\scalebox{0.75}{%
\psset{xunit=1cm,yunit=5.4mm}%
%
% node labels for moving elements will be typeset by the \tmove command
% here we have to provide invisible boxes to get the line drawing right.
\begin{pspicture}(2.25,1)(15.8,10)

\rput[B](3,1){\rnode{er}{er}}
\rput[B](5,1){\rnode{sie}{sie}}
\rput[B](7,1){\rnode{lieben}{\alt<beamer| beamer:6>{\blau{[ \_ ]} }{lieben}}}
\rput[B](11.5,1){\rnode{will}{will}}

\rput[B](7,3){\rnode{vlieben}{\blau<3>{V}}}\nput[labelsep=2pt]{0}{vlieben}{\only<2->{\rnode{sclieben}{\blau<2>{\nliste{ NP[\textit{nom}], NP[\textit{acc}] }}}}}
\rput[B](11.5,3){\rnode{vwill}{V}}\nput[labelsep=2pt]{0}{vwill}{\only<2->{\nliste{ \blau<2>{NP[\textit{nom}], NP[\textit{acc}]}, \blau<3>{V} }}}


\rput[B](5,5){\rnode{np2}{NP[\textit{acc}]}}
\rput[B](8,5){\rnode{vliebenwill}{V}}\nput[labelsep=2pt]{0}{vliebenwill}{\only<4->{\nliste{ NP[\textit{nom}], NP[\textit{acc}] }}}

\rput[B](6.5,7){\rnode{vs}{V$'$}}\nput[labelsep=2pt]{0}{vs}{\only<5->{\nliste{ NP[\textit{nom}] }}}



\rput[B](3,7){\rnode{np1}{NP[\textit{nom}]}}

\rput[B](4.75,9){\rnode{vp}{VP}}\nput[labelsep=2pt]{0}{vp}{\only<5->{\nliste{  }}}




\psset{angleA=-90,angleB=90,arm=0pt}

\ncdiag{vwill}{will}
\ncdiag{vlieben}{lieben}
\ncdiag{vliebenwill}{vwill}\ncdiag{vliebenwill}{vlieben}
\ncdiag{vs}{np2}\ncdiag{vs}{vliebenwill}
\ncdiag{vp}{vs}\ncdiag{vp}{np1}

\ncdiag{np2}{sie}
\ncdiag{np1}{er}



%\psgrid

\end{pspicture}
}
\hfill\hfill\mbox{}%

\bigskip

\begin{itemize}[<+->]
\item Verben bilden einen Komplex
\item Argumente werden vom Matrixverb angezogen.
\item Matrixverb selegiert eingebettetes Verb und dessen Argumente.
\pause\pause
\item Funktioniert auch, wenn infinites Verb vorangestellt ist:\\
      Nicht im Vorfeld gesättigte Argumente, werden angezogen.
\end{itemize}

}


\subsubsection{Details der Analyse (Syntax)}

\subsubsection{Ein leerer Kopf im Vorfeld}

\frame{

\frametitle{Die Analyse: Ein leerer Kopf im Vorfeld}

\savespace
\begin{itemize}
\item<+-> \citet{Fanselow93a} und \citet{Hoberg97a} haben leeren Kopf vorgeschlagen.
\item<+-> Hoberg: Leerer Kopf ist Teil des Prädikatskomplexes.
\item<+-> Annahme: leerer Kopf ist dieselbe Spur wie bei Verbbewegung
\ea
{\small {}[\sub{VP} [\blau{Zum zweiten Mal}] [\blau{die Weltmeisterschaft}] \_\sub{V} ]$_i$ errang Clark 1965 \_$_i$.}
\z
\end{itemize}
}


\subsubsubsection{Verbalkomplex und Voranstellung}

\frame{

\frametitle{Verbalkomplex und Voranstellung}

\savespace%
\footnotesizeexamples%
Analyse des Verbalkomplexes und \emph{Partial Verb Phrase Fronting} (PVP):
\eal
\ex daß Clark 1965 zum zweiten Mal die Weltmeisterschaft errungen hat
\pause
\ex {}[\ssub{V$'$} [Zum zweiten Mal] errungen]$_i$  hat Clark die Weltmeisterschaft 1965 \_$_i$.
\pause
\ex {}[\ssub{VP} [Zum zweiten Mal] [die Weltmeisterschaft] errungen]$_i$   hat Clark 1965 \_$_i$.
\zl
\vspace{-1mm}%
\pause
Voranstellung mit leerem verbalen Kopf ist parallel zu PVP:
\ea
{}[\ssub{VP} [Zum zweiten Mal] [die Weltmeisterschaft] \_\ssub{V} ]$_i$ errang Clark 1965 \_$_i$.
\z

\pause
PVP ist in der HPSG gut verstanden:\\
siehe \ua \citew{Mueller97c,Mueller99a,Meurers99a}.

}

%% \frame{
%% \frametitle{Informationsfluß}

%% Muss Verbindung zwischen leerem Kopf und Verb geben:
%% \eal
%% \ex[*]{
%% Zum zweiten Mal \rot{die Weltmeisterschaft} errang Clark 1965\\
%% \rot{die Goldmedaille}.
%% }
%% \ex[*]{
%% \rot{Drei Stunden lang} die Weltmeisterschaft \rot{errang} Clark 1965.
%% }
%% \zl
%% Sowohl syntaktische als auch semantische Eigenschaften des Verbs müssen 
%% im Vorfeld zugreifbar sein.
%% }

%\if 0


\subsubsection{Eine Beispielanalyse}

%% \mode<beamer>{
%% \frame[shrink]{
%% \frametitle{Eine Beispielanalyse}
%% \parskip0pt
%% \savespace
%% \resizebox{\linewidth}{!}{%
%% \psset{xunit=1cm,yunit=5.4mm}
%% %
%% \begin{pspicture}(-0.4,0.8)(21.2,13.4)
%% %\psgrid

%% \only<5-6,10-11>{
%% \pscurve[%showpoints=true,%
%% linecolor=green,arrows=<->](4.2,9.8)(7.725,11)(13,10)(16.4,7.6)(17.2,5.6)(16.6,3.6)
%% }

%% \rput[B](1,1){\rnode{zzm}{zum zweiten Mal}}
%% \rput[B](5,1){\rnode{dw}{die Weltmeisterschaft}}
%% \rput[B](8,1){\rnode{spurvf}{[ \_ ]}}

%% \rput[B](11,1){\rnode{errang}{errang}}

%% \rput[B](15,1){\rnode{clark}{Clark}}

%% \rput[B](16.5,1){\rnode{extractiontrace}{[ \_ ]}}

%% \rput[B](18.5,1){\rnode{verbtrace}{[ \_ ]}}

%% \rput[B](16.5,3){\rnode{vextractiontrace}{V}}\nput[labelsep=2pt]{0}{vextractiontrace}{\only<6-8>{\sliste{ \textit{nom} }}}
%% \nput[labelsep=2pt]{0}{vextractiontrace}{\only<9-11>{\!/\!/\gruen{V}\sliste{\textit{nom}}}}
%% \nput[labelsep=0pt]{0}{vextractiontrace}{\only<5>{/\!/\gruen{V}}}

%% \rput[B](18.5,3){\rnode{vverbtrace}{V}}\nput[labelsep=2pt,ref=t]{0}{vverbtrace}{\only<2>{\!/\!/\rot{V}}}%
%% \nput[labelsep=2pt]{0}{vverbtrace}{\only<8->{\sliste{ \textit{nom}, V/\!/\gruen{V} }}}


%% \rput[B](17,5){\rnode{vkomplex}{V}}\nput[labelsep=2pt]{0}{vkomplex}{\only<2>{\!/\!/\rot{V}}}%
%% \nput[labelsep=2pt]{0}{vkomplex}{\only<8->{\sliste{ \textit{nom} }}}
%% \rput[B](15,5){\rnode{npclark}{NP[\textit{nom}]}}

%% \rput[B](16,7){\rnode{vp}{VP}}\nput[labelsep=0pt]{0}{vp}{\only<2>{/\!/\rot{V}}}

%% \rput[B](11,3){\rnode{verrang}{\gruen<7-11>{V}}}

%% \rput[B](11,5){\rnode{errang_mvf_lr}{\rot<2>{V}}}
%% \nput[labelsep=2pt]{0}{errang_mvf_lr}{\only<7-11>{\sliste{ \alt<8->{\textit{nom}}{\ldots},  V/\!/\gruen{V} }}}

%% \rput[B](11,7){\rnode{errang_v1_lr}{V}}\nput[labelsep=2pt,ref=B]{0}{errang_v1_lr}{\only<2>{\sliste{ VP/\!/\rot{V} }}}

%% \rput[B](12.5,9){\rnode{vp2}{VP}}


%% \rput[B](8,5){\rnode{vspurvf}{V}}\nput[labelsep=2pt]{0}{vspurvf}{\only<3>{\sliste{ \textit{nom}, \textit{acc} }}}%
%% \nput[labelsep=0pt]{0}{vspurvf}{\only<4-5,11>{/\!/\gruen{V}}}
%% %
%% \rput[B](5,5){\rnode{np_dw}{NP[\textit{acc}]}}

%% \rput[B](6.5,7){\rnode{v'_vf}{V$'$}}\nput[labelsep=2pt]{0}{v'_vf}{\only<3>{\sliste{ \textit{nom} }}}%
%% \nput[labelsep=0pt]{0}{v'_vf}{\only<4-5,11>{/\!/\gruen{V}}}

%% \rput[B](1,7){\rnode{pp_zzm}{PP}}

%% \rput[B](3.25,9){\rnode{v'_vf_2}{V$'$}}\nput[labelsep=2pt]{0}{v'_vf_2}{\only<3,6>{\sliste{ \textit{nom} }}}%
%% \nput[labelsep=0pt]{0}{v'_vf_2}{\only<4-5,11>{/\!/\gruen{V}}}

%% \rput[B](7.725,12){\rnode{vp3}{VP}}

%% \pstriangle(1,1.8)(2.8,4.9)
%% \pstriangle(5,1.8)(3.4,2.9)

%% \psset{angleA=-90,angleB=90,arm=0pt}

%% \ncdiag{vextractiontrace}{extractiontrace}
%% \ncdiag{vverbtrace}{verbtrace}

%% \ncdiag{vkomplex}{vextractiontrace}
%% \ncdiag{vkomplex}{vverbtrace}

%% \ncdiag{npclark}{clark}
%% \ncdiag{vp}{npclark}
%% \ncdiag{vp}{vkomplex}

%% \ncdiag{verrang}{errang}
%% \alt<7>{
%% \ncline[linecolor=red,arrows=->]{verrang}{errang_mvf_lr}
%% }{
%% \ncdiag{errang_mvf_lr}{verrang}
%% }
%% \alt<2>{
%% \ncline[linecolor=red,arrows=->]{errang_mvf_lr}{errang_v1_lr}
%% }{
%% \ncdiag{errang_v1_lr}{errang_mvf_lr}
%% }

%% \ncdiag{vp2}{errang_v1_lr}
%% \ncdiag{vp2}{vp}

%% \ncdiag{vspurvf}{spurvf}
%% \ncdiag{v'_vf}{vspurvf}
%% \ncdiag{v'_vf}{np_dw}

%% \ncdiag{v'_vf_2}{pp_zzm}
%% \ncdiag{v'_vf_2}{v'_vf}

%% \ncdiag{vp3}{v'_vf_2}
%% \ncdiag{vp3}{vp2}


%% \end{pspicture}
%% }
%% \small
%% \pause
%% \begin{itemize}[<+->]
%% \item Verbbewegung im Kernsatz wie erklärt: Lexikonregel lizenziert V1-Verb
%% \item Kombination von Elementen in der Vorfeldprojektion ganz normal
%% \item Da Kopf im VF eine V-Spur ist, werden V-Informationen nach oben gegeben.
%% \item VF-Projektion steht zu Extraktionsspur im VK in Beziehung \only<6>{(auch Valenz)}
%% \pause
%% \item Mult-VF-LR lizenziert Verb, das eine Spur plus deren Argumente selegiert.
%% \item Dieses steht in Verbletztstellung und bildet einen Komplex.
%% \end{itemize}


%% }

%% }

%\mode<handout>{
{

\frame<handout>{
\frametitle{Eine Beispielanalyse}

\parskip0pt
\savespace
\resizebox{\linewidth}{!}{%
\psset{xunit=1cm,yunit=5.4mm}
%
\begin{pspicture}(-0.4,0.8)(21.2,13)
%\psgrid


\rput[B](1,1){\rnode{zzm}{zum zweiten Mal}}
\rput[B](5,1){\rnode{dw}{die Weltmeisterschaft}}
\rput[B](8,1){\rnode{spurvf}{[ \_ ]}}

\rput[B](11,1){\rnode{errang}{errang}}

\rput[B](15,1){\rnode{clark}{Clark}}

\rput[B](16.5,1){\rnode{extractiontrace}{[ \_ ]}}

\rput[B](18.5,1){\rnode{verbtrace}{[ \_ ]}}

\rput[B](16.5,3){\rnode{vextractiontrace}{V}}

\rput[B](18.5,3){\rnode{vverbtrace}{V}}\nput[labelsep=2pt,ref=t]{0}{vverbtrace}{\only<2>{\!/\!/\rot{V}}}%

\rput[B](17,5){\rnode{vkomplex}{V}}\nput[labelsep=2pt]{0}{vkomplex}{\only<2>{\!/\!/\rot{V}}}%
\rput[B](15,5){\rnode{npclark}{NP[\textit{nom}]}}

\rput[B](16,7){\rnode{vp}{VP}}\nput[labelsep=0pt]{0}{vp}{\only<2>{/\!/\rot{V}}}

\rput[B](11,3){\rnode{verrang}{V}}

\rput[B](11,5){\rnode{errang_mvf_lr}{\rot<2>{V}}}

\rput[B](11,7){\rnode{errang_v1_lr}{V}}\nput[labelsep=2pt,ref=B]{0}{errang_v1_lr}{\only<2>{\sliste{ VP/\!/\rot{V} }}}

\rput[B](12.5,9){\rnode{vp2}{VP}}


\rput[B](8,5){\rnode{vspurvf}{V}}\nput[labelsep=2pt]{0}{vspurvf}{\only<3>{\sliste{ \textit{nom}, \textit{acc} }}}%
%
\rput[B](5,5){\rnode{np_dw}{NP[\textit{acc}]}}

\rput[B](6.5,7){\rnode{v'_vf}{V$'$}}\nput[labelsep=2pt]{0}{v'_vf}{\only<3>{\sliste{ \textit{nom} }}}%

\rput[B](1,7){\rnode{pp_zzm}{PP}}

\rput[B](3.25,9){\rnode{v'_vf_2}{V$'$}}\nput[labelsep=2pt]{0}{v'_vf_2}{\only<3,6>{\sliste{ \textit{nom} }}}%

\rput[B](7.725,12){\rnode{vp3}{VP}}

\pstriangle(1,1.8)(2.8,4.9)
\pstriangle(5,1.8)(3.4,2.9)

\psset{angleA=-90,angleB=90,arm=0pt}

\ncdiag{vextractiontrace}{extractiontrace}
\ncdiag{vverbtrace}{verbtrace}

\ncdiag{vkomplex}{vextractiontrace}
\ncdiag{vkomplex}{vverbtrace}

\ncdiag{npclark}{clark}
\ncdiag{vp}{npclark}
\ncdiag{vp}{vkomplex}

\ncdiag{verrang}{errang}
\ncdiag{errang_mvf_lr}{verrang}
\ncline[linecolor=red,arrows=->]{errang_mvf_lr}{errang_v1_lr}

\ncdiag{vp2}{errang_v1_lr}
\ncdiag{vp2}{vp}

\ncdiag{vspurvf}{spurvf}
\ncdiag{v'_vf}{vspurvf}
\ncdiag{v'_vf}{np_dw}

\ncdiag{v'_vf_2}{pp_zzm}
\ncdiag{v'_vf_2}{v'_vf}

\ncdiag{vp3}{v'_vf_2}
\ncdiag{vp3}{vp2}


\end{pspicture}
}\small
\pause
\begin{itemize}
\item<2-> Verbbewegung im Kernsatz wie erklärt: Lexikonregel lizenziert V1-Verb
\item<3-> Kombination von Elementen in der Vorfeldprojektion ganz normal
%% \item<4-> Da Kopf im VF eine V-Spur ist, werden V-Informationen nach oben gegeben.
%% \item<5-> VF-Projektion steht zu Extraktionsspur im VK in Beziehung \only<6>{(auch Valenz)}
%% \pause
%% \item<7-> Mult-VF-LR lizenziert Verb, das eine Spur plus deren Argumente selegiert.
%% \item<8-> Dieses steht in Verbletztstellung und bildet einen Komplex.
\end{itemize}

}

\addtocounter{framenumber}{-1}

\frame<handout>{
\frametitle{Eine Beispielanalyse}
\parskip0pt
\savespace
\resizebox{\linewidth}{!}{%
\psset{xunit=1cm,yunit=5.4mm}
%
\begin{pspicture}(-0.4,0.8)(21.2,13)
%\psgrid

\pscurve[%showpoints=true,%
linecolor=green,arrows=<->](4.2,9.8)(7.725,11)(13,10)(16.4,7.6)(17.2,5.6)(16.6,3.6)


\rput[B](1,1){\rnode{zzm}{zum zweiten Mal}}
\rput[B](5,1){\rnode{dw}{die Weltmeisterschaft}}
\rput[B](8,1){\rnode{spurvf}{[ \_ ]}}

\rput[B](11,1){\rnode{errang}{errang}}

\rput[B](15,1){\rnode{clark}{Clark}}

\rput[B](16.5,1){\rnode{extractiontrace}{[ \_ ]}}

\rput[B](18.5,1){\rnode{verbtrace}{[ \_ ]}}

\rput[B](16.5,3){\rnode{vextractiontrace}{V}}
\nput[labelsep=0pt]{0}{vextractiontrace}{\only<5>{/\!/\gruen{V}}}

\rput[B](18.5,3){\rnode{vverbtrace}{V}}


\rput[B](17,5){\rnode{vkomplex}{V}}
\rput[B](15,5){\rnode{npclark}{NP[\textit{nom}]}}

\rput[B](16,7){\rnode{vp}{VP}}

\rput[B](11,3){\rnode{verrang}{\gruen{V}}}

\rput[B](11,5){\rnode{errang_mvf_lr}{\rot<2>{V}}}
\rput[B](11,7){\rnode{errang_v1_lr}{V}}

\rput[B](12.5,9){\rnode{vp2}{VP}}


\rput[B](8,5){\rnode{vspurvf}{V}}
\nput[labelsep=0pt]{0}{vspurvf}{\only<4-5,11>{/\!/\gruen{V}}}
%
\rput[B](5,5){\rnode{np_dw}{NP[\textit{acc}]}}

\rput[B](6.5,7){\rnode{v'_vf}{V$'$}}
\nput[labelsep=0pt]{0}{v'_vf}{\only<4-5,11>{/\!/\gruen{V}}}

\rput[B](1,7){\rnode{pp_zzm}{PP}}

\rput[B](3.25,9){\rnode{v'_vf_2}{V$'$}}
\nput[labelsep=0pt]{0}{v'_vf_2}{\only<4-5,11>{/\!/\gruen{V}}}

\rput[B](7.725,12){\rnode{vp3}{VP}}

\pstriangle(1,1.8)(2.8,4.9)
\pstriangle(5,1.8)(3.4,2.9)

\psset{angleA=-90,angleB=90,arm=0pt}

\ncdiag{vextractiontrace}{extractiontrace}
\ncdiag{vverbtrace}{verbtrace}

\ncdiag{vkomplex}{vextractiontrace}
\ncdiag{vkomplex}{vverbtrace}

\ncdiag{npclark}{clark}
\ncdiag{vp}{npclark}
\ncdiag{vp}{vkomplex}

\ncdiag{verrang}{errang}
\alt<7>{
\ncline[linecolor=red,arrows=->]{verrang}{errang_mvf_lr}
}{
\ncdiag{errang_mvf_lr}{verrang}
}
\alt<2>{
\ncline[linecolor=red,arrows=->]{errang_mvf_lr}{errang_v1_lr}
}{
\ncdiag{errang_v1_lr}{errang_mvf_lr}
}

\ncdiag{vp2}{errang_v1_lr}
\ncdiag{vp2}{vp}

\ncdiag{vspurvf}{spurvf}
\ncdiag{v'_vf}{vspurvf}
\ncdiag{v'_vf}{np_dw}

\ncdiag{v'_vf_2}{pp_zzm}
\ncdiag{v'_vf_2}{v'_vf}

\ncdiag{vp3}{v'_vf_2}
\ncdiag{vp3}{vp2}


\end{pspicture}
}\small
\pause
\begin{itemize}[<+->]
\item Verbbewegung im Kernsatz wie erklärt: Lexikonregel lizenziert V1-Verb
\item Kombination von Elementen in der Vorfeldprojektion ganz normal
\item Da Kopf im VF eine V-Spur ist, werden V-Informationen nach oben gegeben.
\item VF-Projektion steht zu Extraktionsspur im VK in Beziehung (auch Valenz)
\end{itemize}

}

\addtocounter{framenumber}{-1}

\frame<handout>[label=current]{
\frametitle{Eine Beispielanalyse}
\parskip0pt
\savespace
\resizebox{\linewidth}{!}{%
\psset{xunit=1cm,yunit=5.4mm}
%
\begin{pspicture}(-0.4,0.8)(21.2,13)
%\psgrid

\only<5-6,10-11>{
\pscurve[%showpoints=true,%
linecolor=green,arrows=<->](4.2,9.8)(7.725,11)(13,10)(16.4,7.6)(17.2,5.6)(16.6,3.6)
}

\rput[B](1,1){\rnode{zzm}{zum zweiten Mal}}
\rput[B](5,1){\rnode{dw}{die Weltmeisterschaft}}
\rput[B](8,1){\rnode{spurvf}{[ \_ ]}}

\rput[B](11,1){\rnode{errang}{errang}}

\rput[B](15,1){\rnode{clark}{Clark}}

\rput[B](16.5,1){\rnode{extractiontrace}{[ \_ ]}}

\rput[B](18.5,1){\rnode{verbtrace}{[ \_ ]}}

\rput[B](16.5,3){\rnode{vextractiontrace}{V}}
\nput[labelsep=2pt]{0}{vextractiontrace}{\only<9-11>{\!/\!/\gruen{V}\sliste{\textit{nom}}}}

\rput[B](18.5,3){\rnode{vverbtrace}{V}}
\nput[labelsep=2pt]{0}{vverbtrace}{\only<8->{\sliste{ \textit{nom}, V/\!/\gruen{V} }}}


\rput[B](17,5){\rnode{vkomplex}{V}}
\nput[labelsep=2pt]{0}{vkomplex}{\only<8->{\sliste{ \textit{nom} }}}
\rput[B](15,5){\rnode{npclark}{NP[\textit{nom}]}}

\rput[B](16,7){\rnode{vp}{VP}}

\rput[B](11,3){\rnode{verrang}{\gruen<7-11>{V}}}

\rput[B](11,5){\rnode{errang_mvf_lr}{V}}
\nput[labelsep=2pt]{0}{errang_mvf_lr}{\only<7-11>{\sliste{ \alt<8->{\textit{nom}}{\ldots},  V/\!/\gruen{V} }}}

\rput[B](11,7){\rnode{errang_v1_lr}{V}}\nput[labelsep=2pt,ref=B]{0}{errang_v1_lr}{\only<2>{\sliste{ VP/\!/\rot{V} }}}

\rput[B](12.5,9){\rnode{vp2}{VP}}


\rput[B](8,5){\rnode{vspurvf}{V}}
\nput[labelsep=0pt]{0}{vspurvf}{\only<4-5,11>{/\!/\gruen{V}}}
%
\rput[B](5,5){\rnode{np_dw}{NP[\textit{acc}]}}

\rput[B](6.5,7){\rnode{v'_vf}{V$'$}}
\nput[labelsep=0pt]{0}{v'_vf}{\only<4-5,11>{/\!/\gruen{V}}}

\rput[B](1,7){\rnode{pp_zzm}{PP}}

\rput[B](3.25,9){\rnode{v'_vf_2}{V$'$}}\nput[labelsep=2pt]{0}{v'_vf_2}{\!/\!/\gruen{V}\sliste{\textit{nom}}}

\rput[B](7.725,12){\rnode{vp3}{VP}}

\pstriangle(1,1.8)(2.8,4.9)
\pstriangle(5,1.8)(3.4,2.9)

\psset{angleA=-90,angleB=90,arm=0pt}

\ncdiag{vextractiontrace}{extractiontrace}
\ncdiag{vverbtrace}{verbtrace}

\ncdiag{vkomplex}{vextractiontrace}
\ncdiag{vkomplex}{vverbtrace}

\ncdiag{npclark}{clark}
\ncdiag{vp}{npclark}
\ncdiag{vp}{vkomplex}

\ncdiag{verrang}{errang}
\alt<7>{
\ncline[linecolor=red,arrows=->]{verrang}{errang_mvf_lr}
}{
\ncdiag{errang_mvf_lr}{verrang}
}
\alt<2>{
\ncline[linecolor=red,arrows=->]{errang_mvf_lr}{errang_v1_lr}
}{
\ncdiag{errang_v1_lr}{errang_mvf_lr}
}

\ncdiag{vp2}{errang_v1_lr}
\ncdiag{vp2}{vp}

\ncdiag{vspurvf}{spurvf}
\ncdiag{v'_vf}{vspurvf}
\ncdiag{v'_vf}{np_dw}

\ncdiag{v'_vf_2}{pp_zzm}
\ncdiag{v'_vf_2}{v'_vf}

\ncdiag{vp3}{v'_vf_2}
\ncdiag{vp3}{vp2}


\end{pspicture}
}\small
\pause
\begin{itemize}[<+->]
\item Verbbewegung im Kernsatz wie erklärt: Lexikonregel lizenziert V1-Verb
\item Kombination von Elementen in der Vorfeldprojektion ganz normal
\item Da Kopf im VF eine V-Spur ist, werden V-Informationen nach oben gegeben.
\item VF-Projektion steht zu Extraktionsspur im VK in Beziehung \only<6>{(auch Valenz)}
\pause
\item Mult-VF-LR lizenziert Verb, das eine Spur plus deren Argumente selegiert.
\item Dieses steht in Verbletztstellung und bildet einen Komplex.
\end{itemize}

}

}


