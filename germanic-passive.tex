%% -*- coding:utf-8 -*-

\subtitle{Passive}

\section{Passive}

\huberlintitlepage[22pt]

\settowidth\jamwidth {(Icelandic)}


\frame{
\frametitle{Literature}



For this section there is chapter~7 in \citew{MuellerGermanic}.

Müller, Stefan, \citeyear{MuellerGermanic}. \emph{Germanic Syntax}. Berlin: Language Science
Press.
}


\subsection{Subjects}

\frame{
\frametitle{Subjects (German)}


\begin{itemize}
\item What is a subject?

\pause\medskip

\item In German, non-predicative nominal groups in the nominative case:
	\eal
	\ex
		\gll Der Mann lacht. \\
				the.\NOM{} man laughs \\
		\glt	`The man loughs.'
			
	\ex 
		\gll Der Mann hilft ihr. \\
				the.\NOM{} man helps her.\DAT\ \\
		\glt 	`The man help her.'
	
	\ex 
		\gll Der Mann gibt ihr ein Buch. \\
				the.\NOM{} man gives her.\DAT\ a.\ACC\ book. \\
		\glt 	`The man gives her a book.'
		
	\zl
	
\end{itemize}
}

\frame{
	\frametitle{Subjects (German)}
	
	
\begin{itemize}

\item What about the genitive and dative case in (\mex{1})?

	\eal
		\ex 
			\gll Des Opfers wurde gedacht. \\
					the.\GEN{} victim \AUX{} remembered \\
			\glt `The victim was remembered.'
		
		\ex 
			\gll Dem Mann wurde geholfen. \\
					the.\DAT{} man \AUX{} helped. \\
			\glt `The man received help.'
	\zl

\pause

In German, these are not counted as subjects.

%(Bitte lesen Sie \citew[Abschnitt~1.7]{MuellerGTBuch2}, wenn Ihnen das unklar ist.)

\end{itemize}
}


\frame{
\frametitle{Subjects (Icelandic)}


\begin{itemize}
\item The order for the SVO language Icelandic is identical \citep{ZMT85a}:
\eal
\ex 
	\gll Þeim            var hjálpað.\\
	they.\PL.\DAT{} was helped\\
     
\ex 
	\gll Hennar var saknað.\\
     she.\SG.\GEN{} was missed\\
\zl

Grammatical function cannot be read from the position in V2 sentences,\\
as the NPs could also be preceding objects.

\pause

\item \citet*{ZMT85a} show that it makes sense to treat these non-nominatives as subjects.

\begin{itemize}
\item controllability
\item position
\item \ldots
\end{itemize}

\pause

\item Such non-nominative-subjects are also called \alert{quirky subjects} oder \alert{oblique subjects}; \alert{schräge Subjekte} in German. 
\end{itemize}

}


\frame{
\frametitle{Subject: Position V2-sentences}

\begin{itemize}
\item Subjects immediately follow the finite verb when preceded by another constituent \citep*[Abschnitt~2.3]{ZMT85a}:

\eal
%% \ex[]{
%% \gll Refinn           skaut  Ólafur      með þessari byssu.\\
%%      den.Fuchs.\ACC{} schoss Olaf.\NOM{} mit diesem  Gewehr\\
%% }
% selbst erfunden, check
\ex[]{
	\gll Með  þessari byssu   skaut \gruen{Ólafur}      \rot{refinn}.\\
	with this    shotgun shot  Olaf.\NOM{} the.fox.\ACC{}\\
}
\ex[*]{
	\gll Með þessari byssu  skaut \rot{refinn}         \gruen{Ólafur}.\\
	with this  shotgun shot  the.fox.\ACC{} Olaf.\NOM{}\\
}
\zl

\end{itemize}

}


\frame{
\frametitle{Subject: Position V2-sentences}

\begin{itemize}

\item wh-questions
\settowidth\jamwidth {(wh-question)}
	\eal
		\ex[]{
			\gll Hvenær hafði \gruen{Sigga}        hjálpað \rot{Haraldi}?\\
			when   has   Sigga.\NOM{} helped Harald.\DAT{} \\ \jambox{(wh-question)}
	}
%
		\ex[*]{
			\gll Hvenær hafði \rot{Haraldi}        \gruen{Sigga}        hjálpað?\\
			when   has   Harald.\DAT{} Sigga.\NOM{} helped\\
		}
%
		\zl

\pause

\item dative object can be fronted, but then it has to be realized in initial position
	\ea[]{
		\gll \rot{Haraldi}       hafði \gruen{Sigga}        aldrei hjálpað.\\
		Harald.\DAT{} has   Sigga.\NOM{} never  helped \\ \jambox{(V2-Satz)}
}
\z

\end{itemize}

}


\frame{
\frametitle{Subject: Position V1-sentences}


\begin{itemize}

\item yes/no questions:
	\eal
		\ex[]{
			\gll Hafði \gruen{Sigga}        aldrei hjálpað \rot{Haraldi}?\\
			has   Sigga.\NOM{} never  helped   Harald.\DAT\\
		}
		
		\ex[*]{
			\gll Hafði \rot{Haraldi}       \gruen{Sigga}        aldrei hjálpað?\\
			has   Harald.\DAT{} Sigga.\NOM{} never  helped\\
		}
	\zl

\end{itemize}

}

\exewidth{(135)}


\frame[shrink=5]{
\frametitle{Quirky subjects and position}
\smallexamples

\begin{itemize}
	
\item certain datives: 

	\eal
		\ex[]{
			\gll Hefur \gruen{henni}      alltaf þótt    \rot{Ólafur}      leiðinlegur?\\
			has   she.\DAT{} always thought Olaf.\NOM{} boring.\NOM{}\\
			\glt `Has she always considered Olaf boring?'
		}
		
		\ex[]{
		\gll \rot{Ólafur} hefur \gruen{henni} alltaf þótt leiðinlegur.\\
		     Olaf.\NOM{} has   she.\DAT{} always thought boring.\NOM{}\\
		\glt `She always considered Olaf boring.'
		}
		
		\ex[*]{
		\gll Hefur \rot{Ólafur} \gruen{henni} alltaf þótt leiðinlegur?\\
		     has   Olaf.\NOM{} her.\DAT{} always thought boring.\NOM{}\\
		}
	
		\zl
		

\pause
\item The equivalent in German would be:

	\ea[??]{
	Mich dünkt der Mann langweilig. \\
		I.\ACC{} thinks the.\NOM{} man boring\\
		\glt `The man seems boring to me.' 
	}
	\z
	
\emph{dünkt} is archaic and is usually used with a \emph{dass} `that' clause -- if it is used at all

\pause
\item but:
	\ea
	Mir scheint der Mann langweilig. \\
		I.\DAT{} seems the.\NOM{} man boring\
		\glt `The man seems boring to me.'
	\z

\end{itemize}

}

\frame{
\frametitle{Subjects in control constructions}

\begin{itemize}
\item subjects in control constructions or in so-called \emph{arbitrary control} can be omitted:

	\eal
		\ex
		\gll Ég  vonast til að fara heim.\\
		     I   hope   for to go   home\\
			\glt `I hope to go home.'

	\ex
		\gll Að fara heim snemma er óvenjulegt.\\
		     to go home   early is unusual\\
		     \glt `It is unusual to go home early.'

	\zl

\end{itemize}

}


\frame{
\frametitle{Quircky subjects in control constructions}

\begin{itemize}
\item \emph{vantar} `to lack' takes two accusatives:
	
	\eal
		\gll Mig        vantar peninga.\\
		     I.\ACC{} lack   money.\ACC\ \\
	\zl

\pause

\item but still, the verb can be embedded under \emph{vonast} `to hope':

	\eal
		\gll Ég  vonast til að vanta ekki peninga.\\
		     I  hope   for to lack  not  money.\ACC\ \\
		     \glt `I hope that I do not lack money.'
		
	\zl

\pause

\item to compare with German: 

		\eal[*]{
			\gll Ich hoffe, kein         Geld  zu fehlen.\\
				I hope     not.a.\NOM{} money to lack\\
				\glt Intended: `I hope that I do not lack money.'
		}
		\zl

\end{itemize}

}


\frame{
\frametitle{Subject-verb-congruency?}

\begin{itemize}
\item Verbs are congruent with the nominative element. \\
			If there is none, the verb is third person singular (neuter).

\pause

\item no congruency in (\mex{1}a):

	\eal
		\ex 
		\gll Þeim        var hjálpað.\\
			they.\DAT{} was helped\\
			\glt `They were helped.'
		     
	\ex 
		\gll Hennar var saknað.\\
		     sie.\blau{\SG}.\GEN{} wurde vermisst\\
	\zl
	
	\pause
	
	\item However, the dative and the genitive are still subjects, as we will see in a moment. \\ 

\end{itemize}

}


\frame{
\frametitle{Quirky subjects passive: position}


\begin{itemize}
\item Dative follows finite in V1:

	\ea
		\gll Var \gruen{honum} aldrei hjálpað af foreldrum sinum?\\
			     was he.\DAT{} never  helped   by parents   his\\
			     \glt `Did his parents never help him?'
	\z

\pause

\item Dative follows finite in V2:

	\ea
		\gll Í prófinu  var \gruen{honum} vist hjálpað.\\
		     in Prüfung war er.\DAT{} scheinbar geholfen\\
			\glt `Anscheinend wurde ihm bei der Prüfung geholfen.'
	\z
\end{itemize}

}




\frame{
\frametitle{Quircky  Subject Passiv: Control}


\begin{itemize}
	
\item The dative subject can be omitted:

	\eal
	\ex
		\gll Ég vonast til að verða hjálpað.\\
		  	  I  hope   for to be helped \\
		     
	\ex
		\gll Að vera hjálpað i prófinu er óleyfilegt.\\
		     to be helped in the.exam is un.allowed\\
		     \glt `It is not allowed to be helped in the exam.'
		%(`Es ist verboten in der Prüfung geholfen zu bekommen.')
	\zl

\pause

\item Compare: 

	\eal*{
		\gll Ich hoffe geholfen zu werden. \\
			I   hope  helped   to \AUX\ \\ 
		}
	\zl
	
	\pause
	
	\item Only possible with \emph{bekommen}-passive:
	
		\eal
			\gll Ich hoffe hier geholfen zu bekommen.\footnote{
			\url{http://www.photovoltaikforum.com/sds-allgemein-ueber-solar-log-f38/solarlog-1000-mit-wifi-anschliesen-t96371.html}. 10.01.2014} \\
				 I   hope  here helped   to \AUX\ \\
		\zl


\end{itemize}

}



\subsection{The case of arguments: structural and lexical case}
\label{sec-struk-lex-kas}
\label{sec-struc-lex-kas}

\frame{
\frametitle{Case and case principles}

\begin{itemize}
\item What types of case are there?
\pause

\item How do cases depend on the syntactic context?
\pause

\item Up to now, cases have been defined in valency lists, \\
			but once we know the principles, this no longer needs to be the case.

Capture generalizations and \\ 
only need one lexicon entry for the verb  \emph{lesen} `to read' in (\mex{1}):

	\eal
		\ex 
			\gll \blau{Er} möchte das Buch lesen.\\
					he.\NOM{} wants the.\ACC{} book read \\
					\glt `He wants to read the book.'
					
		\ex \gll Ich sah \blau{ihn} das Buch lesen. \\
					I   saw him the book read\\
					\glt `I saw him read the book.' 
		
	\zl

The case of the subject (and the object) is regulated by principle.

\end{itemize}

}



\subsubsection{Structural case}

\frame{
\frametitle{Structural and lexical case}

\begin{itemize}
	\item If the case of arguments depends on the syntactic environment,\\
	one speaks of {\blue{structural case}}.\\
	Otherwise the arguments have {\blue{lexical case}}.
      
\pause
	\item Examples of structural cases are:
	
		\eal
		\ex 
			\gll \blau{Der} \blau{Installateur} kommt. \\
					the.\NOM{} plumber      comes\\
					\glt `The plumber comes.			
		
		\pause
		
		\ex \gll Der Mann läßt \blau{den} \blau{Installateur} kommen. \\
					the man  lets the.\ACC{} plumber      come\\
					\glt `The man lets the plumber come.'
			
		\pause
		
		\ex \gll das Kommen \blau{des} \blau{Installateurs} \\
					the coming of.the.\GEN{} plumber\\
					\glt `the coming of the plumber'
					
		\zl
\end{itemize}

}

\frame{
	\frametitle{Structural and lexical case}
	
\begin{itemize}
\item In (\mex{0}) the case of the subject of \emph{kommen} `to come' is expressed differently,\\
in (\mex{1}) the case of the object of \emph{schlagen} `to defeat':

		\eal
		\ex \gll Judit schlägt \blau{den} \blau{Weltmeister}. \\
					Judit defeats the.\ACC{} world.champion\\
					\glt `Judit defeats the world champion.'
					
		\ex \gll \blau{Der} \blau{Weltmeister} wird geschlagen.\\
					the.\NOM{} world.champion \AUX{} beaten\\
					\glt `The world champion is beaten.'
		\zl
		
\end{itemize}
}

\subsubsection{Lexical case}

%\subsubsection{Genitiv}

\frame{
\frametitle{Lexical case}

\begin{itemize}
\item Genitive dependent on the verb is a lexical case:\\
The case of a genitive object does not change with passivization.
		
		\eal
			\ex[]{
				\gll Wir gedenken \blau{der Opfer}. \\
							we.\NOM{} remember the victims.\GEN{}\\
							\glt `We remember the victims.'
		}
		
			\ex[]{
				\gll \blau{Der Opfer} wird gedacht. \\
							the.\GEN{} victims \AUX{} remembered\\
							\glt `The victims are remembered.'
		}
		
			\ex[*]{
				\gll \blau{Die} \blau{Opfer} wird / werden gedacht. \\
							the.\NOM{} victims \AUX.3\SG{} / \AUX.3\PL{} remembered\\
		}
		
		\zl
		
		\pause
		
		(\mex{0}b) $=$ impersonal passive, there is no subject.
		
\end{itemize}

}

%\subsubsection{Dativ}

\frame{
\frametitle{The dative is a lexical case?}.

\begin{itemize}
\item Similarly, there are no changes with dative objects:

\eal
\ex \gll Der Mann hat \blau{ihm} geholfen. \\
				the.\NOM{} man  has him.\DAT{} helped\\
				
\ex \gll \blau{Ihm} wird geholfen. \\
				 him.\DAT{} \AUX{} helped\\
\zl

\pause

\item But what about  (\mex{1})?
\eal
\ex \gll Der Mann  hat   den Ball \blau{dem} \blau{Jungen} geschenkt. \\
				 the man   has   the ball the boy given\\
\ex \gll \blau{Der} \blau{Junge} bekam den Ball            geschenkt. \\
				the boy   got   the ball given\\   
\zl

\end{itemize}
}

\frame{
	\frametitle{The dative is a lexical case?}
\begin{itemize}
\item The classification of the dative case is the subject of controversial debate.
\item[] Three possibilities for dative arguments:
\medskip

\begin{enumerate}[<+->]
\item Hypothesis 1: All datives are lexical.
\medskip
\item Hypothesis 2: Some datives are lexical, others are structural.
\medskip
\item Hypothesis 3: All datives are structural.
\end{enumerate}
\end{itemize}

}

\frame{
\frametitle{Hypothesis 1: All datives are lexical}

\begin{itemize}
	
\item If you treat the dative as a lexical case,
you have to assume a conversion from lex.\ to str.\ case for the dative passive.

\pause

\item Haider's \citeyearpar[20]{Haider86} examples are then immediately explained. Analogous examples are:
\medskip

\eal
\ex[]{%\iw{streicheln}
	\gll
		Er streichelt \blau{den} \blau{Hund}. \\
			he.\NOM{} strokes the.\ACC{} dog\\
}
\ex[]{
	\gll
		\blau{Der} \blau{Hund} wird gestreichelt. \\
	 		the.\NOM{} dog  \AUX{} stroked\\
}
\ex[]{
	\gll sein Streicheln \blau{des} \blau{Hundes} \\
		 his  stroking of.the.\GEN{} dog\\
}

\zl 

\end{itemize}
}

\frame{
\frametitle{Hypothesis 1: All datives are lexical}
	
\begin{itemize}
\item With datives:		
\eal
\ex[]{\label{bsp-er-hilft-den-kindern}\iw{helfen}
\gll Er hilft \blau{den} \blau{Kindern}. \\	
				he helps the.\DAT{} children\\
}

\ex[]{
	\gll \blau{Den} \blau{Kindern} wird geholfen. \\
			the.\DAT{} children \AUX{} helped\\
}

\ex[]{ \label{das-helfen-der-Kinder}
	\gll das Helfen \blau{der} \blau{Kinder} {(Kinder nur Agens)} \\
		the helping {of.the.\GEN{}} children {}\\
}

\ex[*]{\label{sein-helfen-der-Kinder}
	\gll sein Helfen \blau{der} \blau{Kinder} \\
	     his  helping the children\\
}

\zl

\pause

\item Dative can only be expressed prenominally:
\ea
	\gll das \blau{Den-Kindern}-Helfen \\
		 the the-children-helping\\
\z

\end{itemize}
}

\frame{
\frametitlefit{Hypothesis 2: Some datives are structural, bivalent verbs}

\begin{itemize}
\item If you only have the distinction structural/lexical,
\item[]  you get a problem with bivalent verbs:

\eal
	\ex 
		\gll Er hilft ihm. \\
			he helps him\\
			
	\ex 
		\gll Er unterstützt ihn. \\
			he supports him\\
\zl


The information in the dictionary entry of \emph{help} 
and \emph{support} must be different.

\pause

\item With ditransitive verbs, case can be derived from general
principles (nom, dat, acc), but this is not possible with bivalent verbs. 

$\to$ dative with \emph{help} is classified as lexical,\\
but dative with \emph{give} as structural.

Prediction: Dative passive is not possible with verbs with lexical dative.

\end{itemize}
}

\frame[allowframebreaks]{
\frametitle{Hypothese 2: Das Dativpassiv mit bivalenten Verben}

\savespace

\eal
	\ex 
		\gll Er kriegte von vielen geholfen / gratuliert / applaudiert.\\
				he got by many helped {} congratulated {} applauded\\

	\ex 
		\gll Man kriegt täglich gedankt.\\
				one gets   daily thanked\\

\zl



The examples in (\mex{1}) are corpus examples:

\eal
\ex 
\gll "`Da kriege ich geholfen."'\footnote{%
      	Frankfurter Rundschau, 26.06.1998, p.\,7.} \\
		\quotespace{}there get  I  helped \\
\glt `Somebody helps me there.'
	
\ex
% auch nach applaudiert geholfen + bekommen und kriegen gesucht 21.09.2003
	\gll Heute morgen bekam ich sogar schon gratuliert.\footnote{%
Brief von Irene G.\ an Ernst G.\ vom 10.04.1943, Feldpost-Archive mkb-fp-0270}\\
%Branich IG-Vorsitzender Friedel Schönel meinte deshalb, 
		today morning \AUX{}  I even already congratulated \\
		\glt `Somebody even wished me a happy birthday this morning already.'


\ex
	\gll "`Klärle"' hätte es wirklich mehr als verdient, auch mal zu einem "`unrunden"' Geburtstag gratuliert zu bekommen.\footnote{
Mannheimer Morgen, 28.07.1999, Lokales; "`Klärle"' feiert heute Geburtstag.} \\ 
	\hphantom{"`}Klärle had it really more than deserved also once to a \hphantom{"`}insignificant birthday congratulated to \AUX \\
		\glt `Klärle would have more than deserved to be wished a happy birthday, even an insignificant birthday.'
		
		
\ex
	\gll Mit dem alten Titel von Elvis Presley "`I can't help falling in love"' bekam Kassier Markus Reiß zum Geburtstag gratuliert, [\ldots]\footnote{
%der dann noch viel später bekannte: "Ich hab' immer noch Gänsehaut, das war der schönste Teil meines Geburtstages." Doch auch die anderen Abteilungen des Bürgervereins können auf ein erfolgreiches Jahr 1998 zurückblicken.
Mannheimer Morgen, 21.04.1999, Lokales; Motor des gesellschaftlichen Lebens.%
} \\ 
		with the old song   by  Elvis Presley \hphantom{"`}I can't help falling in love got cashier Markus Reiß to.the birthday congratulated\\
		\glt `The cashier Markus Reiß was wished a happy birthday with the old Elvis Presley song ``I can't help falling in love with you''.'
\zl

So: Dative passive has to be done differently somehow if dative is lexical with two-character verbs.

Then you can immediately assume hypothesis 1: All datives are lexical.

}

%% \subsubsection{Akkusativ}

%% \frame{
%% \frametitle{Akkusativ}

%% Neben der Möglichkeit des strukturellen Akkusativs gibt es auch lexikalische Akkusative:
%% \eal
%% \ex \blau{Ihn} dürstet.
%% \ex Der Vater lehrte seinen Sohn \blau{einen neuen Tritt}.
%% \zl

%% \ea
%% Die Söhne wurden einen neuen Tritt gelehrt.
%% }

%% \subsection{Adjektivumgebungen}

%% \frame{
%% \frametitle{Lexikalischer Kasus in Adjektivumgebungen}

%% Kasus von Objekten von Adjektiven kann sich nicht ändern.\\
%% Adjektive können Genitiv und Dativ zuweisen:
%% \eal
%% \ex Ich war mir \blau{dessen} sicher.
%% \ex Sie ist \blau{ihm} treu.
%% \zl
%% \pause
%% Die Zuweisung von Akkusativ ist ebenfalls möglich:
%% \eal
%% \ex Das ist \blau{diesen Preis} nicht wert.
%% \ex Der Student ist \blau{das Leben} im Wohnheim nicht gewohnt.\iw{gewohnt}\footnote{
%%         \citep*[S.\,312]{HB72a}
%%       }
%% \ex Du bist mir \blau{eine Erkl"arung} schuldig.\footnote{
%%         \citep*[S.\,620]{HFM81}
%%       }
%% \zl
%% Akkusativ ist bei Adjektivkomplementen aber selten \citep{Haider85b}.
%% }

%% \frame{
%% \frametitle{Struktureller Kasus in Adjektivumgebungen}


%% Kasus der Subjekte von Adjektiven hängt von der syntaktischen
%% Umgebung ab \citep{Wunderlich84}:
%% \eal
%% \ex \blau{Der Mond} wurde kleiner.\iw{klein}
%% \ex Er sah\iw{sehen} \blau{den Mond} kleiner werden.
%% \zl

%% }


%% \subsection{Semantische Kasus}
%% \label{sec-sem-kasus}
%% \is{Kasus!semantischer|(}

%% \frame{
%% \frametitle{Semantische Kasus}

%% \begin{itemize}
%% \item NPen können auch als Adjunkte auf"|treten \citep{Haider85b}:
%% \eal
%% \ex Sie hörten \blau{den ganzen Tag} dieselbe Schallplatte.
%% \ex Laßt \blau{mir} den Hund in Ruhe!
%% \ex \blau{Eines Tages} erschien ein Fremder.
%% \zl
%% \pause
%% \item auch der Urteilsdativ ({\it Dativ iudicantis})  \citep{Wegener85b}:

%% \eal
%% \ex Das ist \blau{mir} zu\iw{zu!Grad} schwer.
%% \ex Das ist \blau{dem Kind} zu langweilig / nicht interessant genug.\iw{genug!Grad}
%% \ex Du läufst \blau{der Oma} zu\iw{zu!Grad} schnell.
%% \ex Das Wasser ist \blau{dem Baby} warm genug.\iw{genug!Grad}
%% \zl
%% \end{itemize}
%% }

%% \frame[shrink=10]{
%% \frametitle{Zuweisung semantischer Kasus durch das Verb?}

%% \begin{itemize}
%% \item
%% Haider: Zuweisung durch Verb in (\mex{1}) nicht sinnvoll:
%% \ea
%% Sie hörten \blau{den ganzen Tag} dieselbe Schallplatte.
%% \z
%% Zeitangaben kommen auch in adjektivischen und nominalen Umgebungen vor:
%% % zitiert Toman83
%% \eal
%% \ex die Ereignisse \blau{letzten Sommer}
%% \ex der Flirt \blau{vorigen Dienstag}
%% \ex die \blau{diesen Sommer} sehr günstige Witterung
%% \ex die \blau{diesen Sommer} sehr teuren Urlaubsreisen
%% \zl
%% NPen mit strukturellem Kasus müssen in Nominalumgebungen
%% Genitiv sein. $\to$\\
%% In (\mex{0}) keine Zuweisung von strukturellem Kasus.

%% \pause
%% \item
%% Die Kasus in (\mex{0}) werden nicht aufgrund ihres Vorkommens in einer bestimmten
%% syntaktischen Struktur zugewiesen,\\
%% sondern sind vielmehr durch die Bedeutung des Nomens bestimmt.
%% \end{itemize}
%% }

%% \frame{
%% \frametitle{Akkusativ und Genitiv}

%% %\citep*{ZMT85a} -> semantische Kasusmarkierung
%% Der freie Akkusativ kommt bei Maß"-angaben\is{Maßangaben} (Zeitdauer und Zeitpunkt)
%% vor (\mex{1}) und Genitiv bei Lokalangaben oder Zeitangaben (\mex{2}).
%% \eal
%% \ex Sie studierte \blau{den ganze Abend}.
%% \ex \blau{Nächsten Monat}\iw{Monat} kommen wir.
%% \zl
%% \eal
%% \ex Ein Mann kam \blau{des Weges}.\iw{Weg}
%% \ex \blau{Eines Tages}\iw{Tag} sah ich sie wieder.
%% \zl
%% }




%% \subsection{Kongruenzkasus}

%% \frame{
%% \frametitle{Kongruenzkasus}

%% \begin{itemize}
%% \item Zwei Akkusative?
%% \eal
%% \ex Er nannte \blau{ihn} \rot{einen Experten}.
%% \ex \blau{Er} wurde \rot{ein Experte} genannt.
%% \zl
%% \pause
%% \item Wären das zwei unabhängige Akkusative,\\
%%       würde sich bei Passivierung nur einer ändern.

%% \pause
%% \item Kasus von \emph{einen Experten} wird \blau{Kongruenzkasus} genannt.\\
%% Die prädikative Phrase \emph{einen Experten} stimmt mit dem
%% Element,\\ über das prädiziert wird, im Kasus überein. 
%% \end{itemize}
%% }

%% \frame{
%% \frametitle{Kongruenzkasus mit Präpositionen}

%% Ähnliche Effekte kann man mit den Präpositionen \emph{als} und \emph{wie}
%% beobachten.
%% \eal
%% \ex \blau{Er} gilt als \rot{großer Künstler}.\footnote{
%%         \citew[S.\,203--204]{Heringer73a}.
%%       }
%% \ex Man läßt \blau{ihn} als \rot{großen Künstler} gelten\iw{gelten als}.
%% \zl
%% \pause
%% \eal
%% \ex Ich sehe \blau{ihn} als \rot{meinen Freund} an.\iw{ansehen}\footnote{
%%         \citew*[S.\,154]{SS88a}.
%% }
%% \ex \blau{Er} wird als \rot{mein Freund} angesehen.
%% \zl
%% }

%% \frame{
%% \frametitle{Kongruenzkasus mit Adjunkten}

%% Wie bei den prädikativen Argumenten gibt es auch Kongruenzkasus bei Adjunkten:
%% \eal
%% \ex Sie verhielt\iw{verhalten} \blau{sich} wie \blau{ihr Vater}.
%% \ex Ich behandelte\iw{behandeln} \blau{ihn} wie \blau{meinen Bruder}.
%% \ex Ich half\iw{helfen} \blau{ihm} wie \blau{einem Freund}.
%% \ex Ich erinnerte\iw{erinnern} mich \blau{dessen} wie \blau{eines fernen Alptraums}.
%% \zl

%% }

%% \frame{
%% \frametitle{Prädikation = Kasuskongruenz?}

%% \begin{itemize}
%% \item Kongruieren prädikative Phrasen immer mit dem Element,\\
%%       über das sie prädizieren?
%% \item Dies würde sofort auch Beispiele wie das in (\mex{1}) erklären:
%% \ea
%% Er wird ein großer Linguist.
%% \z
%% \pause
%% \item In AcI"=Konstruktionen müßten beide NPen im Akkusativ stehen.\\
%% Das ist nicht der Fall:
%% \eal
%% %\ex Laß ihn einen großen Linguisten werden.\label{bsp-lass-ihn-einen-grossen}
%% \ex Laß\iw{lassen|(} den wüsten Kerl [\ldots] meinetwegen ihr Komplize sein.\footnote{
%%         (\ref{bsp-lass-den-wuesten-kerl}) und (\ref{bsp-lass-mich}) sind aus dem \citet*[{\S}\,6925]{Duden66}.\iaf{Duden} %\citet*[{\S}\,1473]{Duden73}.\iaf{Duden}
%%         Die Quellen finden sich dort.
%%       }\label{bsp-lass-den-wuesten-kerl}
%% \ex Laß mich dein treuer Herold sein.\label{bsp-lass-mich}
%% \ex Baby, laß\iw{lassen|)} mich dein Tanzpartner sein.\footnote{
%%         Funny van Dannen, Benno-Ohnesorg-Theater, Berlin, Volksbühne, 11.10.1995
%%         }
%% \zl
%% \pause
%% \item
%% $\to$ Nominativ des Nicht-Subjekts in Kopulakonstruktionen ist\\
%%       ein lexikalischer Kasus \citep[S.\,54]{Thiersch78a}.
%% \end {itemize}
%% }

%% \subsection{Der Kasus nicht ausgedrückter Subjekte}
%% \label{sec-kasus-nicht-realisierter-subj}

%% \frame{
%% \frametitle{Der Kasus nicht ausgedrückter Subjekte (I)}
%% \savespace

%% \begin{itemize}
%% \item \citet*[Kapitel~6]{Hoehle83}:\\
%% Kasus nicht an der Oberfläche auf"|tretender Elemente bestimmbar.

%% {\em ein- nach d- ander-\/} kann sich auf mehrzahlige Konstituenten beziehen. 

%% Dabei muss Kasus und Genus mit der Bezugsphrase übereinstimmen.
%% \pause
%% \item In (\mex{1}) Bezug auf Subjekte bzw.\ Objekte:
%% \eal
%% \ex Die Türen sind eine nach der anderen kaputtgegangen.
%% \ex Einer nach dem anderen haben wir die Burschen runtergeputzt.
%% \ex Einen nach dem anderen haben wir die Burschen runtergeputzt.
%% \ex Ich ließ die Burschen einen nach dem anderen einsteigen.
%% \ex Uns wurde einer nach der anderen der Stuhl vor die Tür gesetzt.
%% \zl
%% \end{itemize}
%% }

%% \frame{
%% \frametitle{Der Kasus nicht ausgedrückter Subjekte (II)}
%% \savespace

%% In (\mex{1}) Bezug auf Dativ- bzw.\ Akkusativobjekte
%% eingebetteter Infinitive:

%% \eal
%% \ex Er hat uns gedroht, die Burschen demnächst einen nach dem anderen wegzuschicken.
%% \ex Er hat angekündigt, uns dann einer nach der anderen den Stuhl vor die Tür zu setzen.
%% \ex Es ist nötig, die Fenster, sobald es geht, eins nach dem anderen auszutauschen.
%% \zl

%% }

%% \frame{
%% \frametitle{Der Kasus nicht ausgedrückter Subjekte (III)}
%% \savespace

%% In (\mex{1}) Bezug auf Subjekt innerhalb der Infinitiv"=VP:
%% \eal
%% \ex Ich habe den Burschen geraten, im Abstand von wenigen Tagen einer nach dem anderen
%%       zu kündigen.
%% \ex Die Türen sind viel zu wertvoll, um eine nach der anderen verheizt zu werden.
%% \ex Wir sind es leid, eine nach der anderen den Stuhl vor die Tür gesetzt zu kriegen.
%% \ex Es wäre fatal für die Sklavenjäger, unter Kannibalen zu fallen und einer nach dem
%%       anderen verspeist zu werden.
%% \zl
%% {\em ein- nach d- ander-\/} im Nominativ $\to$\\
%% Das nicht realisierte Subjekt steht ebenfalls im Nominativ.

%% }


%% \frame{
%% \frametitle{Der Kasus nicht ausgedrückter Subjekte (IV)}

%% Dasselbe gilt für nicht realisierte Subjekte von adjektivischen Partizipien:
%% \eal
%% \ex die eines nach dem anderen einschlafenden Kinder
%% \ex die einer nach dem anderen durchstartenden Halbstarken
%% \ex die eine nach der anderen loskichernden Frauen
%% \zl
%% }

%% \frame{
%% \frametitle{Der Kasus nicht ausgedrückter Subjekte (V)}

%% Man muss also sicherstellen, daß auch nicht realisierte Subjekte Kasus zugewiesen bekommen.
%% Würde man diesen Kasus unterspezifiziert lassen, würden Sätze wie (\mex{1}) falsch analysiert werden.
%% \judgewidth{\#}
%% \ea[\#]{
%% Ich habe den Burschen geraten, im Abstand von wenigen Tagen einen nach dem anderen zu kündigen.
%% }
%% \z
%% In der zulässigen Lesart von (\mex{0}) ist die Phrase 
%% \emph{einen nach dem anderen} das Objekt von \emph{kündigen} und kann
%% sich nicht auf das Subjekt des Infinitivs, das referenzidentisch
%% mit \emph{den Burschen} ist, beziehen.

%% }





\subsection{Case of arguments}

\subsubsection{The case principle}

\frame{
\frametitle{Das Kasusprinzip (I)}

\begin{itemize}
\item Dative is regarded as a lexical case.

\pause

\item All arguments are represented in a list in all languages.\\
\textsc{argument-structure} list or \argst.

\pause

\item ditransitive verb like \word{geben} has the \argstw:

\ea
	\sliste{ NP[\str], NP[\ldat], NP[\str] }
\z

\type{str} stands for structural case and \type{ldat} for lexical dative.

\pause

\item For SVO languages, the first argument is the subject (\spr), the others \comps.

In the SOV languages, all \argst elements in finite verbs are in \comps.
%\pause
\end{itemize}
}


\frame{
\frametitle{The case principle (II)}

\begin{itemize}

\item
The assignment of structural cases is governed by the following principle (\citealp{Prze99}; 
\citealp{Meurers99b}):


\begin{prinzip-break}[\hypertarget{case-p}{case principle}]\is{Prinzip!Kasus-}
\label{case-p}
\begin{itemize}
\item In a list that contains both the subject and the complements of a verbal head, the leftmost element with structural nominative case\is{Kasus!Nominativ} is assigned, unless it is raised by a superordinate head.
\item All other elements in the list that are not raised and have a structural case are given accusative case\is{Kasus!Akkusativ}.
\item In nominal environments, elements with a structural case are assigned the genitive case\is{Kasus!Genitiv}.
\end{itemize}
\end{prinzip-break}

\bigskip
\item Principle goes back to \citet*{YMJ87}.
\end{itemize}
}

%% \frame{
%% \frametitle{Das Kasusprinzip (III)}

%% \begin{itemize}[<+->]
%% \item Prinzip ähnelt sehr stark dem von \citet*{YMJ87} und kann
%% damit auch die Kasussysteme verschiedener Sprachen erklären,\\
%% die von den genannten Autoren besprochen wurden,\\
%% insbesondere auch das komplizierte Kasussystem des Isländischen\il{Isländisch}.
%% \item
%% Ein wesentlicher Unterschied ist, daß das Prinzip~\ref{case-p} monoton ist,\\
%% \dash Kasus, die einmal zugewiesen wurden,\\
%% werden nicht von einem übergeordneten Prädikat überschrieben.
%% \end{itemize}

%% }

\subsubsection{Aktive}

\frame{
\frametitle{Aktive}

prototypical valency lists:
\ea
\begin{tabular}[t]{@{}l@{~}l@{~}l}
a. & \emph{schläft}:     & \argst \sliste{ NP[\type{str}]$_i$ }\\
b. & \emph{unterstützt}: & \argst \sliste{ NP[\type{str}]$_i$, NP[\type{str}]$_j$ }\\
c. & \emph{hilft}:       & \argst \sliste{ NP[\type{str}]$_i$, NP[\type{ldat}]$_j$ }\\
d. & \emph{schenkt}:     & \argst \sliste{ NP[\type{str}]$_i$, NP[\type{ldat}]$_j$, NP[\type{str}]$_k$ }\\
\end{tabular}
\z
\pause
The first element in the \argstl gets nominative.\\
All others with structural case get accusative.

\pause
For the comparison with the passive, it makes sense\\
to provide the NPs with small indices (i, j, k).

}

\subsubsection{Passive}

\frame[shrink=5]{
\frametitle{Passive}

\ea
\begin{tabular}[t]{@{}l@{~}l@{~}l}
a. & \emph{schläft}:     & \argst \sliste{ NP[\type{str}]$_i$ }\\
b. & \emph{unterstützt}: & \argst \sliste{ NP[\type{str}]$_i$, NP[\type{str}]$_j$ }\\
c. & \emph{hilft}:       & \argst \sliste{ NP[\type{str}]$_i$, NP[\type{ldat}]$_j$ }\\
d. & \emph{schenkt}:     & \argst \sliste{ NP[\type{str}]$_i$, NP[\type{ldat}]$_j$, NP[\type{str}]$_k$ }\\
\end{tabular}
\z

Passivizing the verbs results in the following \argst"=lists:

\ea
\begin{tabular}[t]{@{}l@{~}l@{~}l}
a. & \emph{geschlafen wird}:  & \argst \sliste{ }\\
b. & \emph{unterstützt wird}: & \argst \sliste{ NP[\type{str}]$_j$ }\\
c. & \emph{geholfen wird}:    & \argst \sliste{ NP[\type{ldat}]$_j$ }\\
d. & \emph{geschenkt wird}:   & \argst \sliste{ NP[\type{ldat}]$_j$, NP[\type{str}]$_k$ }\\
\end{tabular}
\z

In (\mex{0}) another NP is now in first place.\\
First NP with structural case gets it nominative.\\
Lexical case as in (\mex{0}c--d) remains as it is,
namely lexically specified.
}

%% \subsubsubsection{Dativpassiv}

%% \frame[shrink=15]{
%% \frametitle{Dativpassiv}

%% Bei der Kombination von \emph{geholfen} und
%% \emph{bekommen} bzw.\ von \emph{geschenkt} und \emph{bekommen} wird das Dativargument von 
%% \emph{geholfen} bzw.\ von \emph{geschenkt} zum ersten Argument gemacht und der lexikalische
%% Dativ beim eingebetteten Verb wird zu einem strukturellen Kasus beim Passiv"=Hilfsverb:
%% \ea
%% \begin{tabular}[t]{@{}l@{~}l@{~}l}
%% c. & \emph{hilft}:       & \argst \sliste{ NP[\type{str}]$_j$, NP[\type{ldat}]$_k$ }\\
%% d. & \emph{schenkt}:     & \argst \sliste{ NP[\type{str}]$_j$, NP[\type{str}]$_k$, NP[\type{ldat}]$_l$ }\\
%% \end{tabular}
%% \z
%% \ea
%% \begin{tabular}[t]{@{}l@{~}l@{~}l}
%% a. & \emph{geholfen bekommt}:    & \argst \sliste{ NP[\type{str}]$_k$ }\\
%% b. & \emph{geschenkt bekommt}:   & \argst \sliste{ NP[\type{str}]$_l$, NP[\type{str}]$_k$ }\\
%% \end{tabular}
%% \z
%% Details kommen im Kapitel über Passiv.

%% Kasusvergabe: Dadurch, daß das Dativargument an erster Stelle in der Valenzliste\\
%% von \emph{geholfen bekommen} bzw.\ von \emph{geschenkt bekommen} steht, kriegt es
%% Nominativ. 

%% Bei \emph{geschenkt bekommen} bekommt das zweite Element (das direkte Objekt) Akkusativ.

%% Die Umwandlung eines lexikalischen in einen strukturellen Kasus ist unschön,\\
%% es scheint zur Zeit jedoch keine bessere Alternative zu geben. 

%% }

%% \frame{
%% \frametitle{AcI-Konstruktionen (I)}
%% \smallframe

%% Bei der Analyse der AcI"=Konstruktion findet eine Argumentkomposition statt:\\
%% die Argumente des eingebetteten Verbs werden zu Argumenten des AcI"=Verbs:

%% \ea
%% \begin{tabular}[t]{@{}l@{~}l@{~}l}
%% a. & \emph{schläft}:     & \argst \sliste{ NP[\type{str}]$_j$ }\\
%% b. & \emph{unterstützt}: & \argst \sliste{ NP[\type{str}]$_j$, NP[\type{str}]$_k$ }\\
%% c. & \emph{hilft}:       & \argst \sliste{ NP[\type{str}]$_j$, NP[\type{ldat}]$_k$ }\\
%% d. & \emph{schenkt}:     & \argst \sliste{ NP[\type{str}]$_j$, NP[\type{str}]$_k$, NP[\type{ldat}]$_l$ }\\
%% \end{tabular}
%% \z
%% \ea
%% %{\small
%% \begin{tabular}[t]{@{}l@{~}l@{~}l@{}}
%% a. & \emph{schlafen läßt}:     & \argst \sliste{ NP[\str]$_i$, NP[\type{str}]$_j$ }\\
%% b. & \emph{unterstützen läßt}: & \argst \sliste{ NP[\str]$_i$, NP[\type{str}]$_j$, NP[\type{str}]$_k$ }\\
%% c. & \emph{helfen läßt}:       & \argst \sliste{ NP[\str]$_i$, NP[\type{str}]$_j$, NP[\type{ldat}]$_k$ }\\
%% d. & \emph{schenken läßt}:     & \argst \sliste{ NP[\str]$_i$, NP[\type{str}]$_j$, NP[\type{str}]$_k$, NP[\type{ldat}]$_l$ }\\
%% \end{tabular}
%% %}
%% \z

%% NP[\str]$_i$ steht dabei jeweils für das Subjekt des AcI-Verbs.\\ 
%% NP[\type{str}]$_j$, NP[\type{str}]$_k$ bzw.\ NP[\type{ldat}]$_l$ sind die Argumente des eingebetteten
%% Verbs. 
%% }

%% \frame{
%% \frametitle{AcI-Konstruktionen (II)}
%% \smallframe

%% \ea
%% %{\small
%% \begin{tabular}[t]{@{}l@{~}l@{~}l@{}}
%% a. & \emph{schlafen läßt}:     & \argst \sliste{ NP[\str]$_i$, NP[\type{str}]$_j$ }\\
%% b. & \emph{unterstützen läßt}: & \argst \sliste{ NP[\str]$_i$, NP[\type{str}]$_j$, NP[\type{str}]$_k$ }\\
%% c. & \emph{helfen läßt}:       & \argst \sliste{ NP[\str]$_i$, NP[\type{str}]$_j$, NP[\type{ldat}]$_k$ }\\
%% d. & \emph{schenken läßt}:     & \argst \sliste{ NP[\str]$_i$, NP[\type{str}]$_j$, NP[\type{str}]$_k$, NP[\type{ldat}]$_l$ }\\
%% \end{tabular}
%% %}
%% \z

%% Für die Kasusvergabe sind nur die Valenzlisten in (\mex{0}) relevant. 

%% Die Argumente in den Valenzlisten der eigentlichen Verben spielen für die Kasusvergabe keine Rolle, 
%% da das Kasusprinzip die Kasuszuweisung ausschließt,
%% wenn ein Element angehoben wird. 

%% Das erste Element in den Listen in (\mex{0}) bekommt immer Nominativ,\\
%% die restlichen Elemente mit strukturellem Kasus bekommen Akkusativ. 

%% Die logischen Subjekte der eingebetteten V werden also im Akkusativ realisiert.

%% }

%% \frame{
%% \frametitle{Adjektivsubjekte}


%% Die Kasuszuweisungen an das Subjekt von Adjektiven funktioniert analog. Die Kopula wird mit dem Adjektiv
%% verbunden, und es entsteht eine Valenzliste, die die Argumente des Adjektivs enthält (\mex{1}a).\\
%% Wird ein solcher Komplex noch unter ein AcI"=Verb wie \emph{sehen} eingebettet,\\
%% erhält man die Liste in (\mex{1}b):
%% \ea
%% \begin{tabular}[t]{@{}l@{~}l@{~}l}
%% a. & \emph{kleiner werden}:     & \argst \sliste{ NP[\str]$_j$ }\\
%% b. & \emph{kleiner werden sah}: & \argst \sliste{ NP[\str]$_i$, NP[\type{str}]$_j$ }\\
%% \end{tabular}
%% \z
%% Die Kasuszuweisung funktioniert analog zu den bereits diskutierten Fällen. In den verbalen Umgebungen
%% der Kopula bzw.\ des AcI"=Verbs bekommen die NPen mit strukturellem Kasus Nominativ bzw.\ Akkusativ.%
%% }


%% \subsection{Semantischer Kasus}
%% \frame{
%% \frametitle{Semantischer Kasus (I)}


%% Der Kasus von NPen wie \emph{den ganzen Tag} in (\mex{1}) ist von der syntaktischen Umgebung unabhängig.
%% \eal
%% \ex Sie arbeiten den ganzen Tag.
%% \ex Den ganzen Tag wird gearbeitet, [\ldots].\footnote{
%%   \url{http://www.philo-forum.de/philoforum/viewtopic.html?p=146060}. \urlchecked{12}{05}{2005}.
%% }
%% \zl
%% Daß die NP im Akkusativ steht, hängt mit ihrer Funktion zusammen. 

%% Unterschiedliche Lexikoneinträge für \emph{Tag} in (\mex{0}) und (\mex{1}):
%% \eal
%% \ex Ich liebe diesen Tag.
%% \ex Dieser Tag gefällt mir.
%% \zl

%% In (\mex{0}) liegen ganz gewöhnliche Argumente vor,\\
%% in (\mex{-1}) dagegen ein Adjunkt. 
%% }

%% \frame{
%% \frametitle{Semantischer Kasus (II)}

%% Adjunkte unterscheiden sich von Argumenten durch ihren \modw und durch ihern \contw.

%% Für (\mex{-1}) muss es unter \cont eine Dauer-Relation geben.

%% Zusammen mit dieser Information wird im Lexikoneintrag für das modifizierende Nomen der Kasus fest kodiert. 

%% Die morphologische Komponente kann dann für diesen Lexikoneintrag nur die Akkusativform erzeugen, 
%% da alle anderen Flexionsformen mit der bereits im Lexikoneintrag angegebenen Kasusinformation inkompatibel sind. 

%% Dadurch wird sichergestellt, daß Sätze wie (\mex{1}) nicht analysiert werden:
%% \eal
%% \ex[*]{
%% Er arbeitet der ganze Tag.
%% }
%% \ex[*]{
%% weil der ganze Tag gearbeitet wurde
%% }
%% \zl

%% }


\frame{
\frametitle{Comparison German, Danish, English, Icelandic}

%Comparison German, Danish, English, Icelandic

\begin{itemize}
\item German and Icelandic allow subjectless constructions,\\
Danish and English do not.
\pause
\item German, Icelandic and Danish allow impersonal passive,\\
English does not.
\pause
\item Danish, Icelandic allow promotion of both objects to the subject,\\
German and English do not.
\pause
\item Danish and Icelandic have a morphological passive,\\
German and English do not.
\pause
\item German allows the remote passive, Danish has the complex passive and English and Danish have the reportive passive.

\end{itemize}

}


\subsection{Morphologische und analytische Formen}

\frame[shrink]{
\frametitle{Morphological and analytical forms in Danish}

\begin{itemize}
\item morphological passive: \suffix{s} suffix, present and past variants:

\eal
	\ex[]{\label{ex-laeseract}
		\gll Peter læser avisen.\\
     Peter liest Zeitung.{\sc def}\\
		\glt `Peter liest die Zeitung.'}
		
	\ex[]{\label{ex-laeses}
		\gll Avisen              læses af Peter.\\
		     Zeitung.{\sc def} liest.{\sc pres}.{\sc pass} von Peter\\
		\glt `Die Zeitung wird von Peter gelesen.'}
		
\ex[]{\label{ex-laestes}
	\gll Avisen            læs\gruen{t}es af Peter.\\
	     Zeitung.{\sc def} lesen.{\sc past}.{\sc pass} von Peter\\
	\glt `Die Zeitung wurde von Peter gelesen.'}
\zl

\pause
\item  analytical form with \emph{blive} + participle:

\ea
	\gll Avisen            bliver læst af Peter.\\
		     Zeitung.{\sc def} wird   gelesen von Peter\\
		\glt `Die Zeitung wird von Peter gelesen.'
\z

%% \pause
%% \item

%% The morphological passive may also apply to infinitives:
%% \ea
%% \gll Avisen skal læses hver dag.\\
%%       newspaper.def must read.{\sc inf}.{\sc pass} every day\\
%% \glt `The newspaper must be read every day.'

%% \z
\end{itemize}

}


\frame{
\frametitle{In German and English only analytical forms}


\begin{itemize}
\item English and German do not have a morphological passive:
\eal
	\ex The paper was read.
	\ex 
		\gll Der        Aufsatz wurde  gelesen.\\
			the.\NOM{} paper   \AUX{} read\\
\zl    

\end{itemize}

}



\subsection{Personal and impersonal passive}


\frame{
\frametitle{Personal passive}

\begin{itemize}
\item All the languages considered allow the promotion of an object NP to the subject.

\pause
\item The subject can also be S or VP:
\eal
	\ex
	\gll At regeringen træder tilbage, bliver påstået.\\
	     dass Regierung.{\sc def} tritt zurück wird behauptet\\
	\glt `Dass die Regierung zurücktritt, wird behauptet.'
	
\ex
	\gll At reparere bilen, bliver forsøgt.\\
	     zu reparieren Auto.{\sc def} wird versucht\\
	\glt `Das Auto zu reparieren, wird versucht.'
\zl

%% \pause
%% (We do not assume that Ss or Vs are subjects in German \citep{Reis82}).
\end{itemize}

}


\frame{
\frametitle{Impersonal passives in German and Icelandic}


\begin{itemize}
\item German, Danish and Icelandic have impersonal passives.

\pause
\item German simply as a subjectless construction:
\eal
	\gll weil    noch  getanzt wurde\\
	because still danced  \AUX\\
	\glt `because there was still dancing there'
\zl

\pause
\item Icelandic as well \citep[\page 264]{Thrainsson2007a-u}:
\eal
	\ex 
		\gll Oft var   talað      um   þennan mann.\\
		often was talked about this Mann.\ACC.\SG.\mas\\ \jambox{(Icelandic)}
		\glt `This man was often talked about.' \\ 
		     
\ex
	\gll Aldrei hefur verið    sofið      í  þessu  rúmi.\\
	never    has   been slept in this bed.\DAT\\
	\glt `This bed has never been slept in.'
\zl

\end{itemize}

}

\frame{
\frametitle{Impersonal passives in Danish: Expletive}


\begin{itemize}
\item Danish and English need a subject. Danish has a solution:

\eal
	\ex 
		\gll at \blau{der} bliver danset\\
		that \textsc{expl} \AUX{} danced\\ \jambox{(Danish)}
		\glt `that there is dancing'
		
	\ex
		\gll at \blau{der} danses\\
		     that \textsc{expl} dance.\textsc{pres}.\textsc{pass}\\
		     \glt `that there is dancing'
\zl

\pause
\item In German, an expletive subject is excluded:
\nocite{MOe2011a}
\ea[*]{
		\gll weil es noch getanzt wurde \\
			because \textsc{expl} noch danced \AUX \\
}
\z
%% \eal
%% \ex[*]{ 
%% \gll Bliver danset.\\
%%      is danced\\
%% }
%% \ex[*]{
%% \gll Danses.\\
%%      dance.{\sc pass}\\
%% }
%% \zl


\end{itemize}

}



%% The examples in (\ref{ex-gearbeitet-wurde}) and (\ref{ex-bliver-arbejder}) show passives of
%% mono-valent verbs but of course bi-valent intransitive verbs like the German \emph{denken} `think'
%% and Danish \emph{passe} `take care of' also form impersonal passives:
%% \ea
%% \gll dass an die Männer gedacht wurde\\
%%      that {\sc prep} the men thought was\\
%% \glt `that one thought about the men'
%% \z
%% \eal
%% \label{ex-impersonal-passive-pp}
%% \ex
%% \gll Der passes på børnene.\\
%%      {\sc expl} take.care.of.{\sc pres}.{\sc pass} on children.{\sc def}\\
%% \glt `Somebody takes care of the children.'
%% \ex
%% \gll Der bliver passet  på børnene.\\
%%      {\sc expl} is taken.care.of on children.{\sc def}\\
%% \glt `Somebody takes care of the children.'
%% \zl


\subsection{Promotion of the primary and secondary object}


\frame{
\frametitle{Primary and secondary object in German and English}

\smallexamples

\begin{itemize}
\item German and English only allow the promotion of one object:

\eal
	\ex[]{
		\gll weil der Mann \rot{dem} \rot{Jungen} \gruen{den} \gruen{Ball} schenkt \\
			 because the.\NOM{} man the.\DAT{} boy the.\ACC{} ball gives\\
			\glt `because the man gives the boy a ball as a present'
}

	\ex[]{
		\gll weil \rot{dem} \rot{Jungen} \gruen{der} \gruen{Ball} geschenkt wurde \\
			because the.\DAT{} boy    the.\NOM{} ball given     \AUX \\
			\glt `because the ball was given to the boy'
}

	\ex[*]{
		\gll weil \rot{der} \rot{Junge} \gruen{den} \gruen{Ball} geschenkt wurde \\
			because the.\NOM{} boy the.\ACC{} ball given \AUX\\
}

\zl
\end{itemize}

}

\frame{
\frametitle{Primary and secondary object in German and English}

\begin{itemize}
\item English: only an object can become a subject:

\eal
\ex[]{
because the man gave \rot{the boy} \gruen{the ball}
}
\ex[]{
because \rot{the boy} was given \gruen{the ball}
}
\ex[*]{
because \gruen{the ball} was given \rot{the boy}
}
\zl

\pause
\item However, the effect can be achieved by using a different valency pattern or \emph{get}-passive
\eal
	\ex because the man gave the ball \blau{to} the boy
	\ex because the ball was given \blau{to} the boy
\zl
\end{itemize}

}

\frame{
\frametitle{Primary and secondary object in Danish}

\begin{itemize}
\item In Danish, both objects can become the subject:
\eal
	\ex 
	\gll fordi manden giver \rot{drengen} \gruen{bolden}\\ 
	     weil Mann.{\sc def} gibt Junge.{\sc def} Ball.{\sc def}\\
	\glt `weil der Mann dem Jungen den Ball gibt'
	
\ex\label{ex-boy-was-given-ball-danish}
	\gll fordi \rot{drengen} bliver givet \gruen{bolden}\\ 
	     weil Junge.{\sc def} wird gegeben Ball.{\sc def}\\
	\glt `weil der Junge den Ball gegeben bekommt'
	
\ex\label{ex-ball-was-given-boy-danish}
	\gll fordi \gruen{bolden} bliver givet \rot{drengen}\\ 
	     weil Ball.{\sc def} wird gegeben Junge.{\sc def}\\
	\glt `weil der Ball dem Jungen gegeben wird'
	
\zl
\pause
\item Danish is different from Moro, for example \citep{AMM2017a-u}:\\
Objects are clearly differentiated. For example, their order is fixed:

\ea[*]{
fordi manden giver \gruen{bolden} \rot{drengen}
}
\z

\end{itemize}

}


\frame[shrink=2]{
\frametitle{Primary and secondary object in Icelandic}


\begin{itemize}
\item \citet*[\page 460]{ZMT85a}:\\
      The dative object can become an oblique subject:

\ea
	\gll Konunginum voru gefnar ambáttir.\\
	     the.king.\DAT{} were given.\fem.\PL{} maidservants.\NOM{}.\fem.\PL \\ \jambox{[S$_i$ Aux \_$_i$ V O]} 
	     \glt `The king was given female slaves.'
\z

The accusative object then gets the nominative case.

\pause
\medskip

\item Or the accusative object becomes the subject:
\ea
	\gll Ambáttin var gefin konunginum.\\
	     the.maidservant.\NOM{}.\SG{}  \AUX{} given.\fem.\SG{} the.king.\DAT\\  \jambox{[S$_i$ Aux \_$_i$ V O]} 
		\glt `The female slave was given to the king.'
\z

\pause
\item Side note: Verb always congruent with nominative.

\end{itemize}

}




\subsection{Designated Argument Reduction}


\frame{
\frametitle{Designated Argument Reduction}

\begin{itemize}
\item \citet{Haider86,HM94a,Mueller2003e}:\\
{\sc designated argument} ({\sc da}) the subject of transitive and unergative verbs. (a
``real'' subject)

\pause
\item \daw of unaccusative verbs is the empty list.

\pause
\item Passive = LR that subtracts the \dalist from the argument structure of the input verb or \ stem.


\ea
%\resizebox{\linewidth}{!}
\z

\end{itemize}

}

%% \frame{
%% \frametitle{Kasuszuweisung}

%% \begin{itemize}
%% \item 
%% Lexikalischer Kasus bleibt bei Passivierung unverändert:
%% \eal
%% \ex
%% weil der Mann ihm geholfen hat
%% \ex
%% weil ihm geholfen wurde
%% \zl

%% \end{itemize}

%% }

\frame{
\frametitle{Designated Argument Reduction}

\begin{itemize}
\item Participle formation rule:
\ea
\label{lr-passive-prelim}
Lexical rule for the formation of the participle (provisional):
\ms[stem]{
head   & \ms[verb]{ da & \ibox{1}\\
                  }\\
arg-st & \ibox{1} $\oplus$ \ibox{2} \\
} $\mapsto$
\ms[word]{
arg-st & \ibox{2} \\
}
\z
\pause
\item 
The designated argument is blocked.
\end{itemize}



}


\frame{
\frametitle{Designated Argument Reduction}

\begin{itemize}

\item \argstl of the participle is either empty or begins with the object of the active form:

\ea
\label{partizipien-hm}
%\resizebox{\linewidth}{!}
\z

\item The first element of the \argstl with structural case gets nominative case:

\eal
	\gll Der Aufsatz wurde gelesen. \\
		the.\NOM{} paper   \AUX{} read\\
\zl

\end{itemize}

}


\frame{
\frametitle{English: Promotion of the first object}

\begin{itemize}
\item English: no dative, structural case for first object,\\
lexical accusative for second object of \emph{give}

\ea\label{da-repr-hm-English}
%\resizebox{\linewidth}{!}{%
\begin{tabular}[t]{@{}l@{ }l@{ }l@{ }l@{ }l@{}}
  &                     & {\sc arg-st}\\[2mm]
b.&dance   (unerg):     & \liste{NP[\type{str}]}\\[2mm]
%c.&auf"|fallen (unacc): & \liste{}                         & \liste{NP[\type{str}], NP[\type{ldat}]}\\[2mm]
c.&read      (trans):   & \liste{NP[\type{str}], NP[\type{str}]}\\[2mm]
d.&give      (ditrans): & \liste{NP[\type{str}], NP[\type{str}], \blaubf{NP[{\it lacc\/}]}}\\[2mm]
e.&help      (trans):   & \liste{NP[\type{str}], \blaubf{NP[\type{str}]}}\\
\end{tabular}
%}
\z

\pause
\item German can make the second object (accusative) the subject,\\
English the first (the object that is closer to the verb, OV vs.\ VO):

\eal
	\ex dass \gruen{dem Jungen} \rot{der Ball} gegeben wurde
	\ex because \rot{the boy} was given \gruen{the ball}
\zl

\end{itemize}
}

\frame{
\frametitle{English: Personal passive with \emph{help}}

\begin{itemize}
\item English: no dative, structural case for first object,\\
lexical accusative for second object of \emph{give}

\ea\label{da-repr-hm-English}
%\resizebox{\linewidth}{!}{%
\begin{tabular}[t]{@{}l@{ }l@{ }l@{ }l@{ }l@{}}
  &                     & {\sc arg-st}\\[2mm]
b.&dance   (unerg):     & \liste{NP[\type{str}]}\\[2mm]
%c.&auf"|fallen (unacc): & \liste{}                         & \liste{NP[\type{str}], NP[\type{ldat}]}\\[2mm]
c.&read      (trans):   & \liste{NP[\type{str}], NP[\type{str}]}\\[2mm]
d.&give      (ditrans): & \liste{NP[\type{str}], NP[\type{str}], \blaubf{NP[{\it lacc\/}]}}\\[2mm]
e.&help      (trans):   & \liste{NP[\type{str}], \blaubf{NP[\type{str}]}}\\
\end{tabular}
%}
\z

\item German has an impersonal passive for \emph{helfen},\\
but English has a personal one:

\eal
	\ex weil ihm geholfen wurde
	\ex because he was helped
\zl

\end{itemize}
}


%% \frame{
%% \frametitle{Isländisch}

%% \begin{itemize}
%% \item Dativ und Genitiv sind lexikalisch:



%% \end{itemize}

%% }

\subsection{Primary and secondary objects}



\frame{
\frametitle{Danish: Promotion primary and secondary object}

\begin{itemize}
\item Danish is like English: no dative,\\
but allows the promotion of both objects of ditransitive verbs:

\ea\label{da-repr-hm-Danish}
%\resizebox{\linewidth}{!}
\z

Danish has two objects with a structural case,\\
German and English only one.

\pause
\item Personal passive: Promotion of an object with a structural case.

\end{itemize}

}

\frame[shrink=15]{
\frametitle{Generalized lexicon rule}


\begin{itemize}
\item old:

\ea
\label{lr-passive-prelim}
Lexicon rule for the formation of the participle (provisional):\\
\ms[stem]{
head   & \ms[verb]{ da & \ibox{1}\\
                  }\\
arg-st & \ibox{1} $\oplus$ \ibox{2} \\
} $\mapsto$
\ms[word]{
arg-st & \ibox{2} \\
}
\z

First argument suppressed, second is now the first.

\pause
\item \emph{promote} provides the list \ibox{3}, which either corresponds to the list \ibox{2}
or if \ibox{2} contains two NPs with structural case, additionally also
a list in which the order of the two NPs is swapped, \dash, the second NP with
structural case is placed first.


\ea
Passiv-Lexikonregel für Dänisch, Deutsch, Englisch, Isländisch:\\
\ms[stem]{
head   & \ms[verb]{ da & \ibox{1}\\
                  }\\
arg-st & \ibox{1} $\oplus$ \ibox{2} \\
} $\mapsto$
\ms[word]{
arg-st & \ibox{3} \\
} $\wedge$ promote(\ibox{2}, \ibox{3})
\z

\end{itemize}

}

\frame{
\frametitle{Result of the lexicon rule application for Danish}

\begin{tabular}[t]{@{}l@{ }l@{ }l}
  &                        & \textsc{arg-st}\\[2mm]
%a.&ankomme (unacc):       & \liste{}                         & \liste{NP[\type{str}]}\\[2mm]
a.&danset/-s   (tanzen, unerg):     & \liste{}\\[2mm]
%c.&auf"|fallen (unacc): & \liste{}                         & \liste{NP[\type{str}], NP[\type{ldat}]}\\[2mm]
b.&læst/-s      (lesen, trans):   &  \liste{NP[\type{str}]$_j$ } \\[2mm]
c.&givet/-s      (geben, ditrans): & \liste{NP[\type{str}]$_j$, NP[\type{str}]$_k$ } \\[2mm]
  &                         & \liste{NP[\type{str}]$_k$, NP[\type{str}]$_j$ } \\[2mm]
d.&hjulpet/-s    (helfen, trans):   & \liste{NP[\type{str}]$_j$ }                    \\
\end{tabular}

}

%% \frame{
%% \frametitle{Doppelobjektkonstruktionen im Isländischen}

%% \begin{itemize}
%% \item Valenzspezifikation im Isländischen identisch mit der Deutschen:
%% \ea
%% \label{partizipien-hm}
%% %\resizebox{\linewidth}{!}{%
%% \begin{tabular}[t]{@{}l@{ }ll@{ }l@{}}
%%   &                   & {\sc arg-st}\\[2mm]
%% a.&tanzen    (unerg): & \liste{ NP[\type{str}]$_i$ }\\[2mm]
%% b.&lesen     (trans): & \liste{ NP[\type{str}]$_i$, NP[\type{str}]$_j$ }\\[2mm]
%% c.&geben   (ditrans): & \liste{ NP[\type{str}]$_i$, NP[\type{ldat}]$_j$ , NP[\type{str}]$_k$ }\\[2mm]
%% d.&helfen    (unerg): & \liste{ NP[\type{str}]$_i$, NP[\type{ldat}]$_j$ }\\
%% \end{tabular}
%% %}
%% \z
%% \pause
%% \item einziger Unterschied: Isländisch erlaubt das Mapping von NPen mit lexikalischem Kasus auf \spr.


%% \end{itemize}

%% }


%% \frame{
%% \frametitle{Doppelobjektkonstruktion im Deutschen/Isländischen}







%% }


\subsection{Icelandic: Oblique subjects}

\frame{
\frametitle{Icelandic}

\begin{itemize}
\item Case distribution as in German:

\ea\label{da-repr-hm-English}
%\resizebox{\linewidth}{!}{%
\begin{tabular}[t]{@{}l@{ }l@{ }l@{ }l@{ }l@{}}
  &                     & {\sc arg-st}\\[2mm]
b.&dansa   (unerg):     & \liste{NP[\type{str}]}\\[2mm]
%c.&auf"|fallen (unacc): & \liste{}                         & \liste{NP[\type{str}], NP[\type{ldat}]}\\[2mm]
c.&lesa    (trans):   & \liste{NP[\type{str}], NP[\type{str}]}\\[2mm]
d.&gefa    (ditrans): & \liste{NP[\type{str}], \blaubf{NP[\type{ldat}]}, \blaubf{NP[\type{str}]}}\\[2mm]
e.&hjálpa  (trans):   & \liste{NP[\type{str}], \blaubf{NP[\type{ldat}]}}\\
\end{tabular}
%}
\z
\item Impersonal passive is the same as \emph{tanzen},\\
but \emph{helfen} does not form an impersonal passive but a personal passive.

\pause

\item \emph{geben} allows two variants:\\
Dative becomes oblique subject, accusative becomes subject.

\end{itemize}


}


\frame{
\frametitlefit{Icelandic: Oblique subjects and double object constructions}

\begin{itemize}
\item first NP becomes the subject, also NPs with lexical case\\ \citep[\page 147--148]{Wechsler95a-u}

\ea\label{da-repr-hm-Danish}
%\resizebox{\linewidth}{!}{%
\begin{tabular}[t]{@{}l@{ }l@{ }l@{ }l@{ }l@{~~~~~}l@{}}
  &                        & {\sc arg-st}                     & \spr   & \comps\\[2mm]
%a.&ankomme (unacc):       & \liste{}                         & \liste{NP[\type{str}]}\\[2mm]
a.& dansað    (unerg):     & \liste{}                        & \liste{ } & \liste{} \\[2mm]
%c.&auf"|fallen (unacc): & \liste{}                         & \liste{NP[\type{str}], NP[\type{ldat}]}\\[2mm]
b.& lesið     (trans):   &  \liste{NP[\type{str}]$_j$ }         & \liste{NP[\type{str}]$_j$ } & \eliste\\[2mm]
c.& gefið      (ditrans): & \liste{NP[\type{ldat}]$_j$, NP[\type{str}]$_k$ } & \liste{NP[\type{ldat}]$_j$ } & \liste{NP[\type{str}]$_k$ }\\[2mm]
  &                      & \liste{NP[\type{str}]$_k$, NP[\type{ldat}]$_j$ } & \liste{NP[\type{str}]$_k$ } & \liste{NP[\type{ldat}]$_j$ }\\[2mm]
d.& hjálpað    (trans):   & \liste{NP[\type{ldat}]$_j$ }                  & \liste{NP[\type{ldat}]$_j$ } & \liste{}\\
\end{tabular}
%}
\z
\end{itemize}

}


\subsection{Impersonal passive}
\label{sec-impersonals}

\frame[shrink]{
\frametitle{Impersonal passive}

\begin{itemize}
\item German, Icelandic: subject not obligatory

% Thrainsson 2007 264
      Danish: Introduction of an expletive when mapping \argst\ to \spr/\comps.
      
\pause
\item English and Danish map the first NP/VP/CP to \spr\ and the remaining arguments to \comps\
and\\
Danish inserts an expletive if there are no other elements,\\
that could function as a subject.
\nocite{BB2007a}

\ea\label{da-repr-hm-Danish}
%\resizebox{\linewidth}{!}{%
\begin{tabular}[t]{@{}l@{ }l@{ }l@{ }l@{ }l@{~~~~~}l@{}}
  &                        & {\sc arg-st}                     & \spr   & \comps\\[2mm]
%a.&ankomme (unacc):       & \liste{}                         & \liste{NP[\type{str}]}\\[2mm]
a.&danset/-s   (unerg):     & \liste{}                        & \liste{ \blau{NP$_{expl}$} } & \liste{} \\[2mm]
%c.&auf"|fallen (unacc): & \liste{}                         & \liste{NP[\type{str}], NP[\type{ldat}]}\\[2mm]
b.&læst/-s      (trans):   &  \liste{NP[\type{str}]$_j$ }         & \liste{NP[\type{str}]$_j$ } & \eliste\\[2mm]
c.&givet/-s      (ditrans): & \liste{NP[\type{str}]$_j$, NP[\type{str}]$_k$ } & \liste{NP[\type{str}]$_j$ } & \liste{NP[\type{str}]$_k$ }\\[2mm]
  &                         & \liste{NP[\type{str}]$_k$, NP[\type{str}]$_j$ } & \liste{NP[\type{str}]$_k$ } & \liste{NP[\type{str}]$_j$ }\\[2mm]
d.&hjulpet/-s    (trans):   & \liste{NP[\type{str}]$_j$ }                      & \liste{NP[\type{str}]$_j$ } & \liste{}\\
\end{tabular}
%}
\z
\end{itemize}

}

%% \section{The Lexical Rules}

%% The following lexical rule accounts for the participle formation in German and English:
%% \ea
%% Lexical rule for the formation of the participle in English and German:\\
%% \ms[stem]{
%% head   & \ms[verb]{ da & \ibox{1}\\
%%                   }\\
%% arg-st & \ibox{1} $\oplus$ \ibox{2} \\
%% } $\mapsto$
%% \ms[word]{
%% arg-st & \ibox{2} \\
%% }
%% \z
%% This rule blocks the designated argument, if there is one. The rule is an inflectional rule, that
%% is, it maps a stem onto a word and adds the morphology of the particple.
%% The passive auxiliary requires a particple that has a referential designated argument and hence the
%% passivization of unaccusatives is excluded. See Section~\ref{sec-auxiliary} for details.

%% The lexical rule for the passives in Danish cannot be that simple since we have to take care of the
%% insertion of an expletive in the cases in which there is no argument that can be promoted to
%% subject. Hence we have to distinguish two cases: One for the impersonal passive with an inserted
%% expletive and one for the personal passive. (\mex{1}) shows the lexical rule for the personal
%% passive:
%% \eas
%% Lexical rule for the personal passive in Danish:\\
%% \ms{
%% head   & \ms[verb]{ da & \ibox{1}\\
%%                   }\\
%% arg-st & \ibox{1} $\oplus$ \ibox{2} {\rm ( \sliste{ NP $\vee$ S } $\oplus$ \etag )} \\
%% } $\mapsto$\\
%% \flushright\ms{
%% arg-st & \ibox{2} \\
%% }
%% \zs
%% This rule is similar to the one for English and German, but it requires that the list \ibox{2}
%% starts with an NP or an S. The output of the rule has \ibox{2} as the value of \argst. Since it was
%% required in the input that \ibox{2} starts with something that will be realized as the subject, it
%% is clear that we are talking about personal passives. (\etag stands for an arbitrary list).

%% The lexical rule for the impersonal passive places the reverse restriction on the list \ibox{2}:
%% \eas
%% Lexical rule for the impersonal passive in Danish:\\
%% \ms{
%% head   & \ms[verb]{ da & \ibox{1}\\
%%                   }\\
%% arg-st & \ibox{1} $\oplus$ \ibox{2}  \\
%% } $\wedge$ \ibox{2} $\neq$ ( \sliste{ NP $\vee$ S } $\oplus$ \etag ) $\mapsto$
%% \ms{
%% arg-st & \sliste{ NP\sub{der} } $\oplus$ \ibox{2} \\
%% }
%% \zs
%% In (\mex{0}) it is required that \ibox{2} does not start with an NP or an S and hence the result
%% will be an impersonal passive with \ibox{2} either being the empty list or a list that contiains an
%% oblique argument as for instance a PP (see (\ref{ex-impersonal-passive-pp})).

%% The output of the lexical rule has an \argstl that contains all elements of \ibox{2} but in addition
%% an expletive \emph{der} NP, which will be mapped onto the subject valence feature.

%% Both lexical rules are underspecified for the type of their input. There are subtypes of the lexical
%% rules that are given in (\mex{-1}) and (\mex{0}): These lexical rules can either apply to fully
%% inflected verbs\NOTE{ finite forms and infinitives? } and add the \suffix{s} suffix or they apply to
%% stems and add the participle morphology.

%% The generalization over all passive lexical rules is the following constraint:
%% \ea
%% Constraint that holds for all passive lexical rules in Danish, English, and German (preliminary):\\
%% \ms{
%% head   & \ms[verb]{ da & \ibox{1}\\
%%                   }\\
%% arg-st & \ibox{1} $\oplus$ \ibox{2} \\
%% } $\mapsto$
%% \ms{
%% arg-st & \etag $\oplus$ \ibox{2} \\
%% }
%% \z
%% The individual rules differ as to the value of \etag. In German and English \etag is always the
%% empty list. This is also true of personal passives in Danish. Only for impersonal passives \etag is
%% a list that contains the expletive NP. We believe that the representation in (\mex{0}) captures the
%% phenomenon of passive rather elegantly: Passive is the suppression of the most prominent argument.


\subsection{Mapping to \comps\ in OV languages}

\frame{
\frametitle{Mapping to \comps\ in OV languages}

All elements of \argst\ are mapped to \comps\:

\begin{tabular}[t]{@{}l@{ }l@{~~}l@{ }l@{~~}l@{}}
  &                              & \textsc{arg-st} & \spr & \comps\\[2mm]
a.&getanzt:  & \liste{}        & \liste{} & \liste{}\\[2mm]
b.&gelesen:    & \liste{ NP[\type{str}] } & \liste{} & \liste{ NP[\type{str}] }\\[2mm]
c.&geschenkt: & \liste{ NP[\type{ldat}], NP[\type{str}] } & \liste{} & \liste{ NP[\type{ldat}], NP[\type{str}] }\\[2mm]
d.&geholfen:  & \liste{ NP[\type{ldat}] } & \liste{} & \liste{ NP[\type{ldat}] }\\
\end{tabular}


}


\frame%[shrink=10]{
	[allowframebreaks]{
\frametitle{All arguments of the participle can be prefixed}

%\savespace
%\smallexamples

Complements can be placed in front:

\eal
	\ex 
		\gll Den        Wählern Märchen erzählen sollte man besser nicht.\\
			the.\DAT{} voters fairy.tales tell should one.\NOM{} rather not\\
					\glt `It is better not to tell voters fairy tales.'
	
	\ex  
		\gll Den        Wählern erzählen sollte man solche Märchen     nicht.\\
			the.\DAT{} voters  tell     should one.\NOM{} such   fairy.tales not\\
				\glt `One should not tell voters fairy tales.'
				
	\ex  
		\gll Märchen     erzählen sollte man den        Wählern nicht.\\
			fairy.tales tell     should one.\NOM{} the.\DAT{} voters   not\\
				\glt `One should not tell voters fairy tales.'
				
\zl

\pause

However, the subject normally does not:

\eal
	\ex[*]{
		\gll Dieser Mann erzählen sollte den Wählern solche Geschichten besser nicht.\\
			this.\NOM{} man tell should the.\DAT{} voters such stories rather not\\
}

	\ex[*]{ 
		\gll Dieser      Mann den Wählern       erzählen sollte solche Geschichten besser nicht.\\
			this.\NOM{} man  the.\DAT{} voters tell     should such   stories     rather not\\
}

	\ex[*]{
		\gll Dieser      Mann solche Geschichten erzählen sollte den        Wählern besser nicht.\\
			this.\NOM{} man  such   stories     tell     should the.\DAT{} voters  rather not\\
}

\zl

\pause

Bei Partizipien können beide Objekte der Aktivform vorangestellt werden:
\eal
	\ex 
		\gll Diesen      Plan den        Wählern gegeben hat er nicht.\\
			this.\ACC{} plan the.\DAT{} voters  given   has he not\\
				\glt `He didn't give this plan to the voters.'
				
	\ex 
		\gll Dieser Plan den Wählern gegeben wurde damals nicht.\\
			this.\NOM{} plan the.\DAT{} voters given was back.than not\\
				\glt `This plan wasn't given to the voters back then.'

\zl

PVP see: \citew{Mueller96a}, \citew[Chapter~2.2.2]{Mueller2002b} or \citew[Chapter~15.2]{MuellerLehrbuch4}.

}


%\section{Variation and Generalizations}

\subsection{The analytical passive (auxiliary verb)}
\label{sec-auxiliary}



\frame{
\frametitle{The auxiliary verb}

\begin{itemize}
\item The passive auxiliary verb is similar for all the languages covered:

\ea
Passive auxiliary verb for Danish, German, English:
\ms{
arg-st \ibox{1} $\oplus$ \ibox{2} $\oplus$  \liste{ \ms{ vform & ppp\\
                                                                        da & \sliste{ XP$_{ref}$ }\\
                                                                                      spr   & \ibox{1}\\
                                                                                      comps & \ibox{2}\\
                                                                                    } } 
}
\z

\pause

\item \daw excludes unaccusative verbs and weather verbs

\pause

\item German forms verbal complex: arguments of the participle (\ibox{1} and \ibox{2}) are attracted by the passive auxiliary verb \citep{HN89a}. 

\pause
\item Verbal complex scheme allows unsaturated non-head daughter.

\pause
\item Also works for languages that do not form verbal complexes:\\
\ibox{2}  is then the empty list.

%% \item Hence, we have explained how
%% Danish and English embed a VP and German forms a verbal complex although the lexical item of the
%% auxiliary does not require a VP complement.

\end{itemize}

}

\subsection{The morphological passive}

\frame{
\frametitle{The morphological passive}


%% We assume that the same lexical rule that accounts for the participle forms can be used for the
%% morphological passives in Danish, modulo differences in the realizations of affixes of course. For
%% the morphological passive it is assumed that the \da of the input to the lexical rule has to contain
%% a referential XP. As was discussed in the previous section, this excludes morphological passives of
%% unaccusatives and weather verbs. 

\begin{itemize}
\item Lexicon rule also works for the morphological passive. A \suffix{s} is simply added.
\end{itemize}

}


%% \section{Agent Expressions}

%% We follow \citet[Chapter~7]{Hoehle78a} and \citet[Section~5]{Mueller2003e} and treat the \emph{by}
%% phrases as adjuncts.

\subsection{Perfect}

\frame{
\frametitle{Perfect}

\begin{itemize}
\item German: Only one participle for passive and perfect tense \citep{Haider86}. 
\pause
\item The designated argument is blocked, but is contained in the lexicon element
\pause
\item Perfect auxiliary verb unblocks it.

\eal
	\ex
		\gll dass der        Aufsatz [gelesen wurde] \\
			that the.\NOM{} paper   ~read    \AUX\\
			\glt `that the paper was read'
			
\ex
	\gll dass Kirby den Aufsatz [gelesen hat] \\
		that Kirby the.\ACC{} paper ~read \AUX\\
		\glt `that Kirby has read the paper'


\zl
\pause

Perfect auxiliary verb \emph{haben}:\\
\ms{
arg-st \ibox{1} $\oplus$ \ibox{2} $\oplus$ \ibox{3} $\oplus$  \liste{ \ms{ vform & ppp\\
                                                                        da & \ibox{1}\\
                                                                        spr   & \ibox{2}\\
                                                                        comps & \ibox{3}\\
                                                                       } } 
}


\end{itemize}

}

\frame{
\frametitle{Analysis as a complex predicate for Danish and English?}

\begin{itemize}
\item In an analysis with argument unblocking, \\
	one would have to assume structure in (\mex{1}a--b):
	
\eal
\ex He [has given] the book to Mary.
\ex The book [was given] to Mary.
\ex He has [given the book to Mary].
\ex The book was [given to Mary].
\zl

Otherwise we would know about the deblocked subject too late, because the participle would only require a PP -- as in (\mex{0}d).


\end{itemize}

}


%% \frame{
%% \frametitle{Expletives}

%% \begin{itemize}
%% \item Expletives needed for passive only:

%% \eal
%% \ex[]{
%% \gll  at   der        bliver arbejdet\\
%%       that {\sc expl} is     worked\\
%% }
%% \ex[*]{
%% \gll  at   Peter har arbejdet der\\
%%       that Peter has worked   {\sc expl}\\
%% }
%% \ex[*]{
%% \gll  at   der        har arbejdet Peter\\
%%       that {\sc expl} has worked   Peter\\
%% }
%% \zl

%% \end{itemize}

%% }



%% \frame{
%% \frametitle{A Solution that Almost Works}

%% \begin{itemize}
%% \item Complex Passive: There has to be a way to distinguish between participles that can be used in both perfect and
%%   passive:\\
%% {\sc voice} feature. 

%% \begin{itemize}
%% \item Value is \type{passive} for those participles that cannot be used in perfect constructions.

%% \pause
%% \item Value is underspecified for participles that can be used in both perfect and passive

%% \pause
%% \item Perfect requires {\sc voice} value to be \type{active}.
%% \end{itemize}

%% \pause
%% \item Expletives: Perfect attracts args from \argstl rather than \spr/\comps.
%% \begin{itemize}
%% \item Since expletives are not on \argst, they will not get into the way.
%% \end{itemize}
%% \end{itemize}


%% }


\frame[allowframebreaks]{
%\frametitle{But: (Partial) Fronting}
\frametitle{Problem: (Partial) Fronting}

%\smallexamples

\begin{itemize}
\item \citet{Meurers99b} has found a trick how to analyze the case assignment in (\mex{1}):

\nocite{Meurers2000b,MdK2001a}

\eal
	\ex 
		\gll Gelesen wurde der Aufsatz schon oft. \\
				read \AUX{} the paper.\NOM{} already often \\
			\glt `The paper has been read many times.'
		
	\ex 
		\gll Der Aufsatz gelesen wurde schon oft. \\
				the paper.\NOM{} read \AUX{} already often \\
			\glt `The paper has been read many times.'		

	\ex
		\gll Den Aufsatz gelesen hat er schon oft. \\
				the paper.\ACC{}  read \AUX{} he already often \\ 
			\glt `He has read the paper many times.'
\zl

%\pause
\framebreak

\item However, this does not work for Danish/English, because here we not only have case differences but also position differences:

\eal
\ex The book should have been given to Mary and\\
    {}[given to Mary] it was.
\ex He wanted to give the book to Mary and\\
    {}[given the book to Mary] he has.
\zl

If no sophisticated mechanisms for the underspecification of different mappings can be found,\\
we will probably have to assume two different participle forms.

\end{itemize}


}




\subsection{The remote passive}
\label{sec-remote-passive-phen}

\frame{
\frametitle{The remote passive}


\begin{itemize}
\item \citet[S.\,175--176]{Hoehle78a}: in certain contexts objects
of \emph{zu}-infinitives in the nominative case.

The following sentences are examples of the so-called long-distance passive:
\eal
	\ex
		\gll daß er auch von mir zu überreden versucht wurde\footnote{
        \citew*[S.\,212]{Oppenrieder91a}.} \\
        	that he.\NOM{} also by  me  to persuade  tried    \AUX \\ 
        \glt `that an attempt to persuade him was also made by me'

	\ex
		\gll weil    der Wagen oft zu reparieren versucht wurde \\
			because the.\NOM{} car   often to repair     tried    \AUX \\
		\glt `because many attempts were made to repair the car'
		
\zl
\end{itemize}
}

\frame[allowframebreaks]{
\frametitle{The remote passive}

\smallexamples

Accusative objects of embedded verbs can become nominative in the passive:

\eal
	\ex 
		\gll Dabei darf jedoch nicht vergessen werden, daß in der Bundesrepublik, wo \blauit{ein} \blauit{Mittelweg} \blauit{zu} \blauit{gehen} \blauit{versucht} \blauit{wird}, 
		die Situation der Neuen Musik allgemein und die Stellung der Komponistinnen im besonderen noch recht unbefriedigend ist.\footnote{
		Mannheimer Morgen, 26.09.1989, Feuilleton; Ist's gut, so unter sich zu bleiben?} \\
		 But should however not forgotten get that in the BRD where a middle.way to go tried gets the situation of.the new music generally and the position of.the composers in particular still quite unsatisfactory is \\
		 \glt `One should not forget that the situation of the New Music in general and the position of female composers 
		 in particular is rather unsatisfying in the Bundesrepublik, where one tries to follow a middle course.'
		 
\pause

\ex \gll Noch ist es nicht so lange her, da ertönten gerade aus dem Thurgau jeweils die lautesten Töne, 
    wenn im Wallis oder am Genfersee im Umfeld einer Schuldenpolitik mit den unglaublichsten Tricks 
    \blauit{der} \blauit{sportliche} \blauit{Abstieg} \blauit{zu} \blauit{verhindern} \blauit{versucht} \blauit{wurde}.\footnote{St.\ Galler Tagblatt, 09.02.1999, Ressort: TB-RSP; HCT und das Prinzip Hoffnung.}\\
	still is it not so long ago there sounded just out of.the Thurgau at.the.time the 
	loudest sounds when in.the Valais or at.the Lake.Geneva in.the sphere of.a debt.policy 
	with the most.unbelievable tricks the sporty relegation to prevent tried got \\
	\glt `It still is not too long ago that the loudest protests were heard in the Thurgau itself 
	when the most unbelievable tricks in the sphere of debt policies were applied to prevent relegation in the Valais or at Lake Geneva.' 
	
\pause

\ex 
	\gll Die Auf- und Absteigenden erzeugen ungewollt einen Ton,
        \blauit{der} bewusst nicht als lästig \blauit{zu} \blauit{eliminieren} \blauit{versucht} \blauit{wird}, 
    sondern zum Eigenklang des Hauses gehören soll, so wünschen es sich die Architekten.\footnote{
Züricher Tagesanzeiger, 01.11.1997, S.\,61.} \\
	the up and downclimbers create involuntarily a tone that consciously not as annoying 
	to eliminate tried gets but to.the own.sound of.the house belong should so wish it themselves the architects \\
	\glt `The people who go up and down produce a tune without intention which is not consciously sought to
	be eliminated but which, rather, belongs to the individual sound of the building, as the
	architects intended.'
\zl

}

\frame{
\frametitle{Examples with \word{beginnen}, \word{vergessen} and \word{wagen}}
\smallexamples
\smallframe

\citet{Wurmbrand2003a}:

\eal
%% \ex
%% \blau{dieser} wurde bereits zu bauen begonnen.\footnote{
%%         \url{http://www.hollabrunn.noe.gv.at/mariathal/ortsvorsteher.html}, 28.07.2003.
%% }
%\pause
\ex
	\gll \blau{der} \blau{zweite} \blau{Entwurf} wurde zu bauen begonnen,\footnote{\url{http://www.waclawek.com/projekte/john/johnlang.html}, 28.07.2003.} \\
		the.\NOM{} second        plan           \AUX\ to build started \\
		\glt `It was begun to build the second plan.' 
\zl

\pause

\eal
	\ex 
		\gll Anordnungen, die zu stornieren vergessen \blau{wurden}\footnote{
        \url{http://www.rlp-irma.de/Dateien/Jahresabschluss2002.pdf}, 28.07.2003.} \\
        	orders       that to cancel forgotten were\\
        	\glt `orders that were forgotten to cancel'
        		
\pause

	\ex 
		\gll Aufträge [\ldots], die zu drucken vergessen worden \blau{sind}\footnote{
        \url{http://www.iitslips.de/news.html}, 28.07.2003.} \\
        	 orders   {}        that to print  forgot    were   are\\
        	 \glt `orders that somebody forgot to print'
        	
\zl

}

\frame{
	\frametitle{Examples with \word{beginnen}, \word{vergessen} and \word{wagen}}
\smallexamples
\smallframe
	
\eal
%\ex Ist plötzlich übervoll von Emotionen und längst begrabenen Träumen, die nicht zu leben gewagt wurden\footnote{
% nicht auffindbar
	 	\ex[] {NUR Leere, oder doch noch Hoffnung, weil aus Nichts wieder Gefühle entstehen,}
		\gll die so vorher nicht mal zu träumen gewagt \blau{wurden}?\footnote{
        \url{http://www.ultimaquest.de/weisheiten_kapitel1.htm}, 28.07.2003.} \\
        	 that this.way before not even to dream dared were\\
        	 \glt `that were not even dared to be dreamed of in this way before'
        	 
\pause

	\ex[] {Dem Voodoozauber einer Verwünschung oder die gefaßte Entscheidung} {zu einer Trennung,}
    \gll die bis dato noch nicht auszusprechen gewagt \blau{wurden}\footnote{
        \url{http://www.wedding-no9.de/adventskalender/advent23_shawn_colvin.html}, 28.07.2003.} \\
        	  which until now yet not express dared were\\
        	  \glt `which until now have not been dared to express'
        	  
\zl
% Kasus bei PVP wie Haiders entziffern: Am leichtesten zu erklären fiel den 
% Experten dabei gestern der Kursverlust der Telekom, zu deren Schuldenproblem 
% eine neue Hiobsbotschaft kam.  (taz. 8./9. 9. 01 S. 9.)
%
}


\frame{
\frametitle{Distant passive and verbal complex formation (I)}

\begin{itemize}
\item Object of a verb embedded under a passive participle,\\
becomes the subject of the sentence:
\eal
	\ex 
		\gll weil er den Wagen oft zu reparieren versucht hat \\
				because he.\NOM{} the.\ACC{} car often to repair tried has \\
			\glt `because he made many attempts to repair the car'
				
	\ex 
		\gll weil der Wagen oft \blau<2>{zu} \blau<2>{reparieren} \blau<2>{versucht} \blau<2>{wurde} \\
			     because the.\NOM{} car   often to repair     tried    was \\
			\glt `because many attempts were made to repair the car'
\zl

\end{itemize}
}

\frame{
	\frametitle{Distant passive and verbal complex formation (II)}
	
\begin{itemize}
\item Distant passive only possible with verbal complexes:
\eal
	\ex[]{
		\gll weil    oft   versucht wurde, den        Wagen zu reparieren.\\
			because often tried    \AUX{} the.\ACC{} car   to repair\\
			\glt `because many attempts were made to repair the car.'
}

\ex[*]{
		\gll weil    oft   versucht wurde, der        Wagen zu reparieren.\\
		because often tried    \AUX{} the.\NOM{} car   to repair\\
}

\pause

	\ex[]{
		\gll \blau{Den} \blau{Wagen} \blau{zu} \blau{reparieren} wurde oft versucht\\
			the.\ACC{} car   to repair     \AUX{} often tried\\
}

	\ex[*]{
		\gll Der Wagen zu reparieren wurde oft versucht \\
			the.\NOM{} car   to repair     \AUX{} often tried\\
}

\zl

\end{itemize}

}

\frame{
\frametitle{Fernpassiv und Verbalkomplexbildung (III)}

\smallframe

\begin{itemize}

\item Explanation: Distant passive = passivization of the predicate complex

\eal
	\gll weil    der Wagen     oft   [[zu reparieren versucht] wurde] \\
		 because the.\NOM{} car   often {~~to} repair     tried    was\\
\zl
\pause
\item There are no verbal complexes in (\mex{1}a,c).

\eal
	\ex[]{
		\gll weil oft versucht wurde, \blau<2>{den}  \blau<2>{Wagen}  \blau<2>{zu}  \blau<2>{reparieren} \\
			because often tried    \AUX{} the.\ACC{} car   to repair\\
}

	\ex[*]{
		\gll weil    oft   versucht wurde, der        Wagen zu reparieren.\\
		because often tried    \AUX{} the.\NOM{} car   to repair\\
}

	\ex[]{
			\gll \blau<2>{Den} \blau<2>{Wagen} \blau<2>{zu} \blau<2>{reparieren} wurde oft versucht \\
					     the.\ACC{} car   to repair     \AUX{} often tried\\
}

	\ex[*]{
		\gll Der Wagen zu reparieren wurde oft versucht \\
			the.\NOM{} car   to repair     \AUX{} often tried\\
}

\zl

Object of \emph{to repair} is part of the VP $\to$ gets accusative

\pause

The passives in (\mex{0}a,c) are impersonal passives.

\end{itemize}

}

\frame[shrink=15]{
\frametitle{Analysis of the remote passive}

\centerline{\scalebox{.8}{
\begin{forest}
sm edges
[V\feattab{
              \vform \type{fin},\\
              \comps \highlight{\ibox{1}}<4> } 
        [{\ibox{2} V\feattab{
              \vform \type{ppp},\\
              \da    \sliste{ NP[\str]$_i$ },\\
              \comps \highlight{\ibox{1}}<4> }} 
           [{\highlight{\ibox{3} V}<1>\feattab{
              \vform \type{inf},\\
              \subj  \highlight{\sliste{ NP[\str]$_i$ }}<2>, \\ 
              \comps \highlight{\ibox{1} \sliste{ NP[\str]$_j$ }}<2> }} [zu reparieren;to repair] ]
           [V\feattab{
              \phon \phonliste{ versucht }\\
              \vform \type{ppp},\\
              \da    \ibox{4},\\
              \comps \highlight{\ibox{1} $\oplus$ \sliste{ \ibox{3} }}<3>,\\
              \argst \highlight{\ibox{1} $\oplus$ \sliste{ \ibox{3} }}<3> } 
              [V\feattab{
%              \phon \phonliste{ versuch }\\
              \vform \type{ppp},\\
              \da    \highlight{\ibox{4} \sliste{ NP[\str]$_i$ }}<3>,\\
              \argst \highlight{\highlight{\ibox{4}}<3> $\oplus$ \ibox{1}}<2> $\oplus$ \sliste{ \highlight{\ibox{3}}<1> } } [versuch-;try] ] ] ]
        [V\feattab{
              \vform \type{fin},\\
              \comps \ibox{1} $\oplus$ \sliste{
                \ibox{2} }} [wurde;\textsc{aux}] ] ]
\end{forest}}}


\begin{itemize}
\item \emph{versuchen} embeds infinitive with \emph{zu}.
\pause
\item \emph{versuchen} attracts arguments from \emph{reparieren}: \argstw \sliste{ NP[\str]$_i$, NP[\str]$_j$, V[\type{inf}] }
\pause
\item Passive LR suppresses first argument: \emph{versucht} hat
\argstw \sliste{ NP[\str]$_j$, V[\type{inf}] } 
\pause
\item \emph{zu reparieren versucht}: \argstw \sliste{ NP[\str]$_j$ } and
\emph{zu reparieren versucht wurde} also
\end{itemize}


}


\frame{
\frametitle{Distant passive with object control verbs}

\smallframe 
\smallexamples

\begin{itemize}
\item Distant passive also possible with object control verbs:

\eal
	\ex
		\gll Keine Zeitung         wird ihr       zu lesen erlaubt.\footnote{Stefan Zweig. \emph{Marie Antoinette}. Leipzig: Insel-Verlag. 1932, S.\,515, 
        zitiert nach \citew[S.\,309]{Bech55a}. Siehe \citet[S.\,13]{Askedal88}.} \\
        	no    newspaper.\NOM{} \AUX{} her.\DAT{} to read  allowed\\
        	\glt `She is not allowed to read any newspapers.'

	\ex%\iw{auskosten}
		\gll Der Erfolg        wurde uns      nicht auszukosten erlaubt.\footnote{\citew[S.\,110]{Haider86c}.} \\
        	the success.\NOM{} \AUX{} us.\DAT{} not   to.enjoy    permitted\\
        	\glt `We were not permitted to enjoy our success.'

\zl

\pause
\item Passive of the construction without verbal complex is an impersonal passive:

\eal
	\gll Uns       wurde  erlaubt, den        Erfolg  auszukosten.\\
	us.\DAT{} \AUX{} allowed  the.\ACC{} success to.enjoy\\
\zl

\pause

\item Generalization: In passive constructions in which a verbal complex is embedded under the passive auxiliary verb, the subject is suppressed and of the remaining arguments
the first argument with structural case becomes the subject and receives nominative case.%
\end{itemize}
}


\frame{
\frametitle{Distant passive with object control verbs}

\eal
	\gll Keine Zeitung         wird ihr       zu lesen erlaubt.\footnote{Stefan Zweig. \emph{Marie Antoinette}. Leipzig: Insel-Verlag. 1932, S.\,515, zitiert nach \citew[S.\,309]{Bech55a}. Siehe \citet[S.\,13]{Askedal88}.} \\
		no    newspaper.\NOM{} \AUX{} her.\DAT{} to read  allowed\\
		\glt `She is not allowed to read any newspapers.'%
\zl

\oneline{%
\begin{tabular}{@{}l@{ }l@{}}
\emph{erlauben}: & \sliste{ NP[\str]$_i$, NP[\ldat]$_j$ } $\oplus$ \ibox{1} $\oplus$ \sliste{ V[\comps \ibox{1}] }\\[2mm]

\emph{zu lesen erlauben}: & \sliste{ NP[\str]$_i$, NP[\ldat]$_j$, NP[\str]$_k$, V[\comps \sliste{ NP[\str]$_k$ }] }\\[2mm]

\emph{zu lesen erlaubt wird}: & \sliste{ NP[\ldat]$_j$, NP[\str]$_k$, V[\comps \sliste{ NP[\str]$_k$ } ] }\\
\end{tabular}
}

\pause
\medskip

First NP with structural case is subject.

}


%% \frame{
%% \frametitle{Komplexes Passiv}

%% \begin{itemize}
%% \item Komplexes Passiv:

%% \ea
%% \gll at Bilen           blev forsøgt repareret\\
%%      dass Auto.{\sc def} wurde versucht repariert\\
%% \glt `dass versucht wurde, das Auto zu reparieren'
%% \z


%% \pause
%% \item Anhebung nur im Passiv.


%% \pause
%% \item \emph{forsøgt} `versuchen' nimmt im Aktiv sonst kein Partizip:
%% \eal
%% \ex[]{
%% \gll at   Peter har  forsøgt \blaubf{at} \blaubf{reparere} bilen\\
%%      dass Peter hat versucht zu reparieren Auto.{\sc def}\\
%% \glt `dass Peter versucht hat, das Auto zu reparieren'
%% }
%% \ex[*]{
%% \gll at   Peter har  forsøgt \blaubf{repareret} bilen\\
%%      dass Peter hat  versucht  repariert Auto.{\sc def}\\
%% %\glt `that an attempt was made to repair the car'
%% }
%% \zl

%% %% \item Conclusion: We need special lexical items for passive participles.

%% %% \item analysis of the German passive and perfect can be maintained,\\
%% %% compatible with a more general analysis of the passive

%% \end{itemize}

%% }


\subsection{Summary}


\frame{
\frametitle{Summary}



\begin{itemize}
\item LRs for morphological and analytical passives
\pause
\item The first element of the \argstl is suppressed.
\pause
\item \emph{promote} promotes an NP with a structural case.

\pause
\item Languages differ with regard to the cases and the lexical/structural distinction.
\pause
\item Expletive is used for \argst mapping in Danish.

\pause
\item SVO languages require different lexicon elements for perfect/passive participles, but for
German, analysis works with a participle form.


\end{itemize}

}



\subsection*{Exercises}

\frame{
\frametitle{Exercises}

\begin{enumerate}
\item Which of the NPs in the following sentences have structural and which have lexical cases?
\eal
\ex Der Junge lacht.  
\ex Mich friert.    
\ex Er zerstört das Auto.
\ex Das dauert ein ganzes Jahr.
\ex Er hat nur einen Tag dafür gebraucht.
\ex Er denkt an den morgigen Tag.
\zl

\end{enumerate}

%\pause\pause\pause

}





%      <!-- Local IspellDict: de_DE -->
