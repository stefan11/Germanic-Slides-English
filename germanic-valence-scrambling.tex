%% -*- coding:utf-8 -*-
\author{Stefan Müller}

\subtitle{Valence, argument order and adjunct placement}

\section{Valence, argument order and adjunct placement}


\huberlintitlepage[22pt]


\outline{

\begin{itemize}
%\item {Überblick über die germanischen Sprachen}
\item Phenomena
\item Phrase structure grammars and \xbar theory
\item \alert{Valence, order of arguments and adjuncts}
\item Verb clusters in SOV langauges
\item Verbposition: Verb first and verb second order
\item Passive
%\item embedded sentences
\end{itemize}

}



\frame{
\frametitle{Literature}



Please read \citew[Chapter~4]{MuellerGermanic}.

\begin{refsection}

\nocite{MuellerGermanic}

\printbibliography[heading=none,notkeyword=this]

\end{refsection}

\pause

The theory that is used in the following is Head-Driven Phrase Structure Grammar in a reduced version.

For HPSG, see \citew{ps,ps2,Sag97a,MuellerLehrbuch,HPSGHandbook}.



}

\subsection{Valence}



\frame{
\frametitle{Presence of certain constituents}

The sequences in (\mex{1}) are not well-formed:
\eal
\ex[*]{
that the dolphin devours 
}
\ex[*]{
that of the dolphin the child the ball him his the child gives
}
\zl
Something is missing in (\mex{0}a), there is too much in (\mex{0}b).


}

\frame{
\frametitle{Valenz in der Chemie und in der Linguistik}


\citet{Tesniere59a-Eng} adapts the concept of valence from chemistry:


\vfill

\centerline{
\begin{forest}
[O
  [H] 
  [H] ]
\end{forest}
\hspace{5em}
\begin{forest}
[kennen
 [Aicke]
 [Conny] ]
\end{forest}
}

\vfill

}

\frame{
\frametitle{Arguments}

NPs in (\mex{1}) are arguments of the respective verbs:
\eal
\ex[]{
\gll
\dass der        Delphin den        Menschen erwartet\\
     \that{}  the.\NOM{} dolphin den.\ACC{} human   expects\\
\glt `that the dolphin expects the human'
}
\ex[]{
\gll 
{}[dass] der        Delphin dem        Kind  den        Ball gibt\\
     \that{}  the.\NOM{} dolphin the.\DAT{} child the.\ACC{} ball gives\\
\glt `that the dolphin gives the child the ball'
}
\zl

Syntactic arguments usually fill a semantic role (\zb giver, agent, actor,
\ldots).\nocite{Dowty91a,VanValin99a-u}

\citet[Chapter~48]{Tesniere2015a-u}: 

drama scene: What is needed to act out a giving event?

\begin{itemize}
\item a giver
\item something given
\item a givee (recipient)
\end{itemize}

}

\frame{
\frametitle{Adjucts}

\begin{itemize}
\item There can be adjuncts in addition to arguments:

\eal
\ex
\gll
\dass{} der        Delphin dem        Kind  \alert{schnell} den        Ball gibt\\
\that{} the.\NOM{} dolphin the.\DAT{} child quickly the.\ACC{} ball gives\\
\glt `that the dolphin gives the child the ball quickly'
\ex that the dolphin gives the child the ball \alert{quickly}
\zl

\pause

\item Adjuncts convey additional information but do not fill a semantic role.

\end{itemize}

}

\frame{
\frametitle{Optional arguments}

Almost all arguments can be omitted, provided context provides enough info.

\eal
\ex 
\gll Sie gibt Geld.\\
     she gives money\\
\glt `She gives money.'
\ex 
\gll Sie gibt den Armen.\\
     she gives the poor\\
\glt `She gives to the poor.'
\ex\label{ex-sie-gibt} 
\gll  Sie gibt.\\
     she gives\\
\ex 
\gll Gib!\\
     give\\
\zl

If we are playing the card game skat, it is clear who gives what to whom.

}

\frame{
\frametitle{Obligatory arguments (rare)}

There are some verbs which really have obligatory arguments:
\eal
\ex 
\gll verschlingen\\
     devour\\\german
\ex 
\gll erwarten\\
     await\\
\zl

Arguments may be optional, arguments always are. 


}

\frame{
\frametitle{Chemistry and optional elements}

The analogy with chemical bonds is helpful, but optional arguments remain confusing:
\eal
\ex Kirby helps Sandy.
\ex Kirby helps.
\zl

\hfill
\begin{forest}
[O
  [H] 
  [H] ]
\end{forest}
\hspace{5em}
\begin{forest}
[helps
 [Kirby]
 [Sandy] ]
\end{forest}
\hfill\mbox{}

Solution: syntactic and semantic valency.

Theater analogy helps us to find semantic arguments,\\
chemistry analogy helps for syntactic arguments.

One has to assume a special, one place verb for (\mex{0}b).

}

\frame{
\frametitle{Shopping instead of drama}

\begin{itemize}
\item The analogy to shopping is better:\\
We want to prepare pasta with tofu and tomato sauce.

For tomato sauce we need onions.

\item All ingredients are added to a shopping list.

\item The shop does not have tofu.

\item No problem, we can do pasta without tofu: tofu is optional.

\item There are noodles. 10.000 types of pasta.

\item Tomatos and onions. Done.

\item Ah. Gummy bears. What? They were not on the list?

Gummy bears are adjuncts!

\end{itemize}

}

\frame{
\frametitle{Pasta, tomatos, onions and syntax}

\begin{itemize}
\item There are several ways to make sure that all that is needed is present.

\item Phrase structures as in GPSG 
\ea
\label{ditrans-schema-two}
\begin{tabular}[t]{@{}l@{ }l@{ }l}
S  & $\to$ & NP[\type{nom}] NP[\type{dat}] NP[\type{acc}] V[\type{ditransitive}]\\
\end{tabular}
\z

\pause
\item These were dropped in favour of lexical approaches\\
  \parencites{Jacobson87b}[Section~5.5]{MuellerGT-Eng1}{MWArgSt}
\pause
\item reasons: 
\begin{itemize}
\item Partial VP Fronting \citep{Nerbonne86a,Johnson86a} 
\pause
\item Interactions with morphologie \citep[Section~5.5.1]{MuellerGT-Eng1}
\end{itemize}
\pause
\item Come back of phrasal approaches in Construction Grammar \citep{Goldberg95a}. These do not work.

\parencites{Mueller2006d,MuellerPersian,MuellerUnifying,MWArgSt,MWArgStReply,MuellerFCG,MuellerLFGphrasal,MuellerPotentialStructure,MuellerGT-Eng4,MuellerCxG}

\end{itemize}

}

\frame{
\frametitle{Valenzlisten}

\begin{itemize}
\item Arguments are represented in lists:
\ea
\label{valence-specifications-German}
\begin{tabular}[t]{@{}l@{~}l@{~}l}
a. & \emph{schläft} `sleeps':         & \sliste{ NP[\type{nom}] }\\
b. & \emph{kennt} `knows':           & \sliste{ NP[\type{nom}], NP[\type{acc}] }\\
%b. & \emph{unterstützt}:  & \sliste{ NP[\type{nom}], NP[\type{acc}] }\\
c. & \emph{hilft} `helps':           & \sliste{ NP[\type{nom}], NP[\type{dat}] }\\
d. & \emph{gibt} `gives':            & \sliste{ NP[\type{nom}], NP[\type{dat}], NP[\type{acc}] }\\
e. & \emph{wartet} `waits':          & \sliste{ NP[\type{nom}], PP[\type{auf}] }\\
\end{tabular}
\z

NP[\type{nom}] stands for something like \emph{Nudeln}.\\
There are many ways to satisfy these valence requirements:\\
\emph{sie} `she', \emph{das Kind} `the child', \emph{der lachende Delphin} `the laughing dolphin', \ldots

\pause
\item Elements of lists are ordered in a certain order. This order corresponds to the order in
  English and other languages with fixed order and to the unmarked order in German \citep{Hoehle82a}. 


\end{itemize}



}


\frame{
\frametitle{Valence determining structure}

\vfill
\centerfit{
\begin{forest}
sm edges
[{S \eliste},visible on=6-
  [{NP[\type{nom}]},visible on=5- 
    [niemand,visible on=5-] ]
  [{V$'$\sliste{ \blau<4>{NP[\type{nom}]} } },visible on=3-
    [\blau<3>{NP[\type{acc}]}, visible on=2-
      [ihn,visible on=2-] ]
    [{V \sliste{ \blau<4>{NP[\type{nom}]}, \blau<3>{NP[\type{acc}]}}} [kennt]] ] ]
\end{forest}}

\vfill
\begin{itemize}
\item Valence requirements are represented in a list.
\pause
\item One element of the list is combined with a head.\\
\pause\pause
      List with remaining elements is passed up.
\pause\pause\pause
\item Like shopping with an app:\\
      One element after the other is removed from the list.
%\pause
%\item Wichtig: Die Theorie sagt nichts darüber aus welcher
\end{itemize}

}

\frame{
\frametitle{Constraint-based theories and psycholinguistics}

\centerfit{
\begin{forest}
sm edges
[{S \eliste},visible on=4-
%  [{[dass]},no edge,visible on=3-]
  [{NP[\type{nom}]},visible on=3- 
    [niemand,visible on=3-] ]
  [{V$'$\sliste{ NP[\type{nom}] } },visible on=4-
    [{NP[\type{acc}]},visible on=5- 
      [ihn,visible on=5-] ]
    [{V \sliste{ NP[\type{nom}], NP[\type{acc}] }},visible on=6- [kennt,visible on=7-]] ] ]
\end{forest}}

\begin{itemize}
\item I explain things bottom up.
\pause
\item But this is not required by the theory.

This is important from a psycholinguistic point of view, since processing is incremental.
\parencites{Marslen-Wilson75a,TSKES96a,SW2011a,Wasow2021a}
\pause

\end{itemize}




}


\frame{
\frametitle{Optional Arguments?}

\begin{itemize}
\item There are several ways to deal with optional arguments.
\item The most obviousDie offensichtlichste: Wir haben alternative Lexikoneinträge.
% \ea
% \begin{tabular}[t]{@{}l@{~}l@{~}l}
% a. & \emph{gibt} `gives':            & \sliste{ NP[\type{nom}] }\\
% b. & \emph{wartet} `waits':          & \sliste{ NP[\type{nom}] }\\
% \end{tabular}
% \z
\ea
\begin{tabular}[t]{@{}l@{~}l@{~}l}
a. & \emph{gibt}:            & \sliste{ NP[\type{nom}] }\\
b. & \emph{wartet}:          & \sliste{ NP[\type{nom}] }\\
\end{tabular}
\z
\end{itemize}



}


\subsection{Scrambling}

\frame{
\frametitle{Scrambling: Konstituentenstellung im Deutschen}

\vfill
\centerfit{
\begin{forest}
sm edges
[{S \eliste}
   [{NP[\type{acc}]} [ihn] ]
   [{V$'$\sliste{ NP[\type{acc}] } }
      [{NP[\type{nom}]} [niemand] ]
      [{V \sliste{ NP[\type{nom}], NP[\type{acc}]}} [kennt] ] ] ]
\end{forest}}
\vfill
\pause
\begin{itemize}
\item Ein beliebiges Element der Liste kann mit Kopf kombiniert werden.\\
      $\to$ auch Abfolge Acc $<$ Nom analysierbar.\\
      Liste mit restlichen Elementen wird nach oben gegeben.
\end{itemize}

}



\subsection{SVO: Dänisch/Englisch}


\frame{
\frametitle{VP in SVO-Sprachen}

\begin{itemize}
\item Verben und Objekte bilden eine Wortgruppe:
\eal
\ex John promised to read the book and [\alert{read the book}], he will.
\ex He will [\alert{read the book}].
\ex Kim [[\alert{sold the car}] and [\alert{bought a bicycle}]]. 
\ex He often [\alert{reads the book}].
%\ex He [reads the book] often.
\ex \ldots{} [often [\alert{read the book}] slowly], he will.
\zl
\pause
\item Gut zu erfassen mit zwei Valenzlisten:\\
      eine für Komplemente (\comps für \textsc{complements}) und eine für das Subjekt (\textsc{spr} für \textsc{specifier}).
\end{itemize}


}

\frame{
\frametitle{Dänisch, Englisch, \ldots}

~\vfill

\centerfit{\begin{forest}
sm edges
[{V[\spr \eliste, \comps \eliste]}, name=S
   [{NP[\type{nom}]} [nobody] ]
   [V\feattab{
      \spr \sliste{ NP[\type{nom}] }, \comps \sliste{} }, name=VP
     [V\feattab{
         \spr \sliste{ NP[\type{nom}] },\\
         \comps \sliste{ NP[\type{acc}]}} [knows] ]
        [{NP[\type{acc}]} [him] ] ] ]
\node [right=4cm] at (S)
    {
        = S
    };
\node [right=4cm] at (VP)
    {
        = VP
    };
\end{forest}}


\vfill

\begin{itemize}
\item Englisch ist eine SVO-Sprache:\\
      Komplemente rechts des Verbs, Subjekt links
%\item Komplemente können nicht einfach umgestellt werden.
\item Komplemente bilden mit dem Verb zusammen eine Phrase (VP = \comps \sliste{}).

      Diese wird mit dem Subjekt kombiniert.
\end{itemize}

\vfill

}


\frame{
\frametitle{Kein Scrambling}

\begin{itemize}
\item Dänisch, Englisch:\\
      Elemente aus Valenzliste müssen von links nach rechts abgebunden werden.

\pause
\item Deutsch, Niederländisch:\\
      Elemente können in beliebiger Reihenfolge abgebunden werden.

%\pause
%\item 
\end{itemize}

}


\frame{
\frametitle{SVO: Ditransitive Verben: Abbindung von \comps von links}


\centerline{
\begin{forest}
sm edges
[{V[\spr \eliste, \comps \eliste]}, visible on=6-
   [{NP[\type{nom}]}, visible on=5- 
     [Kim, visible on=5-] ]
   [V\feattab{
      \spr \sliste{ NP[\type{nom}] }, \comps \sliste{}}, visible on=4-
     [V\feattab{
         \spr \sliste{ NP[\type{nom}] },\\
         \comps \sliste{ PP[\type{to}] }}, visible on=2-
       [V\feattab{
           \spr \sliste{ NP[\type{nom}] },\\
           \comps \sliste{ NP[\type{acc}], PP[\type{to}] }} [gave] ]
         [{NP[\type{acc}]} 
            [a book, roof] ] ]
       [{PP[\type{to}]}, visible on=3- 
         [to Sandy, roof, visible on=3-] ] ] ]
\end{forest}}

Bei kopffinalen Sprachen ohne Scrambling Abbindung von rechts.


}

%% \frame{
%% \frametitle{Regeln: Englisch und Deutsch}

%% \begin{itemize}
%% \item Englisch:
%% \ea
%% H[\comps \ibox{B}] $\to$ H[\comps \ibox{B} $\oplus$ \sliste{ \ibox{1} } ] ~~~\ibox{1}
%% \z

%% `$\oplus$' zerlegt Liste in zwei Teillisten.

%% \pause

%% \item Deutsch:
%% \ea
%% H[\comps \ibox{A} $\oplus$ \ibox{B}] $\to$ H[\comps \ibox{A} $\oplus$ \sliste{ \ibox{1} } $\oplus$ \ibox{B} ] ~~~\ibox{1}
%% \z

%% \pause

%% \item Das Englische unterscheidet sich vom Deutschen dadurch,\\
%%       dass \ibox{A} die leere Liste ist.

%% $\to$ Englisch ist restriktiver.

%% \end{itemize}


%% }


\frame{
\frametitle{Deutsch}



\centerfit{\begin{forest}
sm edges
[{V[\spr \eliste, \comps \eliste]}, name=S
        [{NP[\type{nom}]} [niemand] ]
        [{V\feattab{
              \spr \sliste{ }, \comps \sliste{ NP[\type{nom}] } }}, name = Vs
          [{NP[\type{acc}]} [ihn] ] 
          [V\feattab{
              \spr \sliste{  },\\
              \comps \sliste{ NP[\type{nom}], NP[\type{acc}]}} [kennt] ]
] ]
\node [right=4cm] at (S)
    {
        = S
    };
\node [right=4cm] at (Vs)
    {
        = V$'$
    };
\end{forest}}

Das Subjekt ist bei finiten Verben in der \compsl \citep{Pollard90a,Kiss95a}.

Abkürzungen: \begin{tabular}[t]{@{}l@{ = }l}
             S  & [\spr \eliste, \comps \eliste]\\
             VP & [\spr \sliste{ NP[\type{nom}] }, \comps \sliste{}]\\
             V$'$ & alle anderen V-Projektionen (außer Verbalkomplexen)\\
             \end{tabular}

}

\subsection{Immediate Dominance Schemata}

\frame{
\frametitle{Immediate Dominance Schemata}

\begin{itemize}
\item In vielen theoretischen Arbeiten werden einfach Baumstrukturen gezeigt.
\item Diese sind aber nicht einfach da.\\
      Es gibt Regeln oder Schemata, die sie lizenzieren.
\item \ZB \xbar-Schemata oder Regel für Phrasenstrukturen oder Dependenzstrukturen.
\item Regeln für HPSG:
\ea\label{schema-head-spr-and-head-comps-preliminary}
Specifier-Head Schema and Head-Complement Schema (preliminary)
\begin{tabular}[t]{@{}l@{ }l@{}}
H[\spr \ibox{1}]   & $\to$ H[\spr \ibox{1} $\oplus$ \sliste{ \ibox{2} }, \comps \eliste]\hspace{1em}\ibox{2}  \\
H[\comps \ibox{1}] & $\to$ H[\comps \sliste{ \ibox{2} } $\oplus$ \ibox{1}]\hspace{1em}\ibox{2} \\
\end{tabular}
\z
\pause
\item H steht für \emph{head}. Die entsprechende Phrase ist der oder enthält den Kopf.

\end{itemize}

}


\subsubsection{Spezifikator-Kopf-Strukturen}

\frame{
\frametitle{Schemata als Teilbäumchen}

\begin{itemize}
\item Schemata lassen sich auch als Teilbäumchen darstellen.
\ea\label{schema-head-spr-and-head-comps-preliminary}
Specifier-Head Schema (preliminary)\\
H[\spr \ibox{1}] $\to$ H[\spr \ibox{1} $\oplus$ \sliste{ \ibox{2} }, \comps \eliste]\hspace{1em}\ibox{2}
\z

\item Specifier-Head Schema in Baum-Notation:
\vfill
%\centerline{
\begin{forest}
[H\feattab{\spr \ibox{1}}%,\\
           %\comps \eliste}
  [\ibox{2}]
  [H\feattab{\spr \ibox{1} $\oplus$ \sliste{ \ibox{2} },\\
             \comps \eliste}
  ]]
\end{forest}
%}
\vfill

\end{itemize}

}

\frame{
\frametitle{\texttt{append} ($\oplus$)}

\begin{itemize}
\item \texttt{append} verknüpft zwei Listen: \sliste{ \normalfont a } $\oplus$ \sliste{ \normalfont b } =
\sliste{ \normalfont a, b }. 

\pause
\item Die Verknüpfung einer Liste L mit der leeren Liste ergibt die Liste L.

\pause
\item Beispiel: \texttt{append} teilt die Liste \sliste{ NP[\type{nom}], NP[\type{dat}], NP[\type{acc}
    ] } wie folgt:
\eal
\ex \oneline{\eliste{} $\oplus$ \sliste{ NP[\type{nom}], NP[\type{dat}], NP[\type{acc}] } = \sliste{ NP[\type{nom}], NP[\type{dat}], NP[\type{acc}] }}
\pause
\ex \oneline{\sliste{ NP[\type{nom}] } $\oplus$ \sliste{ NP[\type{dat}], NP[\type{acc}] } = \sliste{ NP[\type{nom}], NP[\type{dat}], NP[\type{acc}] }}
\pause
\ex \oneline{\sliste{ NP[\type{nom}], NP[\type{dat}] } $\oplus$ \sliste{ NP[\type{acc}] } = \sliste{ NP[\type{nom}], NP[\type{dat}], NP[\type{acc}] }}
\pause
\ex \oneline{\sliste{ NP[\type{nom}], NP[\type{dat}], NP[\type{acc}] } $\oplus$ \eliste{} = \sliste{ NP[\type{nom}], NP[\type{dat}], NP[\type{acc}] }}
\zl

\end{itemize}

}


\frame{
\frametitle{Zerlegung der Valenzliste}

\begin{itemize}
\item Specifier-Head Schema in Baum-Notation:\\
\vfill
\begin{forest}
[H\feattab{\spr \ibox{1}}%,\\
           %\comps \eliste}
  [\ibox{2}]
  [H\feattab{\spr \ibox{1} $\oplus$ \sliste{ \ibox{2} },\\
             \comps \eliste}
  ]]
\end{forest}
\vfill

\item Schema zerlegt Liste in beliebige Liste \iboxb{1} und eine einelementige Liste (\sliste{ \ibox{2} }).

\item Für unser Beispiel wäre das (\mex{0}c):
\ea
\oneline{\sliste{ NP[\type{nom}], NP[\type{dat}] } $\oplus$ \sliste{ NP[\type{acc}] } = \sliste{ NP[\type{nom}], NP[\type{dat}], NP[\type{acc}] }}
\z

\ibox{1} = \sliste{ NP[\type{nom}], NP[\type{dat}] } und \ibox{2} = NP[\type{acc}]

\end{itemize}

}

\frame{
\frametitle{Listenzerlegung in Spezifikator-Kopf-Strukturen}

\begin{itemize}
\item \sprl hat für gewöhnlich nur ein Element: NP[\type{nom}]/Subjekt für Verben in SVO-Sprachen oder
  Determiner, wenn der Kopf ein Nomen ist.

\vfill
\begin{forest}
[H\feattab{\spr \ibox{1}}%,\\
           %\comps \eliste}
  [\ibox{2}]
  [H\feattab{\spr \ibox{1} $\oplus$ \sliste{ \ibox{2} },\\
             \comps \eliste}
  ]]
\end{forest}

\vfill

\pause
\item \ibox{1} ist leere Liste, \ibox{2} ist NP[\type{nom}] bzw.\ Det. 


\eal
\ex \eliste{} $\oplus$ \sliste{ NP[\type{nom}] } = \sliste{ NP[\type{nom}] }
\ex \eliste{} $\oplus$ \sliste{ Det } = \sliste{ Det }
\zl

\end{itemize}

}

\exewidth{(235)}

\frame{
\frametitle{Kopf hat Wünsche. Schema regelt Passgenauigkeit}

\begin{itemize}
\item Beschreibung der Tochter muss mit der einzusetzenden Tochter kompatibel sein.

\vfill
\centerline{
\begin{forest}
[H\feattab{\spr \ibox{1}}%,\\
           %\comps \eliste}
  [\alert{\ibox{2}}]
  [H\feattab{\spr \ibox{1} $\oplus$ \sliste{ \alert{\ibox{2}} },\\
             \comps \eliste}
  ]]
\end{forest}}

\vfill
\ibox{2} wird vom Kopf bestimmt. Linke Tochter muss dazu passen.
\pause
\vfill
\hfill
\scalebox{.8}{%
\begin{forest}
sm edges
[V\feattab{\spr \eliste,\\
           \comps \eliste}
  [{\ibox{1} NP[\type{nom}]} [she]]
  [V\feattab{\spr \sliste{ \ibox{1} NP[\type{nom}] },\\
             \comps \eliste} [sleeps]]]
\end{forest}}
\hfill
\scalebox{.8}{%
\begin{forest}
sm edges
[V\feattab{\spr \eliste,\\
           \comps \eliste}
  [{\ibox{1} NP[\type{nom}]} [the brown squirrel,roof]]
  [V\feattab{\spr \sliste{ \ibox{1} NP[\type{nom}] },\\
             \comps \eliste} [sleeps]]]
\end{forest}}\hfill\mbox{}
\vfill

\end{itemize}

}

\frame{
\frametitle{Die \compsl im Spezifikatorschema}


\begin{itemize}
\item Die \compsl ist leer. 
\vfill
\centerline{
\begin{forest}
[H\feattab{\spr \ibox{1}}%,\\
           %\comps \eliste}
  [\ibox{2}]
  [H\feattab{\spr \ibox{1} $\oplus$ \sliste{ \ibox{2} },\\
             \comps \eliste}
  ]]
\end{forest}}
\vfill
Erst alle Komplemente mit einem Kopf kombiniert, dann der Spezifikator.
%\eal
%\ex {}[a [picture [of Kim]]]
\ea {}[The dolphin [attacked [the shark]]]
\z

\end{itemize}

}


\frame{
\frametitle{Nominalstrukturen}

\begin{itemize}
\item Neben NP + VP-Strukturen wird das Spezifikator-Schema auch in Nominalstrukturen verwendet.

\vfill
\centerline{
\begin{forest}
[{N[\spr \eliste, \comps \eliste]}
  [\ibox{1} Det [the]]
  [{N[\spr \sliste{ \ibox{1} }, \comps \eliste]} [squirrel]]]
\end{forest}}

\end{itemize}

\vfill

}


\subsubsection{Kopf-Komplement-Strukturen}

\frame{
\frametitle{Kopf-Komplement-Strukturen}

\begin{itemize}
\item So genannte Bilder-Nomina verlangen ein Komplement und sind parallel zu Verben in SVO-Strukuren:
\ea
a picture of Kim
\z

\centerline{%
\begin{forest}
[{H[\comps \ibox{1}]}
  [{H[\comps  \sliste{ \ibox{2} } $\oplus$ \ibox{1}  ]}]
  [\ibox{2}]]
\end{forest}}

\pause
\item Kombination ditransitives Verb in Teilschritten:
\begin{itemize}
\item \emph{gave} und \emph{the child} 
\item \emph{gave the child} und \emph{a book}
\end{itemize}
\ea
\label{ex-nobody-gives-him-the-book}
Nobody [[gave [the child]] [a book]].
\z

\end{itemize}

}

\frame{
\frametitle{Satz mit ditransitivem Verb}

\centerfit{%
\begin{forest}
sm edges
[{V[\spr \eliste, \comps \eliste]}, visible on=6-
   [{\ibox{1} NP[\type{nom}]}, visible on=5- 
     [nobody, visible on=5-] ]
   [V\feattab{
      \spr \sliste{ \ibox{1} NP[\type{nom}] }, \comps \sliste{}}, visible on=4-
     [V\feattab{
         \spr \sliste{ \ibox{1} NP[\type{nom}] },\\
         \comps \sliste{ \ibox{2} NP[\type{acc}] }}, visible on=2- 
        [V\feattab{
           \spr \sliste{ \ibox{1} NP[\type{nom}] },\\
           \comps \sliste{ \ibox{3} NP[\type{acc}], \ibox{2} NP[\type{acc}]}} [gave] ]
        [{\ibox{3} NP[\type{acc}]} [the child,roof] ] ]
     [{\ibox{2} NP[\type{acc}]}, visible on=3- 
       [a book,roof, visible on=3- ] ] ] ]
\end{forest}}

Schemata lizenzieren Teilbäume.

}


\frame{
\frametitle{Fehlende Details}


\begin{itemize}
\item Bisher zur Vereinfachung einige Spezifikation von \spr und \comps weggelassen.

\centerline{%
\begin{forest}
[{H[\comps \ibox{1}]}
  [{H[\comps  \sliste{ \ibox{2} } $\oplus$ \ibox{1}  ]}]
  [\ibox{2}]]
\end{forest}}

\pause
\item Wenn der \spr-Wert nicht beschränkt ist, kann er irgendwelche Werte haben.

\pause
\item Zum Beispiel eine Liste mit zwei Genitiven und einem Akkusativ.\\
Damit könnten wir dann (\mex{1}) analysieren:

\ea[*]{
his his him gave the child a book
}
\z
\pause
\item Müssen \spr und \comps-Werte am Mutterknoten beschränken:

\ea\label{schema-head-spr-and-head-comps}
Specifier-Head Schema and Head-Complement Schema (final)
\begin{tabular}[t]{@{}l@{~}l@{ }l@{}}
a. & H[\spr \ibox{1}, \comps \alert{\ibox{2}}] & $\to$ H[\spr \ibox{1} $\oplus$ \sliste{ \ibox{3} }, \comps \alert{\ibox{2}} \eliste]\hspace{1em}\ibox{3}  \\
b. & H[\spr \alert{\ibox{1}}, \comps \ibox{2}] & $\to$ H[\spr \alert{\ibox{1}}, \comps \ibox{2} $\oplus$ \sliste{ \ibox{3} }]\hspace{1em}\ibox{3} \\
\end{tabular}
\z

\end{itemize}
}

\frame{
\frametitle{Endgültige Versionen der Schemata}

\begin{itemize}
\item Specifier-Head-Schema und Head-Complement-Schema:
\vfill
\hfill
\begin{forest}
[H\feattab{\spr \ibox{1},\\
           \comps \ibox{2} }
  [\ibox{3}]
  [H\feattab{\spr \ibox{1} $\oplus$ \sliste{ \ibox{3} },\\
              \comps \ibox{2} \eliste}]]
\end{forest}
\hfill
\begin{forest}
[H\feattab{\spr \ibox{1},\\
           \comps \ibox{2}}
  [H\feattab{\spr \ibox{1},\\
             \comps  \sliste{ \ibox{3} } $\oplus$ \ibox{2}  ]}]
  [\ibox{3}]]
\end{forest}
\hfill\mbox{}
\vfill
\end{itemize}

}


\subsection{Scrambling and free VO/OV order}

\frame{

\frametitle{Scrambling}

\begin{itemize}
\item Bisher ein Element vom Anfang oder Ende der \compsl mit Kopf verbunden.
\pause
\item Das funktioniert für das Englische, aber keine Erklärung für Scrambling.

\begin{figure}
\begin{forest}
[{H[\comps \rot{\ibox{1}} $\oplus$ \blau{\ibox{2}}]}
  [\gruen{\ibox{3}}]
  [{H[\comps  \rot{\ibox{1}} $\oplus$ \sliste{ \gruen{\ibox{3}} } $\oplus$ \blau{\ibox{2}}  ]}]]
\end{forest}
\end{figure}

\item Länge von \ibox{1} und \ibox{2} ist nicht beschränkt. Für ditransitives Verb:

\eal
\ex \rot{\eliste{}} $\oplus$ \sliste{ \gruen{NP[\type{nom}]} } $\oplus$ \blau{\sliste{ NP[\type{dat}], NP[\type{acc}] }}
\pause
\ex \rot{\sliste{ NP[\type{nom}] }} $\oplus$ \sliste{ \gruen{NP[\type{dat}]} } $\oplus$ \blau{\sliste{ NP[\type{acc}] }} 
\pause
\ex \rot{\sliste{ NP[\type{nom}], NP[\type{dat}] }} $\oplus$ \sliste{ \gruen{NP[\type{acc}]} } $\oplus$ \blau{\eliste} 
\zl
%So \ibox{3} in Figure~\ref{fig-head-comp-free} would be \npnom in (\mex{0}a), \npdat in (\mex{0}b) and \npacc in (\mex{0}c).

\pause
\item \ibox{1} = leere Liste $\to$ VO-Sprachen mit fester Stellung wie Englisch
\pause
\item \ibox{2} = leere Liste $\to$ OV-Sprache mit fester Stellung
\pause
\item keine Restriktionen für \ibox{1} und \ibox{2} $\to$ beliebige Anordnung

\end{itemize}

}

\subsection{Linearisierungsregeln}

\frame{
\frametitle{Linearisierungsregeln}

\begin{itemize}
\item Die Schemata sind sehr abstrakt. Ähneln den \xbar-Regeln.
\pause
\item Allerdings ist Reihenfolge nicht festgelegt.\\
      In Schemata wie (\mex{1}) kann a vor b und b vor a stehen.
\ea
m $\to$ a b
\z
\pause
\item Head-Complement-Schema kann beide Abfolgen:\\
      Kopf Komplement und Komplement Kopf

\eal
\ex
%\gll 
dem Kind  ein Buch gibt\\
%     the child the book gives\\
\ex gives the child the book
\zl


\item Töchter ohne Beschränkungen machen Unfug:


\eal
\ex[]{
%\gll 
\dass{} niemand dem Kind  ein Buch vorliest\\
%     \that{} nobody  the child a book \partic.reads\\
%\glt `that nobody reads a book to the child'
}
\ex[*]{ 
%\gll 
\dass{} dem Kind  niemand vorliest ein Buch\\
%     \that{} the child nobody  \partic.reads a book\\
}
\ex[*]{
%\gll 
\dass{} niemand vorliest dem Kind   ein Buch\\
%     \that{} nobody  \partic.reads the child a book\\
}
\zl
\end{itemize}


}

\frame{
\frametitle{Unerwünschte Struktur}

% \begin{forest}
% sm edges
% [{V[\comps \sliste{ }]},s sep+=1em
%   [{V[\comps \sliste{ \ibox{1} }]}
%     [\ibox{2} \npdat [dem Kind;the child,roof]]
%     [{V[\comps \sliste{ \ibox{2}, \ibox{1} }]} [\ibox{3} \npnom [niemand;nobody]]
%        [{V[\comps \sliste{ \ibox{3}, \ibox{2}, \ibox{1} }]}  [vorliest;\textsc{part}.reads]]]]
%   [\ibox{1} \npacc [ein Buch;a book,roof]]]]
% \end{forest}

% \begin{forest}
% sm edges
% [{V[\comps \sliste{ }]}
%   [{V[\comps \sliste{ \ibox{1} }]},s sep+=1em
%     [{V[\comps \sliste{ \ibox{2}, \ibox{1} }]} [\ibox{3} \npnom [niemand;nobody]]
%        [{V[\comps \sliste{ \ibox{3}, \ibox{2}, \ibox{1} }]}  [vorliest;\textsc{part}.reads]]]
%     [\ibox{2} \npdat [dem Kind;the child,roof]]] 
%   [\ibox{1} \npacc [ein Buch;a book,roof]]]]
% \end{forest}
\vfill
\centerline{%
\begin{forest}
sm edges
[{V[\comps \sliste{ }]}
  [{V[\comps \sliste{ \ibox{1} }]},s sep+=1em
    [{V[\comps \sliste{ \ibox{2}, \ibox{1} }]} [\ibox{3} \npnom [niemand]]
       [{V[\comps \sliste{ \ibox{3}, \ibox{2}, \ibox{1} }]}  [vorliest]]]
    [\ibox{2} \npdat [dem Kind,roof]]] 
  [\ibox{1} \npacc [ein Buch,roof]]]]
\end{forest}}
\vfill

}

\frame{
\frametitle{Linearisierungsregeln für Kopf und Komplement}

\begin{itemize}
\item Regeln:
\eal
\label{lp-regeln}
\ex HEAD [\textsc{initial}+] $<$ COMPLEMENT
\ex COMPLEMENT $<$  HEAD [\textsc{initial}$-$]
\zl
\item Deutsche Verben (SOV): \textsc{initial}$-$\\
      Englische Verben (SVO): \textsc{initial}$+$

\pause
\item Deutsche und Englische Nomina: \textsc{initial}$+$

\end{itemize}

}


\frame{
\frametitle{Übungsaufgaben}


\begin{enumerate}
\item Geben Sie die Valenzlisten für folgende Wörter an:
\eal
\ex lachen
\ex essen
\ex übergießen
\ex bezichtigen
\ex er
\ex der
\zl
\item Zeichnen Sie die Bäume für folgende Beispiele:
\eal
\ex weil der Mann ihm ein Buch schenkt
\ex because the man gave the book to him
\ex
\gll at Bjarne læste bogen\\
     dass Bjarne las Buch.{\sc def}\\
\glt `dass Bjarne das Buch las'
\zl
\end{enumerate}


} 


\subsection{Adjunkte}

\frame{
\frametitle{Adjunkte}

\begin{itemize}
\item Argumente werden von ihrem Kopf ausgewählt.
\pause
\item Adjunkte wählen sich ihren Kopf.
\pause
\item Deutsch, Niederländisch, \ldots:\\
      Adjunkte im Satz gehen an irgendeine Verbprojektion (Verb in Endstellung).
\pause
\item Englisch, Dänisch, \ldots: Adjunkte gehen an VP.

\eal
\ex that everybody \gruen{reads the book} \rot{promptly}
\ex that everybody \rot{promptly} \gruen{reads the book}
\zl

\end{itemize}

}

\frame{
\frametitle{Das \modm}

\begin{itemize}
\item Analog zu \spr und \comps gibt es \textsc{mod}:\\

\ea
\begin{tabular}[t]{@{}l@{}}
Lexical item for \emph{brown}:\\
\ms{
  phon & \phonliste{ brown }\\
  mod  & \nbar\\
  spr  & \eliste\\
  comps & \eliste }
\end{tabular}
\hfill\begin{forest}
sm edges
[{\nbar}, baseline
  [{Adj[\textsc{mod} \ibox{2}]} [brown]]
  [{\ibox{2} \nbar} [squirrel]]]
\end{forest}\hfill\mbox{}
\z

\pause
\item \modw ist eine Beschreibung oder \textit{none}.



\end{itemize}


}

\frame{
\frametitle{Das Kopf-Adjunkt-Schema}

\centerline{%
\begin{forest}
[{H[\spr \ibox{1}, \comps \ibox{2}]}
  [{[\textsc{mod} \ibox{3}, \spr \eliste, \comps \eliste]}]
  [{\ibox{3} H[\spr \ibox{1}, \comps  \ibox{2}]}]]
\end{forest}}

\pause
\begin{itemize}
\item Wie das \spr- oder \comps-Merkmal gibt es auch ein \modm.\\
       Wert von {\sc mod} ist eine Beschreibung des zu modifizierenden Kopfes:
\begin{itemize}
\item Deutsch: {\sc mod} V[{\sc ini}$-$]
\item Englisch: {\sc mod} VP
\end{itemize}
\end{itemize}


}

\frame{
\frametitle{Freie Position der Adjunkte im Deutschen}

\vfill
\centerfit{
\begin{forest}
sm edges
[{V[\spr \eliste, \comps \eliste]}, schema
        [{Adv[\textsc{mod} \ibox{3} V]} [morgen] ]
        [{\ibox{3} V[\spr \eliste, \comps \eliste]}
          [{\ibox{1} NP[\type{nom}]} [Aicke] ]
          [V\feattab{
              \spr \sliste{ }, \comps \sliste{ \ibox{1} } }
            [{\ibox{2} NP[\type{acc}]} [das Buch, roof] ] 
            [V\feattab{
              \spr \sliste{  },\\
              \comps \sliste{ \ibox{1}, \ibox{2} }} [liest] ] ]
] ]
\end{forest}}


%\centerline{{}[dass] morgen Aicke das Buch liest}

\vfill

}


\frame{
\frametitle{Freie Position der Adjunkte im Deutschen}

\vfill


\centerfit{%
\begin{forest}
sm edges
[{V[\spr \eliste, \comps \eliste]},s sep+=1.5em
          [{\ibox{1} NP[\type{nom}]} [Aicke] ]
          [V\feattab{
              \spr \sliste{ }, \comps \sliste{ \ibox{1} } }, schema
            [{Adv[\textsc{mod} \ibox{3} V]} [morgen] ]
            [\ibox{3} V\feattab{
                \spr \sliste{ }, \comps \sliste{ \ibox{1} } }
              [{\ibox{2} NP[\type{acc}]} [das Buch, roof] ] 
              [V\feattab{
                \spr \sliste{  },\\
                \comps \sliste{ \ibox{1}, \ibox{2} }} [liest] ] ]
] ]
\end{forest}}

%\centerline{{}[dass] Aicke morgen das Buch liest}
\vfill


}



\frame{
\frametitle{Freie Position der Adjunkte im Deutschen}

\vfill

\centerfit{%
\begin{forest}
sm edges
[{V[\spr \eliste, \comps \eliste]}
    [{\ibox{1} NP[\type{nom}]} [Aicke] ]
      [V\feattab{
         \spr \sliste{ }, \comps \sliste{ \ibox{1} } }, s sep+=1em
         [{\ibox{2} NP[\type{acc}]} [das Buch, roof] ] 
           [V\feattab{
              \spr \sliste{  },\\
              \comps \sliste{ \ibox{1}, \ibox{2} }}, schema 
             [{Adv[\textsc{mod} \ibox{3} V]} [morgen] ]
             [\ibox{3} V\feattab{
                 \spr \sliste{  },\\
                 \comps \sliste{ \ibox{1}, \ibox{2} }} [liest] ] ] ] ]
\end{forest}}

%\centerline{{}[dass] Aicke das Buch morgen liest}

\vfill

}

\frame{
\frametitle{Feste Position der Adjunkte im Englischen}


\vfill

\centerfit{%
\begin{forest}
sm edges
[{V[\spr \eliste, \comps \eliste]}, s sep+=1.5em % puts more space between the NP[nom] and the VP,
                                    % otherwise the box would overlap
          [{\ibox{1} NP[\type{nom}]} [Kim] ]
          [V\feattab{
              \spr \sliste{ \ibox{1} }, \comps \sliste{  } }, schema
            [{Adv[\textsc{mod} \ibox{3}]} [often] ]
            [\ibox{3} V\feattab{
                \spr \sliste{ \ibox{1} }, \comps \sliste{  } }
              [V\feattab{
                \spr \sliste{ \ibox{1} },\\
                \comps \sliste{  \ibox{2} }} [reads] ]
              [{\ibox{2} NP[\type{acc}]} [books] ] ]
] ]
\end{forest}}

\vfill

}

\frame{
\frametitle{Feste Position der Adjunkte im Englischen}

\vfill

\centerfit{%
\begin{forest}
sm edges
[{V[\spr \eliste, \comps \eliste]},s sep+=1em
          [{\ibox{1} NP[\type{nom}]} [Kim] ]
          [V\feattab{
              \spr \sliste{ \ibox{1} }, \comps \sliste{  } }, schema
            [\ibox{3} V\feattab{
                \spr \sliste{ \ibox{1} }, \comps \sliste{  } }
              [V\feattab{
                \spr \sliste{ \ibox{1} },\\
                \comps \sliste{ \ibox{2} }} [reads] ]
              [{\ibox{2} NP[\type{acc}]} [books] ] 
               ]
            [{Adv[\textsc{mod} \ibox{3}]} [often] ]
] ]
\end{forest}}

\vfill


}


\frame{
\frametitle{Details}

\begin{itemize}
\item Adjunkte ändern nichts an der Valenz/Sättigung.\\ 
(Ob ich Gummibärchen einpacke o.\ nicht, beeinflusst meine Einkaufsliste nicht.)

\vfill
\centerline{%
\begin{forest}
[{H[\spr \rot<1>{\ibox{1}}, \comps \gruen<1>{\ibox{2}}]}
  [{[\textsc{mod} \ibox{3},  \gruen<3>{\spr \eliste}, \gruen<2>{\comps \eliste}]}]
  [{\ibox{3} H[\spr \rot<1>{\ibox{1}}, \comps  \gruen<1>{\ibox{2}}]}]]
\end{forest}}

\vfill

\pause
\item Adjunkte müssen vollständig sein. Sonst:
\ea[*]{
Sandy read the book in.
}
\z

\pause
\eal
\ex[]{
%\gll 
dass Aicke eine Stunde liest
%     that Aicke an   hour reads\\\german
%\glt `Aicke is reading for an hour.'
}
\ex[*]{
%\gll 
dass Aicke Stunde liest
%     that Aicke hour reads\\
%\glt `Aicke is reading for an hour.'
}
\zl

\end{itemize}

}

\subsection{Linking}

\frame[shrink]{
\frametitle{Linking}

\begin{itemize}
\item Alle hier behandelten Sprachen haben eine Liste mit Valenzinformation:
\ea
\sliste{ NP, NP, NP }
\z
\pause
\item Diese wird Argument Structure genannt (\argst). 
%\spr und \comps werden davon abgeleitet.
\pause
\item Die Kasus unterscheiden sich, das besprechen wir später.
\pause
\item Reihenfolge ist aber gleich.
\eal
\ex dass das Kind dem Eichhörnchen die Nuss gibt\\
%    that the child  the squirrel    the nut gives\\
%\glt `that the child gives the squirrel the nut'
\ex that the child gives the squirrel the nut
\zl
\pause
\item Verbindung zischen Syntax und Semantik ist für behandelte Sprachen gleich:

\ea
Lexikoneintrag für \emph{gives}/\emph{gibt}:\\*
\scalebox{.9}{
\ms{
arg-st & \sliste{ NP\ind{1}, NP\ind{2}, NP\ind{3} }\\[2mm]
cont   & \ms[give]{
          agens & \ibox{1}\\
          goal  & \ibox{2}\\
          trans-obj & \ibox{3}\\
        }\\
}}
\z
\end{itemize}


}


%      <!-- Local IspellDict: en_US-w_accents -->
