%% -*- coding:utf-8 -*-
\author{Stefan Müller}

\subtitle{Valence, argument order and adjunct placement}

\section{Valence, argument order and adjunct placement}


\huberlintitlepage[22pt]


\outline{

\begin{itemize}
%\item {Überblick über die germanischen Sprachen}
\item Phenomena
\item Phrase structure grammars and \xbar theory
\item \alert{Valence, order of arguments and adjuncts}
\item Verb clusters in SOV langauges
\item Verbposition: Verb first and verb second order
\item Passive
%\item embedded sentences
\end{itemize}

}



\frame{
\frametitle{Literature}



Please read \citew[Chapter~4]{MuellerGermanic}.

\begin{refsection}

\nocite{MuellerGermanic}

\printbibliography[heading=none,notkeyword=this]

\end{refsection}

\pause

The theory that is used in the following is Head-Driven Phrase Structure Grammar in a reduced version.

For HPSG, see \citew{ps,ps2,Sag97a,MuellerLehrbuch,HPSGHandbook}.



}

\subsection{Valence}



\frame{
\frametitle{Presence of certain constituents}

The sequences in (\mex{1}) are not well-formed:
\eal
\ex[*]{
that the dolphin devours 
}
\ex[*]{
that of the dolphin the child the ball him his the child gives
}
\zl
Something is missing in (\mex{0}a), there is too much in (\mex{0}b).


}

\frame{
\frametitle{Valenz in der Chemie und in der Linguistik}


\citet{Tesniere59a-Eng} adapts the concept of valence from chemistry:


\vfill

\centerline{
\begin{forest}
[O
  [H] 
  [H] ]
\end{forest}
\hspace{5em}
\begin{forest}
[kennen
 [Aicke]
 [Conny] ]
\end{forest}
}

\vfill

}

\frame{
\frametitle{Arguments}

NPs in (\mex{1}) are arguments of the respective verbs:
\eal
\ex[]{
\gll
\dass der        Delphin den        Menschen erwartet\\
     \that{}  the.\NOM{} dolphin den.\ACC{} human   expects\\
\glt `that the dolphin expects the human'
}
\ex[]{
\gll 
{}[dass] der        Delphin dem        Kind  den        Ball gibt\\
     \that{}  the.\NOM{} dolphin the.\DAT{} child the.\ACC{} ball gives\\
\glt `that the dolphin gives the child the ball'
}
\zl

Syntactic arguments usually fill a semantic role (\zb giver, agent, actor,
\ldots).\nocite{Dowty91a,VanValin99a-u}

\citet[Chapter~48]{Tesniere2015a-u}: 

drama scene: What is needed to act out a giving event?

\begin{itemize}
\item a giver
\item something given
\item a givee (recipient)
\end{itemize}

}

\frame{
\frametitle{Adjucts}

\begin{itemize}
\item There can be adjuncts in addition to arguments:

\eal
\ex
\gll
\dass{} der        Delphin dem        Kind  \alert{schnell} den        Ball gibt\\
\that{} the.\NOM{} dolphin the.\DAT{} child quickly the.\ACC{} ball gives\\
\glt `that the dolphin gives the child the ball quickly'
\ex that the dolphin gives the child the ball \alert{quickly}
\zl

\pause

\item Adjuncts convey additional information but do not fill a semantic role.

\end{itemize}

}

\frame{
\frametitle{Optional arguments}

Almost all arguments can be omitted, provided context provides enough info.

\eal
\ex 
\gll Sie gibt Geld.\\
     she gives money\\
\glt `She gives money.'
\ex 
\gll Sie gibt den Armen.\\
     she gives the poor\\
\glt `She gives to the poor.'
\ex\label{ex-sie-gibt} 
\gll  Sie gibt.\\
     she gives\\
\ex 
\gll Gib!\\
     give\\
\zl

If we are playing the card game skat, it is clear who gives what to whom.

}

\frame{
\frametitle{Obligatory arguments (rare)}

There are some verbs which really have obligatory arguments:
\eal
\ex 
\gll verschlingen\\
     devour\\\german
\ex 
\gll erwarten\\
     await\\
\zl

Arguments may be optional, arguments always are. 


}

\frame{
\frametitle{Chemistry and optional elements}

The analogy with chemical bonds is helpful, but optional arguments remain confusing:
\eal
\ex Kirby helps Sandy.
\ex Kirby helps.
\zl

\hfill
\begin{forest}
[O
  [H] 
  [H] ]
\end{forest}
\hspace{5em}
\begin{forest}
[helps
 [Kirby]
 [Sandy] ]
\end{forest}
\hfill\mbox{}

Solution: syntactic and semantic valency.

Theater analogy helps us to find semantic arguments,\\
chemistry analogy helps for syntactic arguments.

One has to assume a special, one place verb for (\mex{0}b).

}

\frame{
\frametitle{Shopping instead of drama}

\begin{itemize}
\item The analogy to shopping is better:\\
We want to prepare pasta with tofu and tomato sauce.

For tomato sauce we need onions.

\item All ingredients are added to a shopping list.

\item The shop does not have tofu.

\item No problem, we can do pasta without tofu: tofu is optional.

\item There are noodles. 10.000 types of pasta.

\item Tomatos and onions. Done.

\item Ah. Gummy bears. What? They were not on the list?

Gummy bears are adjuncts!

\end{itemize}

}

\frame{
\frametitle{Pasta, tomatos, onions and syntax}

\begin{itemize}
\item There are several ways to make sure that all that is needed is present.

\item Phrase structures as in GPSG 
\ea
\label{ditrans-schema-two}
\begin{tabular}[t]{@{}l@{ }l@{ }l}
S  & $\to$ & NP[\type{nom}] NP[\type{dat}] NP[\type{acc}] V[\type{ditransitive}]\\
\end{tabular}
\z

\pause
\item These were dropped in favour of lexical approaches\\
  \parencites{Jacobson87b}[Section~5.5]{MuellerGT-Eng1}{MWArgSt}
\pause
\item reasons: 
\begin{itemize}
\item Partial VP Fronting \citep{Nerbonne86a,Johnson86a} 
\pause
\item Interactions with morphologie \citep[Section~5.5.1]{MuellerGT-Eng1}
\end{itemize}
\pause
\item Come back of phrasal approaches in Construction Grammar \citep{Goldberg95a}. These do not work.

\parencites{Mueller2006d,MuellerPersian,MuellerUnifying,MWArgSt,MWArgStReply,MuellerFCG,MuellerLFGphrasal,MuellerPotentialStructure,MuellerGT-Eng4,MuellerCxG}

\end{itemize}

}

\frame{
\frametitle{Valenzlisten}

\begin{itemize}
\item Arguments are represented in lists:
\ea
\label{valence-specifications-German}
\begin{tabular}[t]{@{}l@{~}l@{~}l}
a. & \emph{schläft} `sleeps':         & \sliste{ NP[\type{nom}] }\\
b. & \emph{kennt} `knows':           & \sliste{ NP[\type{nom}], NP[\type{acc}] }\\
%b. & \emph{unterstützt}:  & \sliste{ NP[\type{nom}], NP[\type{acc}] }\\
c. & \emph{hilft} `helps':           & \sliste{ NP[\type{nom}], NP[\type{dat}] }\\
d. & \emph{gibt} `gives':            & \sliste{ NP[\type{nom}], NP[\type{dat}], NP[\type{acc}] }\\
e. & \emph{wartet} `waits':          & \sliste{ NP[\type{nom}], PP[\type{auf}] }\\
\end{tabular}
\z

NP[\type{nom}] stands for something like \emph{Nudeln}.\\
There are many ways to satisfy these valence requirements:\\
\emph{sie} `she', \emph{das Kind} `the child', \emph{der lachende Delphin} `the laughing dolphin', \ldots

\pause
\item Elements of lists are ordered in a certain order. This order corresponds to the order in
  English and other languages with fixed order and to the unmarked order in German \citep{Hoehle82a}. 


\end{itemize}



}


\frame{
\frametitle{Valence determining structure}

\vfill
\centerfit{
\begin{forest}
sm edges
[{S \eliste},visible on=6-
  [{NP[\type{nom}]},visible on=5- 
    [niemand,visible on=5-] ]
  [{V$'$\sliste{ \blau<4>{NP[\type{nom}]} } },visible on=3-
    [\blau<3>{NP[\type{acc}]}, visible on=2-
      [ihn,visible on=2-] ]
    [{V \sliste{ \blau<4>{NP[\type{nom}]}, \blau<3>{NP[\type{acc}]}}} [kennt]] ] ]
\end{forest}}

\vfill
\begin{itemize}
\item Valence requirements are represented in a list.
\pause
\item One element of the list is combined with a head.\\
\pause\pause
      List with remaining elements is passed up.
\pause\pause\pause
\item Like shopping with an app:\\
      One element after the other is removed from the list.
%\pause
%\item Wichtig: Die Theorie sagt nichts darüber aus welcher
\end{itemize}

}

\frame{
\frametitle{Constraint-based theories and psycholinguistics}

\centerfit{
\begin{forest}
sm edges
[{S \eliste},visible on=4-
%  [{[dass]},no edge,visible on=3-]
  [{NP[\type{nom}]},visible on=3- 
    [niemand,visible on=3-] ]
  [{V$'$\sliste{ NP[\type{nom}] } },visible on=4-
    [{NP[\type{acc}]},visible on=5- 
      [ihn,visible on=5-] ]
    [{V \sliste{ NP[\type{nom}], NP[\type{acc}] }},visible on=6- [kennt,visible on=7-]] ] ]
\end{forest}}

\begin{itemize}
\item I explain things bottom up.
\pause
\item But this is not required by the theory.

This is important from a psycholinguistic point of view, since processing is incremental.\\
\parencites{Marslen-Wilson75a,TSKES96a,SW2011a,Wasow2021a}
\pause

\end{itemize}




}


\frame{
\frametitle{Optional Arguments?}

\begin{itemize}
\item There are several ways to deal with optional arguments.
\item The most obvious: alternative lexical items.
% \ea
% \begin{tabular}[t]{@{}l@{~}l@{~}l}
% a. & \emph{gibt} `gives':            & \sliste{ NP[\type{nom}] }\\
% b. & \emph{wartet} `waits':          & \sliste{ NP[\type{nom}] }\\
% \end{tabular}
% \z
\ea
\begin{tabular}[t]{@{}l@{~}l@{~}l}
a. & \emph{gibt}:            & \sliste{ NP[\type{nom}] }\\
b. & \emph{wartet}:          & \sliste{ NP[\type{nom}] }\\
\end{tabular}
\z
\end{itemize}



}


\subsection{Scrambling}

\frame{
\frametitle{Scrambling: constituent order in German}

\vfill
\centerfit{
\begin{forest}
sm edges
[{S \eliste}
   [{NP[\type{acc}]} [ihn;him] ]
   [{V$'$\sliste{ NP[\type{acc}] } }
      [{NP[\type{nom}]} [niemand;nobody] ]
      [{V \sliste{ NP[\type{nom}], NP[\type{acc}]}} [kennt;knows] ] ] ]
\end{forest}}
\vfill
\pause
\begin{itemize}
\item An arbitrary element of the list can be combined with the head.\\
      $\to$ order Acc $<$ Nom can be analyzed.\\
      List with remaining elements is passed upwards.
\end{itemize}

}



\subsection{SVO: Danish/English}


\frame{
\frametitle{VP in SVO langauges}

\begin{itemize}
\item Verbs and objects form a group:
\eal
\ex John promised to read the book and [\alert{read the book}], he will.
\ex He will [\alert{read the book}].
\ex Kim [[\alert{sold the car}] and [\alert{bought a bicycle}]]. 
\ex He often [\alert{reads the book}].
%\ex He [reads the book] often.
\ex \ldots{} [often [\alert{read the book}] slowly], he will.
\zl
\pause
\item Can be captured easily with two valency lists:\\
      one for complements (\comps) and one for the subject (\textsc{spr} for \textsc{specifier}).
\end{itemize}


}

\frame{
\frametitle{Danish, English, \ldots}

~\vfill

\centerfit{\begin{forest}
sm edges
[{V[\spr \eliste, \comps \eliste]}, name=S
   [{NP[\type{nom}]} [nobody] ]
   [V\feattab{
      \spr \sliste{ NP[\type{nom}] }, \comps \sliste{} }, name=VP
     [V\feattab{
         \spr \sliste{ NP[\type{nom}] },\\
         \comps \sliste{ NP[\type{acc}]}} [knows] ]
        [{NP[\type{acc}]} [him] ] ] ]
\node [right=4cm] at (S)
    {
        = S
    };
\node [right=4cm] at (VP)
    {
        = VP
    };
\end{forest}}


\vfill

\begin{itemize}
\item English is a SVO language:\\
      complements to the right of the verb, subjects to the left
%\item Komplemente können nicht einfach umgestellt werden.
\item Verb and complements form a phrase (VP = \comps \sliste{}).

      This phrase is combined with the subject.
\end{itemize}

\vfill

}


\frame{
\frametitle{No scrambling}

\begin{itemize}
\item Danish, English, \ldots:\\
      Elementes of the valence list have to be bound off from left to right.

\pause
\item German, Dutch, \ldots:\\
      Elements can be combined with their heads in any order.

%\pause
%\item 
\end{itemize}

}


\frame{
\frametitle{SVO: Ditransitive verbs: Saturation of \comps starting left}


\centerline{
\begin{forest}
sm edges
[{V[\spr \eliste, \comps \eliste]}, visible on=6-
   [{NP[\type{nom}]}, visible on=5- 
     [Kim, visible on=5-] ]
   [V\feattab{
      \spr \sliste{ NP[\type{nom}] }, \comps \sliste{}}, visible on=4-
     [V\feattab{
         \spr \sliste{ NP[\type{nom}] },\\
         \comps \sliste{ PP[\type{to}] }}, visible on=2-
       [V\feattab{
           \spr \sliste{ NP[\type{nom}] },\\
           \comps \sliste{ NP[\type{acc}], PP[\type{to}] }} [gave] ]
         [{NP[\type{acc}]} 
            [a book, roof] ] ]
       [{PP[\type{to}]}, visible on=3- 
         [to Sandy, roof, visible on=3-] ] ] ]
\end{forest}}

For head-final languages without scrambling, combination from right to left.



}

%% \frame{
%% \frametitle{Regeln: Englisch und Deutsch}

%% \begin{itemize}
%% \item Englisch:
%% \ea
%% H[\comps \ibox{B}] $\to$ H[\comps \ibox{B} $\oplus$ \sliste{ \ibox{1} } ] ~~~\ibox{1}
%% \z

%% `$\oplus$' zerlegt Liste in zwei Teillisten.

%% \pause

%% \item Deutsch:
%% \ea
%% H[\comps \ibox{A} $\oplus$ \ibox{B}] $\to$ H[\comps \ibox{A} $\oplus$ \sliste{ \ibox{1} } $\oplus$ \ibox{B} ] ~~~\ibox{1}
%% \z

%% \pause

%% \item Das Englische unterscheidet sich vom Deutschen dadurch,\\
%%       dass \ibox{A} die leere Liste ist.

%% $\to$ Englisch ist restriktiver.

%% \end{itemize}


%% }


\frame{
\frametitle{German}



\centerfit{\begin{forest}
sm edges
[{V[\spr \eliste, \comps \eliste]}, name=S
        [{NP[\type{nom}]} [niemand;nobody] ]
        [{V\feattab{
              \spr \eliste, \comps \sliste{ NP[\type{nom}] } }}, name = Vs
          [{NP[\type{acc}]} [ihn;him] ] 
          [V\feattab{
              \spr \eliste,\\
              \comps \sliste{ NP[\type{nom}], NP[\type{acc}]}} [kennt;knows] ]
] ]
\node [right=4cm] at (S)
    {
        = S
    };
\node [right=4cm] at (Vs)
    {
        = V$'$
    };
\end{forest}}

The subject of a finite verb is in the \comps list \citep{Pollard90a,Kiss95a}.

Abbreviations: \begin{tabular}[t]{@{}l@{ = }l}
             S  & [\spr \eliste, \comps \eliste]\\
             VP & [\spr \sliste{ NP[\type{nom}] }, \comps \sliste{}]\\
             V$'$ & all other projections of V (except verbal complexes)\\
             \end{tabular}

}

\subsection{Immediate Dominance Schemata}

\frame{
\frametitle{Immediate Dominance Schemata}

\begin{itemize}
\item Theoretical papers often just show tree structures.
\item But they do not appear out of the blue.\\
      There are rules or schemata that license them.
\item For example, \xbar schemata or rules for phrase structure or dependency structures.
\item Rules for HPSG:
\ea\label{schema-head-spr-and-head-comps-preliminary}
Specifier-Head Schema and Head-Complement Schema (preliminary)
\begin{tabular}[t]{@{}l@{ }l@{}}
H[\spr \ibox{1}]   & $\to$ H[\spr \ibox{1} $\oplus$ \sliste{ \ibox{2} }, \comps \eliste]\hspace{1em}\ibox{2}  \\
H[\comps \ibox{1}] & $\to$ H[\comps \sliste{ \ibox{2} } $\oplus$ \ibox{1}]\hspace{1em}\ibox{2} \\
\end{tabular}
\z
\pause
\item H stands for \emph{head}. The respective phrase contains or is the head.

\end{itemize}

}


\subsubsection{Specifier head structures}

\frame{
\frametitle{Schemata as partial trees}

\begin{itemize}
\item Schemata can be visualized as partial trees.
\ea\label{schema-head-spr-and-head-comps-preliminary}
Specifier-Head Schema (preliminary)\\
H[\spr \ibox{1}] $\to$ H[\spr \ibox{1} $\oplus$ \sliste{ \ibox{2} }, \comps \eliste]\hspace{1em}\ibox{2}
\z

\item Specifier-Head Schema in tree notation:
\vfill
%\centerline{
\begin{forest}
[H\feattab{\spr \ibox{1}}%,\\
           %\comps \eliste}
  [\ibox{2}]
  [H\feattab{\spr \ibox{1} $\oplus$ \sliste{ \ibox{2} },\\
             \comps \eliste}
  ]]
\end{forest}
%}
\vfill

\end{itemize}

}

\frame{
\frametitle{\texttt{append} ($\oplus$)}

\begin{itemize}
\item \texttt{append} concatenates two lists: \sliste{ \normalfont a } $\oplus$ \sliste{ \normalfont b } =
\sliste{ \normalfont a, b }. 

\pause
\item Concatenating a list L with the empty list results in L.

\pause
\item Example: \texttt{append} divides the list \sliste{ NP[\type{nom}], NP[\type{dat}],
    NP[\type{acc} ] } as follows:
\eal
\ex \oneline{\eliste{} $\oplus$ \sliste{ NP[\type{nom}], NP[\type{dat}], NP[\type{acc}] } = \sliste{ NP[\type{nom}], NP[\type{dat}], NP[\type{acc}] }}
\pause
\ex \oneline{\sliste{ NP[\type{nom}] } $\oplus$ \sliste{ NP[\type{dat}], NP[\type{acc}] } = \sliste{ NP[\type{nom}], NP[\type{dat}], NP[\type{acc}] }}
\pause
\ex \oneline{\sliste{ NP[\type{nom}], NP[\type{dat}] } $\oplus$ \sliste{ NP[\type{acc}] } = \sliste{ NP[\type{nom}], NP[\type{dat}], NP[\type{acc}] }}
\pause
\ex \oneline{\sliste{ NP[\type{nom}], NP[\type{dat}], NP[\type{acc}] } $\oplus$ \eliste{} = \sliste{ NP[\type{nom}], NP[\type{dat}], NP[\type{acc}] }}
\zl

\end{itemize}

}


\frame{
\frametitle{Splitting the valence list}

\begin{itemize}
\item Specifier-Head Schema in tree notation:\\
\vfill
\begin{forest}
[H\feattab{\spr \ibox{1}}%,\\
           %\comps \eliste}
  [\ibox{2}]
  [H\feattab{\spr \ibox{1} $\oplus$ \sliste{ \ibox{2} },\\
             \comps \eliste}
  ]]
\end{forest}
\vfill

\item Schema splits list into an arbitrary list \iboxb{1} and a singleton list (\sliste{ \ibox{2} }).

\item For our example, this would be (\mex{0}c):
\ea
\oneline{\sliste{ NP[\type{nom}], NP[\type{dat}] } $\oplus$ \sliste{ NP[\type{acc}] } = \sliste{ NP[\type{nom}], NP[\type{dat}], NP[\type{acc}] }}
\z

\ibox{1} = \sliste{ NP[\type{nom}], NP[\type{dat}] } und \ibox{2} = NP[\type{acc}]

\end{itemize}

}

\frame{
\frametitle{Splitting the \spr list in specifier head structures}

\begin{itemize}
\item The \spr list usually has just one element: NP[\type{nom}]/subject for verbs in SVO languages or
  determiner, if the head is a noun.

\vfill
\begin{forest}
[H\feattab{\spr \ibox{1}}%,\\
           %\comps \eliste}
  [\ibox{2}]
  [H\feattab{\spr \ibox{1} $\oplus$ \sliste{ \ibox{2} },\\
             \comps \eliste}
  ]]
\end{forest}

\vfill

\pause
\item \ibox{1} is the empty list, \ibox{2} is NP[\type{nom}] or Det. 


\eal
\ex \eliste{} $\oplus$ \sliste{ NP[\type{nom}] } = \sliste{ NP[\type{nom}] }
\ex \eliste{} $\oplus$ \sliste{ Det } = \sliste{ Det }
\zl

\end{itemize}

}

\exewidth{(235)}

\frame{
\frametitle{Head has requirements. Schema cares for fulfillment}

\begin{itemize}
\item Description of the daughter has to match with the daughter to be inserted.

\vfill
\centerline{
\begin{forest}
[H\feattab{\spr \ibox{1}}%,\\
           %\comps \eliste}
  [\alert{\ibox{2}}]
  [H\feattab{\spr \ibox{1} $\oplus$ \sliste{ \alert{\ibox{2}} },\\
             \comps \eliste}
  ]]
\end{forest}}

\vfill
\ibox{2} is determined by the head. Left daughter has to match.
\pause
\vfill
\hfill
\scalebox{.8}{%
\begin{forest}
sm edges
[V\feattab{\spr \eliste,\\
           \comps \eliste}
  [{\ibox{1} NP[\type{nom}]} [she]]
  [V\feattab{\spr \sliste{ \ibox{1} NP[\type{nom}] },\\
             \comps \eliste} [sleeps]]]
\end{forest}}
\hfill
\scalebox{.8}{%
\begin{forest}
sm edges
[V\feattab{\spr \eliste,\\
           \comps \eliste}
  [{\ibox{1} NP[\type{nom}]} [the brown squirrel,roof]]
  [V\feattab{\spr \sliste{ \ibox{1} NP[\type{nom}] },\\
             \comps \eliste} [sleeps]]]
\end{forest}}\hfill\mbox{}
\vfill

\end{itemize}

}

\frame{
\frametitle{The \comps list in the specifier schema}


\begin{itemize}
\item The \comps list is empty. 
\vfill
\centerline{
\begin{forest}
[H\feattab{\spr \ibox{1}}%,\\
           %\comps \eliste}
  [\ibox{2}]
  [H\feattab{\spr \ibox{1} $\oplus$ \sliste{ \ibox{2} },\\
             \comps \eliste}
  ]]
\end{forest}}
\vfill
First all complements are combined with a head, then the specifier.
%\eal
%\ex {}[a [picture [of Kim]]]
\ea {}[The dolphin [attacked [the shark]]]
\z

\end{itemize}

}


\frame{
\frametitle{Nominal structures}

\begin{itemize}
\item The specifier schema is not just used for NP-VP structures,\\ but also for nominal structures.

\vfill
\centerline{
\begin{forest}
[{N[\spr \eliste, \comps \eliste]}
  [\ibox{1} Det [the]]
  [{N[\spr \sliste{ \ibox{1} }, \comps \eliste]} [squirrel]]]
\end{forest}}

\end{itemize}

\vfill

}


\subsubsection{Head complement structures}

\frame{
\frametitle{Head complement structures}

\begin{itemize}
\item Picture nouns require a complement and are parallel to verbs in SVO structures:
\ea
a picture of Kim
\z

\centerline{%
\begin{forest}
[{H[\comps \ibox{1}]}
  [{H[\comps  \sliste{ \ibox{2} } $\oplus$ \ibox{1}  ]}]
  [\ibox{2}]]
\end{forest}}

\pause
\item stepwise combination with a ditransitive verb:
\begin{itemize}
\item \emph{gave} und \emph{the child} 
\item \emph{gave the child} und \emph{a book}
\end{itemize}
\ea
\label{ex-nobody-gives-him-the-book}
Nobody [[gave [the child]] [a book]].
\z

\end{itemize}

}

\frame{
\frametitle{Sentence with a ditransitive verb}

\centerfit{%
\begin{forest}
sm edges
[{V[\spr \eliste, \comps \eliste]}, visible on=6-
   [{\ibox{1} NP[\type{nom}]}, visible on=5- 
     [nobody, visible on=5-] ]
   [V\feattab{
      \spr \sliste{ \ibox{1} NP[\type{nom}] }, \comps \sliste{}}, visible on=4-
     [V\feattab{
         \spr \sliste{ \ibox{1} NP[\type{nom}] },\\
         \comps \sliste{ \ibox{2} NP[\type{acc}] }}, visible on=2- 
        [V\feattab{
           \spr \sliste{ \ibox{1} NP[\type{nom}] },\\
           \comps \sliste{ \ibox{3} NP[\type{acc}], \ibox{2} NP[\type{acc}]}} [gave] ]
        [{\ibox{3} NP[\type{acc}]} [the child,roof] ] ]
     [{\ibox{2} NP[\type{acc}]}, visible on=3- 
       [a book,roof, visible on=3- ] ] ] ]
\end{forest}}

Schemata license partial trees.

}


\frame{
\frametitle{Missing Details}


\begin{itemize}
\item Until now some specifications of \spr and \comps values were omitted.

\centerline{%
\begin{forest}
[{H[\comps \ibox{1}]}
  [{H[\comps  \sliste{ \ibox{2} } $\oplus$ \ibox{1}  ]}]
  [\ibox{2}]]
\end{forest}}

\pause
\item If the \spr value is not constrained, it can have arbitrary values.

\pause
\item For example, a list with two genitive NPs and one accusative NP.\\
With such a list we could analyze (\mex{1}):

\ea[*]{
his his him gave the child a book
}
\z
\pause
\item \spr and \comps values have to be provided at the mother nodes:

\ea\label{schema-head-spr-and-head-comps}
Specifier-Head Schema and Head-Complement Schema (final)
\begin{tabular}[t]{@{}l@{~}l@{ }l@{}}
a. & H[\spr \ibox{1}, \comps \alert{\ibox{2}}] & $\to$ H[\spr \ibox{1} $\oplus$ \sliste{ \ibox{3} }, \comps \alert{\ibox{2}} \eliste]\hspace{1em}\ibox{3}  \\
b. & H[\spr \alert{\ibox{1}}, \comps \ibox{2}] & $\to$ H[\spr \alert{\ibox{1}}, \comps \ibox{2} $\oplus$ \sliste{ \ibox{3} }]\hspace{1em}\ibox{3} \\
\end{tabular}
\z

\end{itemize}
}

\frame{
\frametitle{Final versions of the Schemata}

\begin{itemize}
\item Specifier-Head-Schema and Head-Complement-Schema:
\vfill
\hfill
\begin{forest}
[H\feattab{\spr \ibox{1},\\
           \comps \ibox{2} }
  [\ibox{3}]
  [H\feattab{\spr \ibox{1} $\oplus$ \sliste{ \ibox{3} },\\
              \comps \ibox{2} \eliste}]]
\end{forest}
\hfill
\begin{forest}
[H\feattab{\spr \ibox{1},\\
           \comps \ibox{2}}
  [H\feattab{\spr \ibox{1},\\
             \comps  \sliste{ \ibox{3} } $\oplus$ \ibox{2}  ]}]
  [\ibox{3}]]
\end{forest}
\hfill\mbox{}
\vfill
\end{itemize}

}


\subsection{Scrambling and free VO/OV order}

\frame{

\frametitle{Scrambling}

\begin{itemize}
\item Until now one element of the beginning or the end of the \comps list is combined with the head.
\pause
\item Works for English, but no explanation for scrambling.

\begin{figure}
\begin{forest}
[{H[\comps \rot{\ibox{1}} $\oplus$ \blau{\ibox{2}}]}
  [\gruen{\ibox{3}}]
  [{H[\comps  \rot{\ibox{1}} $\oplus$ \sliste{ \gruen{\ibox{3}} } $\oplus$ \blau{\ibox{2}}  ]}]]
\end{forest}
\end{figure}

\item Length of \ibox{1} and \ibox{2} is not constrained. For a ditransitive verb:

\eal
\ex \rot{\eliste{}} $\oplus$ \sliste{ \gruen{NP[\type{nom}]} } $\oplus$ \blau{\sliste{ NP[\type{dat}], NP[\type{acc}] }}
\pause
\ex \rot{\sliste{ NP[\type{nom}] }} $\oplus$ \sliste{ \gruen{NP[\type{dat}]} } $\oplus$ \blau{\sliste{ NP[\type{acc}] }} 
\pause
\ex \rot{\sliste{ NP[\type{nom}], NP[\type{dat}] }} $\oplus$ \sliste{ \gruen{NP[\type{acc}]} } $\oplus$ \blau{\eliste} 
\zl
%So \ibox{3} in Figure~\ref{fig-head-comp-free} would be \npnom in (\mex{0}a), \npdat in (\mex{0}b) and \npacc in (\mex{0}c).

\pause
\item \ibox{1} = empty List $\to$ VO language with fixed order like English
\pause
\item \ibox{2} = empty List $\to$ OV language with fixed order
\pause
\item no restriction for \ibox{1} and \ibox{2} $\to$ free order of arguments

\end{itemize}

}

\subsection{Linearization rules}

\frame{
\frametitle{Linearization rules}

\begin{itemize}
\item The schemata are very abstract. Like \xbar rules.
\pause
\item But the order of the daughters is not fixed.\\
      a can be placed before b or b before a in a schema like (\mex{1}):
\ea
m $\to$ a b
\z
\pause
\item Head-Complement-Schema can have both orders:\\
      head before complement and complement before head

\eal
\ex
%\gll 
dem Kind  ein Buch gibt\\
%     the child the book gives\\
\ex gives the child the book
\zl

\pause
\item daughters without constraints do unwanted things:


\eal
\ex[]{
\gll \dass{} niemand dem Kind  ein Buch vorliest\\
     \that{} nobody  the child a book \partic.reads\\
\glt `that nobody reads a book to the child'
}
\ex[*]{ 
\gll \dass{} dem Kind  niemand vorliest ein Buch\\
     \that{} the child nobody  \partic.reads a book\\
}
\ex[*]{
\gll \dass{} niemand vorliest dem Kind   ein Buch\\
     \that{} nobody  \partic.reads the child a book\\
}
\zl
\end{itemize}


}

\frame{
\frametitle{Unwanted structure}

% \begin{forest}
% sm edges
% [{V[\comps \sliste{ }]},s sep+=1em
%   [{V[\comps \sliste{ \ibox{1} }]}
%     [\ibox{2} \npdat [dem Kind;the child,roof]]
%     [{V[\comps \sliste{ \ibox{2}, \ibox{1} }]} [\ibox{3} \npnom [niemand;nobody]]
%        [{V[\comps \sliste{ \ibox{3}, \ibox{2}, \ibox{1} }]}  [vorliest;\textsc{part}.reads]]]]
%   [\ibox{1} \npacc [ein Buch;a book,roof]]]]
% \end{forest}

% \begin{forest}
% sm edges
% [{V[\comps \sliste{ }]}
%   [{V[\comps \sliste{ \ibox{1} }]},s sep+=1em
%     [{V[\comps \sliste{ \ibox{2}, \ibox{1} }]} [\ibox{3} \npnom [niemand;nobody]]
%        [{V[\comps \sliste{ \ibox{3}, \ibox{2}, \ibox{1} }]}  [vorliest;\textsc{part}.reads]]]
%     [\ibox{2} \npdat [dem Kind;the child,roof]]] 
%   [\ibox{1} \npacc [ein Buch;a book,roof]]]]
% \end{forest}
\vfill
\centerline{%
\begin{forest}
sm edges
[{V[\comps \sliste{ }]}
  [{V[\comps \sliste{ \ibox{1} }]},s sep+=1em
    [{V[\comps \sliste{ \ibox{2}, \ibox{1} }]} [\ibox{3} \npnom [niemand;nobody]]
       [{V[\comps \sliste{ \ibox{3}, \ibox{2}, \ibox{1} }]}  [vorliest;\textsc{part}.reads]]]
    [\ibox{2} \npdat [dem Kind;the child,roof]]] 
  [\ibox{1} \npacc [ein Buch;a book,roof]]]
\end{forest}}
\vfill

}

\frame{
\frametitle{Linearization rules for head and complement}

\begin{itemize}
\item rules:
\eal
\label{lp-regeln}
\ex HEAD [\textsc{initial}+] $<$ COMPLEMENT
\ex COMPLEMENT $<$  HEAD [\textsc{initial}$-$]
\zl
\item German verbs (SOV): \textsc{initial}$-$\\
      English verbs (SVO): \textsc{initial}$+$

\pause
\item German and English nouns: \textsc{initial}$+$

\end{itemize}

}


\frame{
  \frametitle{Exercises -- I}


\begin{enumerate}
\item Please, provide valence lists for the following words:
\eal
\ex laugh
\ex eat
\ex to douse
\ex 
\gll bezichtigen\\
     accuse\\\german
\ex he
\ex the
\ex 
\gll Ankunft\\
     arrival\\\german
\zl
If you are uncertain as far as case assignment is concerned,\\
you may use the Wiktionary: \url{https://de.wiktionary.org/}.
\end{enumerate}


} 

\frame{
\frametitle{Exercises -- II}

\smallexamples
\begin{enumerate}
\setcounter{enumi}{1}
\item Draw the trees for the following examples. NPs can be abbreviated.
\eal
\ex
\gll weil    Aicke dem        Kind  ein      Buch schenkt\\
     because Aicke the.\DAT{} child a.\ACC{} book gives.as.a.present\\\hfill(German)%\german 
\glt `because Aicke gives the child a book as a present'
\ex because Kim gave a book to him
\ex Sandy saw this yesterday.
\ex
\gll at Bjarne læste bogen\\
     that Bjarne read book.\textsc{def}\\\hfill(\ili{Danish})
\glt `that Bjarne read the book'
\zl
\end{enumerate}


} 


\subsection{Adjuncts}

\frame{
\frametitle{Adjuncts}

\begin{itemize}
\item Arguments are selected by their head.
\pause
\item Adjuncts select their head.
\pause
\item Dutch, German, \ldots:\\
      Adjuncts in sentences attach to any verbal projection (verb in final position).
\pause
\item English, Danish, \ldots: Adjuncts attach to VP.

\eal
\ex that everybody \gruen{reads the book} \rot{promptly}
\ex that everybody \rot{promptly} \gruen{reads the book}
\zl

\end{itemize}

}

\frame{
\frametitle{The \textsc{mod} feature}

\begin{itemize}
\item \textsc{mod} is parallel to \spr and \comps:\\

\ea
\begin{tabular}[t]{@{}l@{}}
Lexical item for \emph{brown}:\\
\ms{
  phon & \phonliste{ brown }\\
  mod  & \nbar\\
  spr  & \eliste\\
  comps & \eliste }
\end{tabular}
\hfill\begin{forest}
sm edges
[{\nbar}, baseline
  [{Adj[\textsc{mod} \ibox{2}]} [brown]]
  [{\ibox{2} \nbar}\hphantom{~\ibox{2}} [squirrel]]]
\end{forest}\hfill\mbox{}
\z

\pause
\item The \textsc{mod} value is a description or \type{none}.



\end{itemize}


}

\frame{
\frametitle{The Head-Adjunct-Schema}

\centerline{%
\begin{forest}
[{H[\spr \ibox{1}, \comps \ibox{2}]}
  [{[\textsc{mod} \ibox{3}, \spr \eliste, \comps \eliste]}]
  [{\ibox{3} H[\spr \ibox{1}, \comps  \ibox{2}]}]]
\end{forest}}

\pause
\begin{itemize}
\item \modf like \spr and \comps feature.\\
      Value of  \textsc{mod} is a description of the head that can be modified:
\begin{itemize}
\item German: \textsc{mod} V[\textsc{ini}$-$]
\item English: \textsc{mod} VP
\end{itemize}
\end{itemize}


}

\frame{
\frametitle{Free placement of adjuncts in German -- I}

\vfill
\centerfit{
\begin{forest}
sm edges
[{V[\spr \eliste, \comps \eliste]}, schema
        [{Adv[\textsc{mod} \ibox{3} V]} [morgen;tomorrow] ]
        [{\ibox{3} V[\spr \eliste, \comps \eliste]}
          [{\ibox{1} NP[\type{nom}]} [Aicke;Aicke] ]
          [V\feattab{
              \spr \sliste{ }, \comps \sliste{ \ibox{1} } }
            [{\ibox{2} NP[\type{acc}]} [das Buch;the book, roof] ] 
            [V\feattab{
              \spr \sliste{  },\\
              \comps \sliste{ \ibox{1}, \ibox{2} }} [liest;reads] ] ]
] ]
\end{forest}}


%\centerline{{}[dass] morgen Aicke das Buch liest}

\vfill

}


\frame{
\frametitle{Free placement of adjuncts in German -- II}

\vfill


\centerfit{%
\begin{forest}
sm edges
[{V[\spr \eliste, \comps \eliste]},s sep+=1.5em
          [{\ibox{1} NP[\type{nom}]} [Aicke;Aicke] ]
          [V\feattab{
              \spr \sliste{ }, \comps \sliste{ \ibox{1} } }, schema
            [{Adv[\textsc{mod} \ibox{3} V]} [morgen;tomorrow] ]
            [\ibox{3} V\feattab{
                \spr \sliste{ }, \comps \sliste{ \ibox{1} } }
              [{\ibox{2} NP[\type{acc}]} [das Buch;the book, roof] ] 
              [V\feattab{
                \spr \sliste{  },\\
                \comps \sliste{ \ibox{1}, \ibox{2} }} [liest;reads] ] ]
] ]
\end{forest}}

%\centerline{{}[dass] Aicke morgen das Buch liest}
\vfill


}



\frame{
\frametitle{Free placement of adjuncts in German -- III}

\vfill

\centerfit{%
\begin{forest}
sm edges
[{V[\spr \eliste, \comps \eliste]}
    [{\ibox{1} NP[\type{nom}]} [Aicke;Aicke] ]
      [V\feattab{
         \spr \sliste{ }, \comps \sliste{ \ibox{1} } }, s sep+=1em
         [{\ibox{2} NP[\type{acc}]} [das Buch;the book, roof] ] 
           [V\feattab{
              \spr \sliste{  },\\
              \comps \sliste{ \ibox{1}, \ibox{2} }}, schema 
             [{Adv[\textsc{mod} \ibox{3} V]} [morgen;tomorrow] ]
             [\ibox{3} V\feattab{
                 \spr \sliste{  },\\
                 \comps \sliste{ \ibox{1}, \ibox{2} }} [liest;reads] ] ] ] ]
\end{forest}}

%\centerline{{}[dass] Aicke das Buch morgen liest}

\vfill

}

\frame{
\frametitle{Fixed position of adjuncts in English -- I}


\vfill

\centerfit{%
\begin{forest}
sm edges
[{V[\spr \eliste, \comps \eliste]}, s sep+=1.5em % puts more space between the NP[nom] and the VP,
                                    % otherwise the box would overlap
          [{\ibox{1} NP[\type{nom}]} [Kim] ]
          [V\feattab{
              \spr \sliste{ \ibox{1} }, \comps \sliste{  } }, schema
            [{Adv[\textsc{mod} \ibox{3}]} [often] ]
            [\ibox{3} V\feattab{
                \spr \sliste{ \ibox{1} }, \comps \sliste{  } }
              [V\feattab{
                \spr \sliste{ \ibox{1} },\\
                \comps \sliste{  \ibox{2} }} [reads] ]
              [{\ibox{2} NP[\type{acc}]} [books] ] ]
] ]
\end{forest}}

\vfill

}

\frame{
\frametitle{Fixed position of adjuncts in English -- II}

\vfill

\centerfit{%
\begin{forest}
sm edges
[{V[\spr \eliste, \comps \eliste]},s sep+=1em
          [{\ibox{1} NP[\type{nom}]} [Kim] ]
          [V\feattab{
              \spr \sliste{ \ibox{1} }, \comps \sliste{  } }, schema
            [\ibox{3} V\feattab{
                \spr \sliste{ \ibox{1} }, \comps \sliste{  } }
              [V\feattab{
                \spr \sliste{ \ibox{1} },\\
                \comps \sliste{ \ibox{2} }} [reads] ]
              [{\ibox{2} NP[\type{acc}]} [books] ] 
               ]
            [{Adv[\textsc{mod} \ibox{3}]} [often] ]
] ]
\end{forest}}

\vfill


}


\frame{
\frametitle{Details}

\begin{itemize}
\item Adjuncts do not change the valence/saturation of a projection.\\
(Whether I do put gummy bears into the trolley does not change the shopping list.)

\vfill
\centerline{%
\begin{forest}
[{H[\spr \rot<1>{\ibox{1}}, \comps \gruen<1>{\ibox{2}}]}
  [{[\textsc{mod} \ibox{3},  \gruen<3>{\spr \eliste}, \gruen<2>{\comps \eliste}]}]
  [{\ibox{3} H[\spr \rot<1>{\ibox{1}}, \comps  \gruen<1>{\ibox{2}}]}]]
\end{forest}}

\vfill

\pause
\item Adjuncts must be complete. Otherwise:
\ea[*]{
Sandy read the book in {\lightgray the closet}.
}
\z

\pause
\eal
\ex[]{
\gll dass Aicke eine Stunde liest\\
     that Aicke an   hour reads\\\german
\glt `Aicke is reading for an hour.'
}
\ex[*]{
\gll dass Aicke Stunde liest\\
     that Aicke hour reads\\
\glt `Aicke is reading for an hour.'
}
\zl

\end{itemize}

}

\subsection{Linking}

\frame[shrink]{
\frametitle{Linking}

\begin{itemize}
\item All languages covered here have a list with valence information:
\ea
\sliste{ NP, NP, NP }
\z
\pause
\item It is called Argument Structure (\argst). 
%\spr und \comps werden davon abgeleitet.
\pause
\item The case values differ (we come back to this later.)
\pause
\item The order is parallel.
\eal
\ex 
\gll dass das Kind dem Eichhörnchen die Nuss gibt\\
     that the child  the squirrel    the nut gives\\
\glt `that the child gives the squirrel the nut'
\ex that the child gives the squirrel the nut
\zl
\pause
\item Linking between syntax and semantics is the same for all Germanic languages:

\ea
Lexical entry for \emph{gives}/\emph{gibt}:\\*
\scalebox{.9}{
\ms{
arg-st & \sliste{ NP\ind{1}, NP\ind{2}, NP\ind{3} }\\[2mm]
cont   & \ms[give]{
          agens & \ibox{1}\\
          goal  & \ibox{2}\\
          trans-obj & \ibox{3}\\
        }\\
}}
\z
\end{itemize}


}


%      <!-- Local IspellDict: en_US-w_accents -->
