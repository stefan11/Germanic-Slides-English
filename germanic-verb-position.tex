%% -*- coding:utf-8 -*-


%\subtitle{Verbstellung: Verberst- und Verbzweitstellung in den V2-Sprachen}
\subtitle{Verb position: Verb-first and verb-second in V2 languages}

%\section{Verbstellung: Verberst- und Verbzweitstellung in den V2-Sprachen}
\section{Verb position: Verb-first and verb-second in V2 languages}

\huberlintitlepage[22pt]


\frame{
%\frametitle{Literaturhinweis}
\frametitle{References}

%Zu diesem Abschnitt gibt es das Kapitel~6 in \citew{MuellerGermanic}.
%
%Müller, Stefan, \citeyear{MuellerGermanic}. \emph{Germanic Syntax}. Berlin: Language Science
%Press. In Vorbereitung. 

This section is covered in chapter~6 in \citew{MuellerGermanic}.

Müller, Stefan, \citeyear{MuellerGermanic}. \emph{Germanic Syntax}. Berlin: Language Science
Press. In Vorbereitung. 


}

%\subsection{Verberststellung}
\subsection{Verb-first position}

\frame{
%\frametitle{Lehrmeinung: Deutsch SPO}
\frametitle{Claim: German SPO}

\begin{itemize}
%\item Behauptung: Deutsch ist Subjekt Prädikat Objekt
\item Claim: German is subject predicate object
\pause
%\item Das ist das häufigste Muster,\\
%      wenn man nur Aussagesätze mit Subjekt, Prädikat und Objekt ansieht.
\item This is the most common pattern,\\
if you only look at declarative sentences with subject, predicate and object.
\pause

%\item Es gilt aber schon nicht mehr für psychologische Prädikate:
\item However, it already does not apply to psychological predicates:
\eal 
\gll Dem Mann gefallen die Bilder. \\
     the.\DAT{} man please the.\NOM{} pictures \\
\glt `The man likes the pictures.'
\zl

\pause
%\item Es gilt nicht für freien Text, in dem insbesondere Adverbialien vorkommen,\\ die die erste Stelle
%  im Satz einnehmen können.
\item It does not apply to free text, in which in particular adverbials occur,\\ 
	which can occupy the first position in the sentence.

\pause
%\item Deutsch ist eine SOV-Sprache und außerdem noch eine Verbzweitsprache (V2).
\item German is an SOV language and also a verb second language (V2).
\pause
%\item V2-Sprachen:\\
%Beliebige Konstituenten können vor das finite Verb gestellt werden.
\item V2 languages:\\
Any constituents can be placed in front of the finite verb.

%Alle germanischen Sprachen außer Englisch.
All Germanic languages except English.
\end{itemize}

}

\frame{
%\frametitle{Lehrmeinung: Deutsch SPO, nachgezählt}
\frametitle{Claim: German SPO, recounted}
\small

taz, 01.02.2013:

\rotit{Die Linke} fordert in dem Entwurf auch eine Vermögensteuer von fünf Prozent auf Privatvermögen
ab einer Million Euro, eine stärkere Besteuerung von Erbschaften und eine einmalige Vermögensabgabe
für Reiche. \gruenbf{Ab Jahreseinkommen von 65.000 Euro} soll ein Spitzensteuersatz von 53 Prozent gelten, das
Ehegattensplitting abgeschafft werden.

\rotit{SPD-Fraktionsvize Joachim Poß} kritisierte die Pläne als "`jenseits aller Vernunft und
Realitätstauglichkeit"'. \gruenbf{Mit solchen Vorschlägen} werde das wichtige Thema der Steuergerechtigkeit
diskreditiert. \gruenbf{Zwar} sei es notwendig, Spitzenverdiener stärker an der Finanzierung wichtiger
Zukunftsaufgaben zu beteiligen, "`aber mit Augenmaß und Vernunft"'. \gruenbf{Für eine Begrenzung von
Managergehältern} setzt sich auch die SPD ein.

\alt<beamer>{red}{kursiv} = subject = 2, \alt<beamer>{green}{fett} = non-subject = 4

%natürlich nicht repräsentativ \ldots
of course not representative \ldots

}

\frame[shrink=40]{
\frametitle{A9 soll Teststrecke werden}

taz: 27.01.2015

\gruenbf{Für selbstfahrende Autos} soll es in Deutschland nach Angaben von Bundesverkehrsminister
Alexander Dobrindt (CSU) bald eine Teststrecke geben. \gruenbf{Auf der Autobahn A9 in Bayern} \rotit{sei ein Pilotprojekt „Digitales
TestfeldAutobahn“ geplant}, wie aus einem Papier des Bundesverkehrsministeriums
hervorgeht. \gruenbf{Mit den ersten Maßnahmen für diese Teststrecke} solle schon in diesem Jahr begonnen
werden. \gruenbf{Mit dem Projekt} soll die Effizienz von Autobahnen generell
gesteigert werden. \gruenbf{„\rotit{Die Teststrecke} soll so digitalisiert und technisch ausgerüstet
werden, dass es dort zusätzliche Angebote der Kommunikation zwischen Straße und Fahrzeug
wie auch von Fahrzeug zu Fahrzeug geben wird“}, sagte Dobrindt zur Frankfurter Allgemeinen Zeitung.
\gruenbf{Auf der A9} sollten sowohl Autos mit Assistenzsystemen als
auch später vollautomatisierte Fahrzeuge fahren können. \gruenbf{Dort}
soll die Kommunikation nicht nur zwischen Testfahrzeugen,
sondern auch zwischen Sensoren an der Straße und den Autos
möglich sein, etwa zur Übermittlung von Daten zur Verkehrslage
oder zum Wetter. \gruenbf{\rotit{Das Vorhaben}
solle im Verkehrsministerium von einem runden Tisch mit Forschern
und Industrievertretern begleitet werden,} sagte Dobrindt. \rotit{Dieser} solle sich unter anderem
auch mit den komplizierten Haftungsfragen beschäftigen.
Also: \rotit{Wer} zahlt eigentlich, wenn ein automatisiertes Auto
einen Unfall baut?
\gruenbf{[\gruenbf{Mithilfe der Teststrecke}] solle
die deutsche Automobilindustrie auch beim digitalen Auto
„Weltspitze sein können“,} sagte der CSU-Minister. \rotit{Die deutschen
Hersteller} sollten die Entwicklung nicht Konzernen wie etwa
Google überlassen. \gruenbf{Derzeit} ist Deutschland noch
an das „Wiener Übereinkommen
für den Straßenverkehr“ gebunden,
das Autofahren ohne Fahrer
nicht zulässt. \gruenbf{Nur unter besonderen
Auflagen} sind Tests möglich.
\rotit{Die Grünen} halten die Pläne für
unnütz. \rotit{Grünen-Verkehrsexpertin
Valerie Wilms} sagte der Saarbrücker
Zeitung: „\rotit{Der Minister}
hat wichtigere Dinge zu erledigen,
als sich mit selbstfahrenden
Autos zu beschäftigen.“ \rotit{Die Technologie}
sei im Verkehrsbereich
nicht vordringlich, \gruenbf{auch} stehe sie
noch ganz am Anfang.
\gruenbf{Aus dem grün-rot regierten
Baden-Württemberg – mit dem
Konzernsitz von Daimler –} kamen
hingegen andere Töne. \gruenbf{\rotit{Was
in Bayern funktioniere,} müsse
auch in Baden-Württemberg
möglich sein,} sagte Wirtschaftsminister
Nils Schmid (SPD). \gruenbf{Von
den topografischen Gegebenheiten}
biete sich die Autobahn A81
an.

\alt<beamer>{red}{kursiv} = subject = 11, \alt<beamer>{green}{fett} = non-subject = 16

%natürlich nicht repräsentativ \ldots
of course not representative \ldots

}

\frame{
%\frametitle{Subjekte in Korpora}
\frametitle{Subjects in corpora}

\begin{itemize}
%\item \citet{HK2005a} 38.342 und 22.087 Bäume aus TüBa-D/S und Z
\item \citet{HK2005a} 38.342 and 22.087 trees from TüBa-D/S und Z
\pause
%\item gesprochene und geschriebene Sprache (\verbmobil und taz)
\item Spoken and written language (\verbmobil and taz)
\pause
%\item 50,3\,\% und 52,1\,\% der Sätze enthielten das Subjekt im Vorfeld. 
\item 50.3\.\% and 52.1\.\% of the sentences: subject in the Vorfeld (before the finite
  verb).
\pause
%\item Annahme von SVO-Stellung würde also auch nicht helfen, denn man müsste erklären, wie das
%  Subjekt nachgestellt und etwas anderes vorangestellt wird.
\item assumption of SVO position would therefore not help either, because one would have to explain how the
subject is placed after and something else is placed before it.
\end{itemize}


}


\frame{
%\frametitlefit{Motivation der Verbletztstellung als Grundstellung: Partikeln}
\frametitlefit{Motivation of the final position as basic position: particles}

\citew%[S.\,34--36]
{Bierwisch63a}: %Sogenannte Verbzusätze oder Verbpartikel\\
%bilden mit dem Verb eine enge Einheit.
Verb particles form a close unit with the verb.

\eal
%\ex weil er morgen \alert{anfängt}
\ex 
\gll weil er morgen \alert{anfängt}\\
		because he tomorrow at.catches\\
\glt `because he starts tomorrow'
%\ex Er \alert{fängt} morgen \alert{an}.
\ex 
\gll Er \alert{fängt} morgen \alert{an}.\\
he catches tomorrow at\\
\glt `He starts tomorrow.'
\zl

%Diese Einheit ist nur in der Verbletzstellung zu sehen, was dafür spricht,\\
%diese Stellung als Grundstellung anzusehen.
This unit can only be seen in verb-last position, which suggests that \\
this position should be viewed as the basic position.
}

\frame{
%\frametitle{Stellung von Idiomen}
\frametitle{Position of idioms}

\eal
\judgewidth{?*}
%\ex[]{
%dass niemand dem Mann \alert{den Garaus macht}
\ex[]{
	\gll dass niemand dem Mann den Garaus macht\\
	that nobody  the man  the \textsc{garaus} makes\\
	\glt `that nobody kills the man'
}
%\ex[?*]{
%dass dem Mann \alert{den Garaus} niemand \alert{macht}
\ex[?*]{
	\gll dass dem Mann den Garaus niemand macht\\
	that the man  the \textsc{garaus} nobody makes\\
}
%\ex[]{
%Niemand \alert{macht} ihm \alert{den Garaus}.
\ex[]{
	\gll Niemand macht ihm den Garaus.\\
	nobody makes him the \textsc{garaus}\\
	\glt `Nobody kills him.'
}
\zl

%Idiomteile wollen nebeneinader stehen (\mex{0}a,b).
%
%Umstellung des Verbs ist abgeleitete Stellung. Nur zur Markierung des Satztyps.

Idiom parts want to be next to each other (\mex{0}a,b).

Verb in initial position is derived. Only to mark the sentence type.
}

\frame{
%\frametitle{Stellung in Nebensätzen}
\frametitle{Position in subordinate clauses}

%Verben in infiniten Nebensätzen und in durch eine Konjunktion eingeleiteten
%finiten Nebensätzen stehen immer am Ende\\
%(von Ausklammerungen ins Nachfeld abgesehen):
Verbs in infinite subordinate clauses and in
finite subordinate clauses introduced by a conjunction are always placed at the end\\
(apart from parentheses in the subordinate clause):
\eal
%\ex Der Clown versucht, Kurt-Martin die Ware \alert{zu geben}.
\ex 
\gll Der Clown versucht, Kurt-Martin die Ware zu geben.\\
		the clown tries     Kurt-Martin the goods to give\\
\glt `The clown tries to give Kurt-Martin the goods.'
%\ex dass der Clown Kurt-Martin die Ware \alert{gibt}
\ex 
\gll dass der Clown Kurt-Martin die Ware gibt\\
		that the clown Kurt-Martin the goods gives\\
\glt `that the clown gives Kurt-Martin the goods'
\zl
}

\frame{
%\frametitle{Stellung der Verben in SVO und SOV-Sprachen}
\frametitle{Position of verbs in SVO and SOV languages}

\citet{Oersnes2009b}: 
\eal
%\ex dass er ihn gesehen$_3$ haben$_2$ muss$_1$
\ex
\gll dass er ihn gesehen$_3$ haben$_2$ muss$_1$\\
     that he him seen        have      must\\\german
\glt `that he must have seen him'
\ex 
\gll at han må$_1$ have$_2$ set$_3$ ham\\
     that he must have seen him\\
\zl
\pause

%Nur das finite Verb wird umgestellt, die anderen Verben bleiben hinten:
Only the finite verb is moved, the other verbs remain behind:
\eal
%\ex Muss er ihn gesehen haben?
\ex 
\gll Muss er ihn gesehen haben?\\
     must he him seen have\\%\german
\glt `Must he have seen him?'
\ex 
\gll Må han have set ham?\\
     must he have seen him\\
\glt `Must he have seen him?'
\zl


}

\frame%[shrink]
{
%	\frametitle{Skopus}
\frametitle{Scope}

\citew[Abschnitt~2.3]{Netter92}:
%Skopusbeziehungen der Adverbien hängt von ihrerer Reihenfolge ab (Präferenzregel?):\\
%Links stehendes Adverb hat Skopus über folgendes Adverb und Verb.
Scope of adverbs depend on their order (preference rule?):\\
Adverb on the left scope over the following adverb and verb.

\eal
%\ex weil er [absichtlich [nicht lacht]]
\ex 
\gll weil er  [absichtlich [nicht lacht]]\\
because he \hphantom{[}deliberately \hphantom{[}not laughs\\
\glt `because he deliberately does not laugh'
%\ex weil er [nicht [absichtlich lacht]]
\ex 
\gll weil er [nicht [absichtlich lacht]]\\
because he \hphantom{[}not \hphantom{[}deliberately laughs\\
\glt `because he does not laugh deliberately'
\zl
}

\frame%[shrink]
{
%	\frametitle{Skopus}
\frametitle{Scope}

%Bei Verberststellung ändern sich die Skopusverhältnisse nicht.
The scope relations do not change when the verb position is changed.
\eal
\ex 
\gll Er lacht absichtlich nicht. \\
		he laughs deliberately not \\
\glt `He deliberately doesn't laugh.'

\ex 
\gll Er lacht nicht absichtlich. \\
		he laughs not deliberately \\
\glt `He doesn't laugh deliberately.'
\zl

\pause
%Analyse:
Analysis:

\eal
\ex 
\gll Er lacht$_i$ [absichtlich [nicht \_$_i$]].\\
		he laughs$_i$ ~not ~deliberately \_$_i$ \\
\ex 
\gll Er lacht$_i$  [nicht [absichtlich \_$_i$]]. \\
		he laughs$_i$ ~not ~deliberately \_$_i$ \\

\zl

%Struktur ist in (\mex{0}) und (\mex{-2}) genau gleich.
Structure is exactly the same in (\mex{0}) and (\mex{-2}).

}

\frame{
%\frametitle{Mitunter nur SOV-Stellung möglich}
\frametitle{Sometimes only SOV position possible}


\citet{Haider97c}, \citet{Meinunger2001a}: %Manche Verben lassen in Verbindung mit \emph{mehr als} nur Verbletztstellung zu:
Some verbs in combination with \emph{mehr als} only allow verb last position:

\eal
%\ex[]{
%dass Hans seinen Profit letztes Jahr \alert{mehr als verdreifachte}
\ex[]{
\gll dass Hans seinen Profit letztes Jahr \alert{mehr} \alert{als} \alert{verdreifachte}\\
	that Hans his         profit last       year more than tripled\\
\glt `that Hans increased his profit last year by a factor greater than three'
}
%\ex[]{
%Hans hat seinen Profit letztes Jahr \alert{mehr als verdreifacht}.
\ex[]{
\gll Hans hat seinen Profit letztes Jahr \alert{mehr} \alert{als} \alert{verdreifacht}.\\
	Hans has his    profit last    year more than tripled\\
\glt `Hans increased his profit last year by a factor greater than three.'
}
%\ex[*]{
%Hans \alert{verdreifachte} seinen Profit letztes Jahr \alert{mehr als}.
\ex[*]{
	\gll Hans \alert{verdreifachte} seinen Profit letztes Jahr \alert{mehr} \alert{als}.\\
	Hans tripled       his    profit last year more than\\
}
\zl

}

\frame{
	%\frametitle{Mitunter nur SOV-Stellung möglich}
	\frametitle{Sometimes only SOV position possible}

\citet{Hoehle91b}, \citet[\page 62]{Haider93a}: %Über Rückbildung entstandene Verben können oft nicht getrennt/umgestellt werden:
Verbs created via backformation often cannot be separated/ rearranged:

\eal
%\ex[]{
%weil sie das Stück heute \alert{uraufführen}%\\
\ex[]{
\gll weil sie das Stück heute \alert{uraufführen}\\
	because they the play today play.for.the.first.time\\
\glt `because they premiere the play today'
%     because they the play today play.for.the.first.time\\
%\glt `because they premiered the play today'
}
%\ex[*]{
%Sie \alert{uraufführen} heute das Stück.%\\
%     they play.for.the.first.time  today the play\\
\ex[*]{
\gll Sie  \alert{uraufführen} heute das Stück.\\
	they play.for.the.first.time  today the play\\
}
%\ex[*]{
%Sie \alert{führen} heute das Stück \alert{urauf}.%\\
%%     they guide today the play  {\sc prefix}.{\sc part}\\
\ex[*]{
\gll Sie \alert{führen} heute das Stück \alert{urauf}.\\
	they guide today the play  \textsc{prefix}.\textsc{part}\\
}
\zl


%Zu einem Überblick siehe \citew{MuellerGermanHandbook}.
For an overview, see \citew{MuellerGermanHandbook}.
}


\frame{
%\frametitle{Dänisch}
\frametitle{Danish}

\begin{itemize}
%\item \rot{Negation} verbindet sich mit der \blau<1>{VP}:
\item \rot{Negation} combines with the \blau<1>{VP}:

\ea
\gll  at   Jens \rot{ikke} [\sub{VP} {\blau<1>{læser}} \blau<1>{bogen}]\\
%      dass Jens nicht      {}        liest          Buch.{\sc def}\\
%\glt `dass Jens das Buch nicht liest'
		that Jens not  {} reads   book.{\sc def}\\
\glt `that Jens does not read the book'
\z

\pause
%\item In V2-Sätzen wird das \blau{finite Verb} links von der \rot{Negation} realisiert:
\item In V2 sentences, the \blau{finite verb} is realized to the left of the \rot{negation}:

\ea
\gll  Jens \blau{læser} \rot{ikke} bogen.\\
%      Jens liest        nicht      Buch.{\sc def}\\
%\glt `Jens liest das Buch nicht.'
		Jens reads   not  book.{\sc def}\\
\glt `Jens is not reading the book.'
\z
\pause
%\item Das wird von vielen als Evidenz für Verbumstellung gesehen:
\item This is seen by many as evidence of verb fronting:

\ea
\gll  Jens \blau{læser}$_i$ \rot{ikke} [\sub{VP} \_$_i$ bogen].\\
%      Jens liest            nicht      {}        {}     Buch.{\sc def}\\
      Jens reads      not  {} {}    book.{\sc def}\\
\z
\end{itemize}
\nocite{KS2002a}
}

\frame{
%\frametitle{Entscheidungsfragen wie im Deutschen V1-Stellung}
\frametitle{Yes/No questions as in the German V1 position}

\eal
\ex
\gll at Jens læser bogen\\
%     dass Jens liest Buch.{\sc def}\\
%\glt `dass Jens das Buch liest'
		that Jens reads book.\textsc{def}\\
\glt \glt `that Jens reads the book'
\ex
\gll Læser Jens bogen?\\
%     liest Jens Buch.{\sc def}\\
%\glt `Liest Jens das Buch?'
		reads Jens book.\textsc{def}\\
\glt `Does Jens read the book?'
\zl

\pause

%Analyse:
Analysis:
\eal
\ex
\gll at Jens [\sub{VP} læser bogen]\\
%     dass Jens {} liest Buch.{\sc def}\\
%\glt `dass Jens das Buch liest'
	 that Jens {} reads book.\textsc{def}\\
\glt `that Jens reads the book'
\ex
\gll Læser$_i$ Jens [\sub{VP} \_$_i$ bogen]?\\
%     liest Jens {}        {}     Buch.{\sc def}\\
%\glt `Liest Jens das Buch?'
     reads     Jens {}        {}     book.\textsc{def}\\
\glt `Does Jens read the book?'
\zl


}


\frame{
%\frametitle{Verbumstellung im Deutschen als Informationsweitergabe}
\frametitle{Verb movement in German as information transfer}

%% \begin{tikzpicture}
%% \tikzset{level 1+/.style={level distance=3\baselineskip}}
%% \tikzset{frontier/.style={distance from root=12\baselineskip}}
%% %\draw (-3,-5) to[grid with coordinates] (4,0);
%% \Tree[.S
%%         [.{V \sliste{ S/\!/V }} 
%%           [.V liest$_j$ ] ]
%%         [.{S$/\!/$V}
%%            [.NP Jens ]
%%            [.{V$'$$\!/\!/$V}
%%              [.NP \edge[roof]; {das Buch} ]
%%              [.{V$\!/\!/$V} \_$_j$ ] ] ] ]
%% \draw[semithick,<->,color=green] (3.1,-3.9) ..controls +(south east:.5) and +(south west:.5)..(2.7,-3.9);
%% \draw[semithick,<->,color=green] (3.5,-3.7) ..controls +(east:.5) and +(east:.5)..(2.8,-2.5);
%% \draw[semithick,<->,color=green] (2.8,-2.3) ..controls +(east:.5) and +(east:.5)..(1.7,-1.1);
%% \draw[semithick,<->,color=green] (1.5,-0.9) ..controls +(north:.5) and +(north:.5)..(-0.8,-0.9);
%% \draw[semithick,<->,color=green] (-0.7,-1.1) ..controls +(south east:.2) and +(north east:.5)..(-1.0,-2.4);
%% \end{tikzpicture}}

~\vfill
\centerfit{
\begin{forest}
sm edges
[S
  [{V \sliste{ S$/\!/$V }} 
    [V [liest$_j$;reads] ] ]
       [{S$/\!/$V}
           [NP [Jens;Jens] ]
           [{V$'$$\!/\!/$V}
             [NP [das Buch;the book, roof] ]
             [{\mybox[v1]{V}$\!/\!/$\mybox[v2]{V}} [\_$_j$] ] ] ] ] ]
%\draw[semithick,<->,color=green] (v1.south)--(v2.south);
%% \draw[semithick,<->,color=green] (3.1,-3.9) ..controls +(south east:.5) and +(south west:.5)..(2.7,-3.9);
%% \draw[semithick,<->,color=green] (3.5,-3.7) ..controls +(east:.5) and +(east:.5)..(2.8,-2.5);
%% \draw[semithick,<->,color=green] (2.8,-2.3) ..controls +(east:.5) and +(east:.5)..(1.7,-1.1);
%% \draw[semithick,<->,color=green] (1.5,-0.9) ..controls +(north:.5) and +(north:.5)..(-0.8,-0.9);
%% \draw[semithick,<->,color=green] (-0.7,-1.1) ..controls +(south east:.2) and +(north
       %% east:.5)..(-1.0,-2.4);
\end{forest}
}
\vfill
}


\frame{
%\frametitle{Skopus}
\frametitle{Scope}

\vfill
\centerfit{\begin{forest}
sm edges
[S
        [{V \sliste{ S$/\!/$V }} 
          [V [lacht$_j$;laughs] ] ]
        [{S$/\!/$V}
           [NP [er;he] ]
           [{V$'$$\!/\!/$V}
             [Adv [nicht;not] ]
             [{V$'$$\!/\!/$V}
               [Adv [absichtlich;deliberately] ]
               [{V$\!/\!/$V} [\_$_j$] ] ] ] ] ]
\end{forest}
}

\vfill

}


\frame{
%\frametitle{Verbumstellung im Dänischen}
\frametitle{Verb movement in Danish}

~\vfill
\centerfit{\begin{forest}
sm edges
[S
        [{V \sliste{ S$/\!/$V }} 
          [V [læser$_j$;reads] ] ]
        [{S$/\!/$V}
           [NP [Jens;Jens] ]
           [{VP$\!/\!/$V}
             [{V$\!/\!/$V} [\_$_j$] ] 
             [NP [bogen;book.\textsc{def}] ] ] ] ]
\end{forest}}
\vfill


}


\frame[shrink]{
%\frametitle{Verbumstellung im Dänischen mit Negation}
\frametitle{Verb movement with negation in Danish}

~\vfill
\centerfit{\begin{forest}
sm edges
[S
        [{V \sliste{ S$/\!/$V }} 
          [V [læser$_j$;reads] ] ]
        [{S$/\!/$V}
           [NP [Jens;Jens] ]
           [{VP$\!/\!/$V}
             [Adv [ikke;not] ]
             [{VP$\!/\!/$V}
               [{V$\!/\!/$V} [\_$_j$] ] 
               [NP [bogen;book.\textsc{def}] ] ] ] ] ]
\end{forest}}
\vfill

}



\frame{
%\frametitle{Übungsaufgaben}
\frametitle{Exercises}

\begin{enumerate}
%\item Skizzieren Sie die Analyse für die folgenden Beispiele:
\item Sketch the analysis for the following examples:
\eal
\ex dass er darüber lachen wird
\ex Wird er darüber lachen?
\zl

\ea
\gll Arbejder Bjarne ihærdigt  på bogen.\\
     arbeitet Bjarne ernsthaft an Buch.{\sc def}\\\jambox{(Danish)}
\glt `Arbeitet Bjarne ernsthaft an dem Buch?'
\z

\end{enumerate}

}


%      <!-- Local IspellDict: en_US-w_accents -->
