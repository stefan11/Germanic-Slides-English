%% -*- coding:utf-8 -*-


%\subtitle{Verbalkomplexbildung in den SOV-Sprachen}
\subtitle{Verbal complex in SOV languages}

%\section{Verbalkomplexbildung in den SOV-Sprachen}
\section{Verbal complex in SOV languages}


\huberlintitlepage[22pt]

\outline{

\begin{itemize}
%\item {Überblick über die germanischen Sprachen}
\item {Overview of the Germanic languages}
%\item Phänomene
\item Phenomena 
%\item Phrasenstrukturgrammatiken und \xbart
\item Phrase structure grammars and \xbart
%\item {Valenz, Argumentanordnung und Adjunkte}
\item {Valence, argument order and adjuncts}
%\item \alert{Verbalkomplexbildung in den SOV-Sprachen}
\item \alert{Verbal complex in SOV languages}
%\item Verbstellung: Verberst- und Verbzweitstellung
\item Verb position: Verb-first and Verb-second
%\item Passiv
\item Passive
%\item Eingebettete Sätze
\item Embedded clauses
\end{itemize}

}


\frame{
\frametitle{Reference}



%Zu diesem Abschnitt gibt es das Kapitel~4 in \citew{MuellerGermanic}.
This section is covered in chapter~4 in \citew{MuellerGermanic}.

Müller, Stefan, \citeyear{MuellerGermanic}. \emph{Germanic Syntax}. Berlin: Language Science
Press. In Vorbereitung. 



}


\frame{
%\frametitle{Verbalkomplexbildung}
\frametitle{Verbal complex}
\smallexamples 
\smallframe

\begin{itemize}
%\item Deutsch, Niederländisch erlauben Verbalkomplexbildung:
\item German, Dutch allow verbal complex:

\eal
\gll weil \highlight{es}<1> \highlight{ihr}<2> \highlight{jemand}<3> \highlight{zu} \highlight{lesen}<1> \highlight{versprochen}<2> \highlight{hat}<3> \citep{Haider90b} \\
		because it her somebody to read promised has\\
\glt `because somebody promised her to read it'
\zl

%\pause
%\pause
\pause
%Die Verben am Ende verhalten sich wie ein einfaches Verb. 
The verbs at the end behave like a simple verb.
%Umordnung der Argumente ist möglich.
Reordering the arguments is possible.


\pause
%\item Niederländisch:
\item Dutch:
\eal
\ex
\gll dat Jan het boek wil lezen\\
     dass Jan das Buch will lesen\\
\glt `dass Jan das Buch lesen will'
% that John the book wants read
% 'that John wants to read the book' 
\ex
\gll dat Jan Marie het boek laat lezen\\
     dass Jan Maria das Buch lässt lesen\\
%that John Mary the book lets read
%'that John lets Mary read the book'
\ex 
\gll dat Jan Marie het boek wil laten lezen\\
     dass Jan Marie das Buch will lassen lesen\\
\glt `dass Jan Maria das Buch lesen lassen will'
%that John Mary the book wants let read 'that John wants to let Mary read the book'
\zl

\pause
%\item Englisch, Dänisch, \ldots{} erlauben keine Umordnung von Konstituenten
\item English, Danish, \ldots{} do not allow reordering of constituents

\end{itemize}

}

\frame{
\frametitle{Variation}


\begin{itemize}
%\item Bei den Abfolgen im Verbalkomplex gibt es extreme Variation. 
\item There is extreme variation in the sequences in the verbal complex.

\pause
%\item Standardsprachlich: übergeordnetes Verb steht rechts: V$_3$ V$_2$ V$_1$
\item Standard: superordinate verb is on the right: V$_3$ V$_2$ V$_1$

\eal
\ex weil es ihr jemand zu lesen versprochen hat \citep{Haider90b} \\
		because it her somebody to read promised has\\
\glt `because somebody promised her to read it'
\zl

\end{itemize}

}


\frame{
\frametitle{Variation}
	
	
\begin{itemize}
%\item In den Dialekten gibt es aber die buntesten Abfolgen.
\item In dialects, however, there are the most colorful sequences.

%Z.B. folgende \citep[376]{Mueller99a}:
E.g. the following \citep[376]{Mueller99a}:

\eal
\ex 
\gll Ich hätte stapelweise Akten kön\-nen haben.\\
		I   had   by.the.pile files can      have\\ %\hfill (\ili{German}, Berlin dialect)
\glt `I could have had files by the pile.'

\ex 
\gll weil ich mir das  nich hab' lassen gefallen\\
		because I me that not  have let    please\\
\glt `because I did not put up with it'

\ex 
\gll wenn se   mir hier würden rausschmeißen, \ldots\\
		if   they me  here would  out.throw\\
\glt `if they would kick me out here'

\zl
%(Interviewpartner in:\\
%\emph{Insekten und andere Nachbarn -- ein Haus in Berlin}, ARD 15.11.1995)
(Interview partner: \\
\emph{Insekten und andere Nachbarn -- ein Haus in Berlin}, ARD 15.11.1995)


\end{itemize}


}

%\hfsetfillcolor{green!50!lime!30}
%\hfsetbordercolor{green!40!black}

\frame{
%\frametitle{Argumentanziehung}
\frametitle{Argument attraction}


\begin{forest}
sm edges
[V\feattab{
%              \vform \type{fin},\\
              \highlight<4>{\sliste{ NP[\type{nom}], NP[\type{acc}] }} } 
        [{\highlight<1,3>{V}\feattab{
%              \vform \type{bse},\\
              \sliste{ \highlight<2,4>{NP[\type{nom}], NP[\type{acc}]}} }}, name=lesen [lesen;read] ]
        [V\feattab{
%              \vform \type{fin},\\
              \sliste{ \visible<2->{\highlight<2>{NP[\type{nom}], NP[\type{acc}]},} \highlight<1,3>{V}}}, name=wird [wird;will] ]
]
\only<2>{\draw[semithick,->] (lesen)..controls +(south east:2) and +(south west:2)..(wird);}
\end{forest}

%% \begin{forest}
%% sm edges
%% [V\feattab{
%% %              \vform \type{fin},\\
%%               \sliste{ NP[\type{nom}], NP[\type{acc}] } } 
%%         [\highlight<1>{V}\feattab{
%% %              \vform \type{bse},\\
%%               \sliste{ NP[\type{nom}], NP[\type{acc}]} }, name=lesen [lesen] ]
%%         [V\feattab{
%% %              \vform \type{fin},\\
%%               \sliste{ NP[\type{nom}], NP[\type{acc}], \highlight<1>{V}}}, name=wird [wird] ]
%% ]
%% \draw[semithick,->] (lesen)..controls +(south east:2) and +(south west:2)..(wird);
%% \end{forest}

\begin{itemize}
%\item \emph{wird} verlangt Infinitiv ohne \emph{zu} \visible<2->{und dessen Argumente\\
%      \citep{Geach70a,HN94a}}
\item \emph{wird} requires infinitive without \emph{zu} \visible<2->{and its arguments\\
	\citep{Geach70a,HN94a}}
	
\pause
\pause

%\item Verb wird gesättigt und ist am Mutterknoten nicht mehr in der Valenzliste
\item Verb is saturated and is no longer in the valency list at the parent node

\pause
%\item Kombination aus \emph{lesen} und \emph{wird} verhält sich wie einfaches Verb und kann mit den
%  Argumenten in beliebiger Reihenfolge kombiniert werden.
\item Combination of \emph{lesen} and \emph{wird} behaves like a simple verb and can be combined with the
arguments in any order.

\end{itemize}

}


\frame{
%\frametitle{Verbalkomplexbildung und Scrambling: Normalstellung}
\frametitle{Verbal complex formation and scrambling: normal position}

\centerfit{
\begin{forest}
sm edges
[V\feattab{
              \sliste{ }}
        [{NP[\type{nom}]} [keiner;nobody] ]
        [V\feattab{
              \sliste{ NP[\type{nom}] }}, s sep+=2ex
          [{NP[\type{acc}]} [das Buch;the book, roof] ]
          [V\feattab{
%              \vform \type{fin},\\
              \sliste{ NP[\type{nom}], NP[\type{acc}]}} 
             [V\feattab{
%              \vform \type{bse},\\
              \sliste{ NP[\type{nom}], NP[\type{acc}]}} [lesen;read] ]
             [V\feattab{
%              \vform \type{fin},\\
                \sliste{ NP[\type{nom}], NP[\type{acc}], V }} [wird;will] ] ] ] ]
\end{forest}}


}

\frame{
%\frametitle{Verbalkomplexbildung und Scrambling: Acc $<$ Nom}
\frametitle{Verbal complex formation and scrambling: Acc $<$ Nom}

\centerfit{\begin{forest}
sm edges
[V\feattab{
              \sliste{ }}
        [{NP[\type{acc}]} [das Buch;the book, roof] ]
        [V\feattab{
              \sliste{ NP[\type{acc}] }}
          [{NP[\type{nom}]} [keiner;nobody] ]
          [V\feattab{
%              \vform \type{fin},\\
              \sliste{ NP[\type{nom}], NP[\type{acc}] }} 
             [V\feattab{
%              \vform \type{bse},\\
              \sliste{ NP[\type{nom}], NP[\type{acc}]}} [lesen;read] ]
             [V\feattab{
%              \vform \type{fin},\\
                \sliste{ NP[\type{nom}], NP[\type{acc}], V }} [wird;will] ] ] ] ]
\end{forest}}


}

%\subsection{Argumentanziehung im Detail}
\subsection{Argument attraction in detail}

\frame{
%\frametitle{Subjekte von nicht-finiten Verben}
\frametitle{Subjects of non-finite verbs}

\ea
%\emph{lesen} infinite Form:\\
\emph{lesen} infinite form:\\
\ms{
subj  & \sliste{ NP[\type{nom}] }\\[1mm]
comps & \sliste{ NP[\type{acc}] }\\
}
\z

%Subjekte können nur mit finiten Verben kombiniert werden:
Subjects can only be combined with finite verbs:
\eal
\ex[]{
\gll Kim hat Sandy versprochen, [das Buch zu lesen]. \\
		Kim has Sandy promised     \spacebr{}the book to read\\
\glt `Kim promised Sandy to read the book.'
}

\ex[*]{
\gll Kim hat Sandy versprochen, [sie das Buch zu lesen]. \\
		Kim has Sandy promised     \spacebr{}the book to read\\
\glt Intended: `Kim promised Sandy that she will read the book.'
}
\zl
}

\frame{
	%\frametitle{Subjekte von nicht-finiten Verben}
	\frametitle{Subjects of non-finite verbs}
	
\eal
\judgewidth{?*}
\ex[]{
\gll {}[Das Buch lesen] wird Aicke morgen.\\
     \spacebr{}the book read  will Aicke  tomorrow\\
\glt `Aicke will read the book tomorrow.'
}

\ex[*]{
\gll {}[Aicke lesen] wird das Buch morgen.\\
     \spacebr{}Aicke  read  will the book tomorrow\\
}

\ex[?*]{
\gll {}[Aicke das Buch lesen] wird morgen.\\
     \spacebr{}Aicke  the book read will tomorrow\\
}
\zl

}

\frame{
%\frametitle{Lexikoneintrag für Hilfsverb}
\frametitle{Lexicon entry for auxiliary verb}

\ea
%\emph{werden} infinite Form:\\
\emph{werden} infinite form:\\
\ms{
subj  & \blau<3>{\ibox{1}}\\
comps & \blau<3>{\ibox{2}} $\oplus$ \sliste{ V[\blau<1>{\vform \type{bse}}, \blau<2>{\textsc{lex}+}, 
                                   \subj \blau<3>{\ibox{1}}, \comps \blau<3>{\ibox{2}}] }\\
}
\z

%Das Hilfsverb \emph{werden} verlangt einen Infinitiv ohne zu (\vform \type{bse}).
The auxiliary verb \emph{werden} requires an infinitive without \emph{zu} (\vform \type{bse}).

\pause
%\textsc{lex}+ sorgt dafür, dass das verbale Komplement ein Wort oder Verbalkomplex ist.
\textsc{lex}+ ensures that the verbal complement is a word or verbal complex.

\pause

%Das Subjekt \iboxb{1} und die anderen Argumente \iboxb{2} werden übernommen.
The subject \iboxb{1} and the other arguments \iboxb{2} are adopted.

}

\frame{
%\frametitle{\emph{werden} finite Form}
\frametitle{\emph{werden} finite form}

%Finite Verben haben das Subjekt auf der \compsl.
Finite verbs have the subject on the \compsl.

\ea
%\emph{wird} finite Form:\\
\emph{wird} finite form:\\
\ms{
subj  & \blau{\sliste{}}\\
comps & \blau{\ibox{1}} $\oplus$ \ibox{2} $\oplus$ \sliste{ V[\vform \type{bse}, \textsc{lex}+, \subj \blau{\ibox{1}}, \comps \ibox{2} ] }\\
}
\z


}

\frame{
%\frametitle{Argumentanziehung im Detail}
\frametitle{Argument attraction in detail}

%% \begin{tikzpicture}
%% \tikzset{level 1+/.style={level distance=4\baselineskip}}
%% \tikzset{level 2/.style={level distance=4\baselineskip}}
%% \tikzset{frontier/.style={distance from root=9\baselineskip}}
%% \Tree[.V\feattab{
%%               \vform \type{fin},\\
%%               \comps \highlight{\ibox{1} $\oplus$ \ibox{2}}<3> } 
%%         [.{\highlight{\ibox{3} V}<1>\feattab{
%%               \highlight{\vform \type{bse}}<1>,\\
%%               \subj  \highlight{\ibox{1} \sliste{ NP[\type{nom}] }}<2,3>, \\ 
%%               \comps \highlight{\ibox{2} \sliste{ NP[\type{acc}] }}<2,3> }} lesen ]
%%         [.V\feattab{
%%               \vform \type{fin},\\
%%               \comps \highlight{\ibox{1} $\oplus$ \ibox{2}}<2> $\oplus$ \sliste{ \highlight{\ibox{3}}<1> } } wird ]
%% ]
%% \end{tikzpicture}

%% \begin{tikzpicture}
%% \tikzset{level 1+/.style={level distance=4\baselineskip}}
%% \tikzset{level 2/.style={level distance=4\baselineskip}}
%% \tikzset{frontier/.style={distance from root=9\baselineskip}}
%% \Tree[.V\feattab{
%%               \vform \type{fin},\\
%%               \comps \highlight{\ibox{1} $\oplus$ \ibox{2}}<3> } 
%%         [.{\highlight{\ibox{3} V}<1>\feattab{
%%               \highlight{\vform \type{bse}}<1>,\\
%%               \subj  \highlight{\ibox{1} \sliste{ NP[\type{nom}] }}<2,3>, \\ 
%%               \comps \highlight{\ibox{2} \sliste{ NP[\type{acc}] }}<2,3> }} lesen ]
%%         [.V\feattab{
%%               \vform \type{fin},\\
%%               \comps \highlight{\ibox{1} $\oplus$ \ibox{2}}<2> $\oplus$ \sliste{ \highlight{\ibox{3}}<1> } } wird ]
%% ]
%% \end{tikzpicture}

\centerline{%
\begin{forest}
sm edges
[V\feattab{
              \vform \type{fin},\\
              \comps \blau<3>{\ibox{1}} $\oplus$ \blau<3>{\ibox{2}} } 
        [\blau<1>{\ibox{3} V}\feattab{
              \blau<1>{\vform \type{bse}},\\
              \subj  \blau<2-3>{\ibox{1}} \sliste{ NP[\type{nom}] }, \\ 
              \comps \blau<2-3>{\ibox{2}} \sliste{ NP[\type{acc}] } } [lesen;read] ]
        [V\feattab{
              \vform \type{fin},\\
              \comps \blau<2>{\ibox{1}} $\oplus$ \blau<2>{\ibox{2}} $\oplus$ \sliste{ \blau<1>{\ibox{3}} } } [wird;will] ] ]
\end{forest}
}

\begin{itemize}
%\item Hilfsverb verlangt Infinitiv ohne \emph{zu} \iboxb{3}.
\item Auxiliary verb requires infinitive without \emph{zu} \iboxb{3}.
\pause
%\item Subjekt \iboxb{1} und Komplemente \iboxb{2} werden übernommen.
\item Subject \iboxb{1} and complements \iboxb{2} are adopted.
\pause
%\item \emph{lesen wird} hat dieselben Argumente wie \emph{liest}
\item \emph{lesen wird} has the same arguments as \emph{liest}
\end{itemize}

}


\frame{
%\frametitle{Komplexere Komplexe}
\frametitle{More complex complexes}

%% %\centerfit{%
%% \begin{tikzpicture}
%% \tikzset{level 1+/.style={level distance=4\baselineskip}}
%% \tikzset{level 2+/.style={level distance=5\baselineskip}}
%% \tikzset{frontier/.style={distance from root=14\baselineskip}}
%% \Tree[.V\feattab{
%%               \vform \type{fin},\\
%%               \comps \highlight{\ibox{1} $\oplus$ \ibox{2}}<3> } 
%%         [.{\ibox{4} V\feattab{
%%               \vform \type{bse},\\
%%               \subj  \ibox{1},\\
%%               \comps \highlight{\ibox{2}}<3> }} 
%%            [.{\highlight{\ibox{3} V}<1>\feattab{
%%               \highlight{\vform \type{bse}}<1>,\\
%%               \subj  \highlight{\ibox{1} \sliste{ NP[\type{nom}] }}<2,3>, \\ 
%%               \comps \highlight{\ibox{2} \sliste{ NP[\type{acc}] }}<2,3> }} lesen ]
%%            [.V\feattab{
%%               \vform \type{bse},\\
%%               \subj  \ibox{1},\\
%%               \comps \ibox{2} $\oplus$ \sliste{
%%                 \highlight{\ibox{3}}<1> } } können ] ]
%%         [.V\feattab{
%%               \vform \type{fin},\\
%%               \comps \highlight{\ibox{1} $\oplus$ \ibox{2}}<2> $\oplus$ \sliste{
%%                 \highlight{\ibox{4}}<1> } } wird ] 
%% ]
%% \end{tikzpicture}}



\centerline{\scalebox{0.9}{
\begin{forest}
sm edges
[V\feattab{
              \vform \type{fin},\\
              \comps \ibox{1} $\oplus$ \ibox{2} } 
        [{\ibox{4} V\feattab{
              \vform \type{bse},\\
              \subj  \ibox{1},\\
              \comps \ibox{2} }} 
           [{\ibox{3} V\feattab{
              \vform \type{bse},\\
              \subj  \ibox{1} \sliste{ NP[\type{nom}] }, \\ 
              \comps \ibox{2} \sliste{ NP[\type{acc}] } }} [lesen;read] ]
           [V\feattab{
              \vform \type{bse},\\
              \subj  \ibox{1},\\
              \comps \ibox{2} $\oplus$ \sliste{
                \ibox{3} } } [können;can] ] ]
        [V\feattab{
              \vform \type{fin},\\
              \comps \ibox{1} $\oplus$ \ibox{2} $\oplus$ \sliste{
                \ibox{4} } } [wird;will] ] 
]
\end{forest}}}

%Das geht auch zu dritt, Hauptsache, einer übernimmt die Verantwortung
It also works for three, as long as one takes responsibility.


}


\frame{
%\frametitle{Oder auch mal andersrum}
\frametitle{Or also the other way around}


%\centerline{\scalebox{0.7}{
%% \centerfit{%
%% \begin{tikzpicture}
%% \tikzset{level 1+/.style={level distance=4\baselineskip}}
%% \tikzset{level 2+/.style={level distance=5\baselineskip}}
%% \tikzset{frontier/.style={distance from root=14\baselineskip}}
%% \Tree[.V\feattab{
%%               \vform \type{fin},\\
%%               \comps \highlight{\ibox{1} $\oplus$ \ibox{2}}<3> } 
%%         [.V\feattab{
%%               \vform \type{fin},\\
%%               \comps \highlight{\ibox{1} $\oplus$ \ibox{2}}<2> $\oplus$ \sliste{
%%                 \highlight{\ibox{4}}<1> } } wird ]
%%         [.{\ibox{4} V\feattab{
%%               \vform \type{bse},\\
%%               \subj  \ibox{1},\\
%%               \comps \highlight{\ibox{2}}<3> }} 
%%            [.{\highlight{\ibox{3} V}<1>\feattab{
%%               \highlight{\vform \type{bse}}<1>,\\
%%               \subj  \highlight{\ibox{1} \sliste{ NP[\type{nom}] }}<2,3>, \\ 
%%               \comps \highlight{\ibox{2} \sliste{ NP[\type{acc}] }}<2,3> }} lesen ]
%%            [.V\feattab{
%%               \vform \type{bse},\\
%%               \subj  \ibox{1},\\
%%               \comps \ibox{2} $\oplus$ \sliste{
%%                 \highlight{\ibox{3}}<1> } } können ] ] 
%% ]
%% \end{tikzpicture}}
%% %}

\centerline{%
\begin{forest}
sm edges
[V\feattab{
              \vform \type{fin},\\
              \comps \ibox{1} $\oplus$ \ibox{2} } 
        [V\feattab{
              \vform \type{fin},\\
              \comps \ibox{1} $\oplus$ \ibox{2} $\oplus$ \sliste{
                \ibox{4} } } [wird;will] ]
        [{\ibox{4} V\feattab{
              \vform \type{bse},\\
              \subj  \ibox{1},\\
              \comps \ibox{2} }} 
           [{\ibox{3} V\feattab{
              \vform \type{bse},\\
              \subj  \ibox{1} \sliste{ NP[\type{nom}] }, \\ 
              \comps \ibox{2} \sliste{ NP[\type{acc}] } }} [lesen;read] ]
           [V\feattab{
              \vform \type{bse},\\
              \subj  \ibox{1},\\
              \comps \ibox{2} $\oplus$ \sliste{
                \ibox{3} } } [können;can] ] ] 
]
\end{forest}}


}

%\subsubsection{Das Verbalkomplexschema}
\subsubsection{The Predicate Complex Schemaa}

\frame{
%\frametitle{Das Verbalkomplexschema}
\frametitle{The Predicate Complex Schema}

\centerline{
\begin{forest}
[{[\comps \ibox{1}]}
  [\ibox{2} ]
  [{[\comps \ibox{1} $\oplus$ \sliste{ \ibox{2} }]}]]
\end{forest}
}

\begin{itemize}
%\item Zusätzlich zu Kopf-Komplement-Schema noch ein Verbalkomplexschema
\item In addition to the Head-Complement schema, \\
			there is also a Predicate Complex Schema
\pause
%\item Das ist sehr ähnlich, das Ergebnis, ist aber nicht \textsc{lex}$-$.\\
%Somit ist wiederholte Einbettung von Verbalkomplexen möglich.
\item This is very similar, but the result is not \textsc{lex}$-$.\\
Thus, repeated embedding of verbal complexes is possible.
\end{itemize}

}

%\subsection{Keine Verbalkomplexe bei VO-Sprachen}
\subsection{No verbal complexes for VO languages}

\frame{
%\frametitle{Englisch, Dänisch, \ldots}
\frametitle{English, Danish, \ldots}


\begin{itemize}
%\item Normalerweise muss ein Argument vollständig sein,\\
%      wenn es mit seinem Kopf kombiniert wird.
\item Normally, an argument must be complete,\\
when combined with its head.

\pause
%\item Verbalkomplexe sind anders: Wörter werden direkt verbunden.
\item verbal complexes are different: words are connected directly.

\pause
%\item Englisch und Dänisch haben nur die normale Regel, im Deutschen, Niederländischen gibt es
%  zusätzlich die Verbalkomplexregel.
\item English and Danish only have the normal rule, in German and Dutch there is
also the verbal complex rule.

\pause
%\item Die Hilfsverben betten in den SVO-Sprachen eine Verbphrase ein:
\item The auxiliary verbs embed a verb phrase in the SVO languages:
\ea
Nobody [will [read the book]].\\
\z
\end{itemize}

}



\frame{
\frametitle{Verbalkomplexe ohne Verbalkomplexbildung?}

\begin{itemize}
%\item Vorschläge, Hilfsverben auch mit einer VP zu kombinieren \citep{Wurmbrand2003b}:
\item Suggestions for combining auxiliary verbs with a VP \citep{Wurmbrand2003b}:
\ea
\gll dass keiner [[das Buch lesen] wird]\\
that nobody \hphantom{[[}the book read will\\
\z

\pause
%\item Wie funktioniert dann Scrambling?
\item So, how does scrambling work?
\ea
\gll dass [das Buch]$_i$ keiner [[ \_$_i$ lesen] wird]\\
that \spacebr{}the book nobody {} {} read will\\
\glt `that nobody will read the book'
\z

\pause
%\item Scrambling als Bewegung ist problematisch:\\
%      zusätzliche Lesarten bei der Umstellung von NPen mit Quantoren vorhergesagt\\(\citealp[\page 146]{Kiss2001a}; \citealp[Abschnitt~2.6]{Fanselow2001a}).
\item scrambling as movement is problematic:\\
additional readings predicted when converting NPs with quantifiers\\(\citealp[\page 146]{Kiss2001a}; \citealp[Abschnitt~2.6]{Fanselow2001a}).

\end{itemize}

}

\frame{
%\frametitle{Übungsaufgaben}
\frametitle{Exercises}

\begin{enumerate}
%\item Skizzieren Sie die Analyse der Verbalkomplexe für die folgenden Beispiele:
\item Sketch the analysis of the verbal complexes for the following examples:
\eal
\ex dass er darüber lachen wird
\ex dass er darüber wird lachen müssen
\ex dass er über diesen Witz wird haben lachen müssen
\zl
%\item Suchen Sie in der Zeitung oder in Korpora (COSMAS, COW) zwei Verbalkomplexe mit mindestens
%  drei Verben und analysieren Sie diese.
\item Search in the newspaper or in corpora (COSMAS, COW) for two verbal complexes with at least
three verbs and analyze them.

%\item Suchen Sie in Korpora Verbalkomplex mit mehr als vier Verben. Dokumentieren Sie Ihr Vorgehen.
\item Look for verbal complexes with more than four verbs in corpora. Document your approach.

\end{enumerate}

%\pause\pause

}


%      <!-- Local IspellDict: de_DE -->



