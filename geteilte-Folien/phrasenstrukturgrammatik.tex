\subsection{Phrase structure grammar}


\subsubsection{Phrase structure}

\frame{
\frametitle{Phrase structure}

\smallframe
\hfill%
\begin{tabular}{@{}l@{\hspace{1cm}}l@{}}
\scalebox{.7}{%
\begin{forest}
sm edges
[S
  [NP [Aicke;Aicke] ]
  [NP
    [Det [dem;the] ]
    [N [Affen;monkey] ] 
  ]
  [NP
    [Det [den;the] ]
    [N [Stock;stick] ] 
  ]
  [V [gibt;gives] ]
]
\end{forest}} &
\scalebox{.7}{%
\begin{forest}
sm edges
[V
  [NP [Aicke;Aicke] ]
  [V
    [NP
      [Det [dem;the] ]
      [N [Affen;monkey] ] ]
    [V
      [NP
        [Det [den;the] ]
        [N [Stock;stick] ] ]
      [V [gibt;gives] ] ] ] ]
\end{forest}}
\\
\\[-0.4ex]
\begin{tabular}{@{~}l@{ }l@{}}
NP & $\to$ Det, N            \\
S  & $\to$ NP, NP, NP, V  \\
\end{tabular} & \begin{tabular}{@{~}l@{ }l@{}}
NP & $\to$ Det, N  \\
V  & $\to$ NP, V\\
\end{tabular}\\
\end{tabular}
\hfill\mbox{}

\medskip
Rewrite rules are the real thing! Trees are just visualizations.\\
%
\pause%
%
Sometimes a bracket notation is used:\\
{}[\sub{S} [\sub{NP} Aicke] [\sub{NP} [\sub{Det} dem] [\sub{N} Affen]]  [\sub{NP} [\sub{Det} den] [\sub{N} Stock]] [\sub{V} gibt]]

\handoutspace
}

\iftoggle{psgbegriffe}{
\subsubsection{Terminology}


\frame{
\frametitle{Nnode}

\vfill
\psset{xunit=5mm,yunit=5mm,nodesep=8pt}
\hfill
\begin{pspicture}(0,0)(14,7.4)
\rput(3,7){\rnode{xp}{A}}
\rput(1,4){\rnode{up}{B}}\rput(5,4){\rnode{xs1}{C}}
\rput(5,1){\rnode{vp}{D}}

\psset{angleA=-90,angleB=90,arm=0pt}
\ncdiag{xp}{up}\ncdiag{xp}{xs1}%
\ncdiag{xs1}{vp}\ncdiag{xs1}{xs2}%
\ncdiag{xs2}{wp}\ncdiag{xs2}{x}%
\ncdiag{wp}{yp}\ncdiag{wp}{ws}%

\pause

%\mode<beamer>{
\psset{linecolor=red}%radius=1em}
%}
%\pscircle(3,7){2ex}
\cnode[linewidth=1.5pt](3,7){1.7ex}{nodeA}
\pscircle[linewidth=1.5pt](1,4){1.7ex}\cnode[linewidth=1.5pt](5,4){1.7ex}{nodeC}
\cnode[linewidth=1.5pt](5,1){1.7ex}{nodeD}

\pause

\rput[l](8,7){\rnode{verz}{verzweigend}}
\rput[l](8,6){\rnode{nverz}{n}icht verzweigend}

%\psset{angleA=180,angleB=0,arm=0pt,arrows=->}
\only<3>{
\ncline{->}{verz}{nodeA}
}
\pause
\only<4>{
\ncline{->}{nverz}{nodeC}
}
%\psgrid
\end{pspicture}
\hfill\hfill\mbox{}
\vfill
}

\frame{

\frametitle{Mother, daughter and sister}

\vfill
\psset{xunit=5mm,yunit=5mm,nodesep=8pt}
\hspace{1cm}%
%\begin{tabular}{@{}l@{\hspace{1cm}}l@{}}
\begin{pspicture}(0,0)(7.4,7.4)
\rput(3,7){\rnode{xp}{A}}
\rput(1,4){\rnode{up}{B}}\rput(5,4){\rnode{xs1}{C}}
\rput(5,1){\rnode{vp}{D}}

\psset{angleA=-90,angleB=90,arm=0pt}
\ncdiag{xp}{up}\ncdiag{xp}{xs1}%
\ncdiag{xs1}{vp}\ncdiag{xs1}{xs2}%
\ncdiag{xs2}{wp}\ncdiag{xs2}{x}%
\ncdiag{wp}{yp}\ncdiag{wp}{ws}%

%\psgrid
\end{pspicture}
\hspace{1cm}\raisebox{3cm}{\begin{tabular}[t]{@{}l@{}}
A is the mother of B and C\\
C is the mother of D\\
B is the sister of C\\
\end{tabular}}


As in genealogies

\vfill

}

\iftoggle{einfsprachwiss-exclude}{
\frame{
\frametitle{Dominance}

\vfill
\psset{xunit=5mm,yunit=5mm,nodesep=8pt}
\hspace{1cm}
\begin{pspicture}(0,0)(7.4,7.4)
\rput(3,7){\rnode{xp}{A}}
\rput(1,4){\rnode{up}{B}}\rput(5,4){\rnode{xs1}{C}}
\rput(5,1){\rnode{vp}{D}}

\psset{angleA=-90,angleB=90,arm=0pt}
\ncdiag{xs1}{xs2}%
\ncdiag{xs2}{wp}\ncdiag{xs2}{x}%
\ncdiag{wp}{yp}\ncdiag{wp}{ws}%

\alt<2>{
\mode<beamer>{
\psset{linecolor=red}
}
\ncdiag{->}{xp}{up}\ncdiag{->}{xp}{xs1}
}{
\ncdiag{xp}{up}\ncdiag{xp}{xs1}%
}
\alt<2,4>{
\mode<beamer>{
\psset{linecolor=red}
}
\ncdiag{->}{xs1}{vp}
}{
\ncdiag{xs1}{vp}
}
%\psgrid
\end{pspicture}
\hspace{1cm}\raisebox{3cm}{\begin{tabular}[t]{@{}l@{}}
A dominates \only<2->{B, C and D}\\
\only<3->{C dominates} \only<4->{D} \\
\end{tabular}}

\bigskip

A dominates B iff A is higher in the tree and\\
if there is a connection from A to B that goes downwards only.

\pause\pause\pause

\vfill

}

\frame{

\frametitle{Immediate dominance}

\psset{xunit=5mm,yunit=5mm,nodesep=8pt}
\hspace{1cm}
\begin{pspicture}(0,0)(7.4,7.4)
\rput(3,7){\rnode{A}{A}}
\rput(1,4){\rnode{B}{B}}\rput(5,4){\rnode{C}{C}}
\rput(5,1){\rnode{D}{D}}

\psset{angleA=-90,angleB=90,arm=0pt}
\ncdiag{C}{xs2}%
\ncdiag{xs2}{wp}\ncdiag{xs2}{x}%
\ncdiag{wp}{yp}\ncdiag{wp}{ws}%

\alt<2>{
\mode<beamer>{
\psset{linecolor=red}
}
\ncdiag{->}{A}{B}\ncdiag{->}{A}{C}
}{
\ncdiag{A}{B}\ncdiag{A}{C}%
}
\alt<4>{
\mode<beamer>{
\psset{linecolor=red}
}
\ncdiag{->}{C}{D}
}{
\psset{linecolor=black}
\ncdiag{C}{D}
}
%\psgrid
\end{pspicture}
\hspace{1cm}\raisebox{3cm}{\begin{tabular}[t]{@{}l@{}}
A dominates immediately \only<2->{B and C}\\
\only<3->{C dominates immediately} \only<4->{D} \\
\end{tabular}}

\bigskip

A dominates B immediately iff \\
A dominates B and there is no node C between A and B.

\pause\pause\pause


}


\frame{
\frametitle{Precedence}

\begin{description}[<+->]
\item[Precedence]~\\ A preceeds B iff A is to the left of B in the tree and\\
     none of the two nodes dominates the other.
\item[Immediate precedence]~\\ There is no element C between A and B.
\end{description}

}
}%\end{einfsprachwiss-exclude}
}%psgbegriffe


\subsubsection{A sample grammar}


\frame[shrink=8]{
\frametitle{Example derivation with a flat structure}

\vfill

\bigskip
\parskip0pt
\begin{tabular}[t]{@{}l@{ }l}
\highlight{NP}<5,8> & \highlight{$\to$ Det N}<5,8>\\          
\highlight{S}<10>  & \highlight{$\to$ NP NP NP V}<10>
\end{tabular}\hspace{2cm}%
\begin{tabular}[t]{@{}l@{ }l}
\highlight{NP}<2> & \highlight{$\to$ Aicke}<2>\\
\highlight{Det}<3>  & \highlight{$\to$ dem}<3>\\
\highlight{Det}<6>  & \highlight{$\to$ den}<6>\\
\end{tabular}\hspace{8mm}
\begin{tabular}[t]{@{}l@{ }l}
\highlight{N}<4> & \highlight{$\to$ Affen}<4>\\
\highlight{N}<7> & \highlight{$\to$ Stock}<7>\\
\highlight{V}<9> & \highlight{$\to$ gibt}<9>\\
\end{tabular}
\vfill

\begin{tabular}{@{}llllll@{\hspace{2.5cm}}l}
Aicke            & dem          & Affen          & den          & Stock & gibt                \pause\\
\highlight{NP}<2> & dem          & Affen          & den          & Stock & gibt & \only<handout>{NP $\to$ Aicke}  \pause\\
NP            & \highlight{Det}<3> & Affen          & den          & Stock & gibt & \only<handout>{Det $\to$ das}  \pause\\
NP            & Det            & \highlight{N}<4>  & den          & Stock & gibt & \only<handout>{N $\to$ Buch} \pause\\
NP            &              & \highlight{NP}<5> & den          & Stock & gibt & \only<handout>{NP $\to$ Det N}\pause\\
NP            &              & NP            & \highlight{Det}<6> & Stock & gibt & \only<handout>{Det $\to$ den}  \pause\\
NP            &              & NP            & Det            & \highlight{N}<7>    & gibt & \only<handout>{N $\to$ Stock} \pause\\
NP            &              & NP            &              & \highlight{NP}<8>       & gibt & \only<handout>{NP $\to$ Det N}\pause\\
NP            &              & NP            &              & NP       & \highlight{V}<9>   & \only<handout>{V $\to$ gibt}  \pause\\
              &              &               &              &      & \highlight{S}<10>   & \only<handout>{S $\to$ NP NP NP V}\\
\end{tabular}

\vfill
}


\begin{frame}[fragile]
\frametitle{Do try this at home!}

You may try such grammars on your own.
\begin{itemize}
\item Go to \url{https://swish.swi-prolog.org/}.
\item Click "`Program"'.
\item Enter the following:
\begin{verbatim}
s --> np, v, np, np.
np --> det, n.
np --> [Aicke].
det --> [dem].
det --> [den].
n --> [affen].
n --> [stock].
v --> [gibt].
\end{verbatim}
\item Enter the following into the right box: \texttt{s([Aicke,gibt,dem,affen,den,stock],[]).}
\item The word "`true"' should appear in the box on top. If so, celebrate!
\end{itemize}

\end{frame}

\frame{
\frametitle{A generative grammar}

\begin{itemize}
\item The grammar that you entered can generate sentences.
\pause
\item You can tests, which sentences the grammar generates by entering the following:
\texttt{s([X],[]),print(X),nl,fail.}

\pause
\item \texttt{s([X],[])} asks Prolog to find an X that is an "`s"'.
\pause
\item \texttt{print(X),nl} prints the X and a newline.
\pause
\item \texttt{fail} tells Prolog that we are not satisfied an expect it to find another solution.
\pause
\item It tries to find and print other solutions and fails if it runs out of possibilities to try.
\pause
\item Some grammars generate infinitely many Xs. So this process would never terminate (except if
  the computer runs out of memory \ldots).

\end{itemize}

}




\frame{

\frametitle{Sentences described by the grammar}



\begin{itemize}
\item The grammar is not detailed enough:\\
\begin{tabular}{@{}l@{ }l}
NP & $\to$ Det N\\
S  & $\to$ NP NP NP V\\
\end{tabular}
\eal
\ex[]{
\gll Aicke dem Affen den Stock gibt\\
     Aicke the monkey the stick gives\\
}
\ex[*]{
\gll ich dem Affen den Stock gibt\\
     I   the monkey the stick gives\\\\
\pause
(subject verb agreement \emph{ich}, \emph{gibt})}
\pause
\ex[*]{
\gll Aicke dem Affen dem Stock gibt\\
     Aicke the monkey the stick gives\\\\
\pause
(case assignment of the verb: \emph{gibt} needs accusative)
}
\pause
\ex[*]{
\gll Aicke dem Affen das Stock gibt\\
     Aicke the monkey the stick gives\\\\
\pause
(determinator noun agreement \emph{das}, \emph{Stock})
}
\zl
\end{itemize}

}

% geht hier nicht, weil das von anderen eingebunden wird
%\exewidth{\exnrfont(12)}

\frame{

\frametitle{Subject verb agreement (I)}


\begin{itemize}
\item agreement in person (1, 2, 3) and number (sg, pl)
\eal
\ex Ich schlafe. (1, sg)
\ex Du schläfst.  (2, sg)
\ex Er schläft.  (3, sg)
\ex Wir schlafen. (1, pl)
\ex Ihr schlaft.  (2, pl)
\ex Sie schlafen. (3,pl)
\zl
\item How can this be expressed in rules?
\end{itemize}

}

\frame{
\frametitle{Subject verb agreement (II)}

\begin{itemize}
\item making the symbols more specific\\
            aus S $\to$ NP NP NP V wird\\[2ex]
\begin{tabular}{@{}l@{ }l}
S  & $\to$ NP\_1\_sg NP NP V\_1\_sg\\
S  & $\to$ NP\_2\_sg NP NP V\_2\_sg\\
S  & $\to$ NP\_3\_sg NP NP V\_3\_sg\\
S  & $\to$ NP\_1\_pl NP NP V\_1\_pl\\
S  & $\to$ NP\_2\_pl NP NP V\_2\_pl\\
S  & $\to$ NP\_3\_pl NP NP V\_3\_pl\\
\end{tabular}

\item six symbols for noun phrases, six for verbs
\item six rules instead of one
\end{itemize}

}

\frame{

\frametitle{Case assignment by the verb}

\begin{itemize}
\item Case has to be represented:
\begin{tabular}{@{}l@{ }l}
S  & $\to$ NP\_1\_sg\_nom NP\_dat NP\_acc V\_1\_sg\_ditransitiv\\
S  & $\to$ NP\_2\_sg\_nom NP\_dat NP\_acc V\_2\_sg\_ditransitiv\\
S  & $\to$ NP\_3\_sg\_nom NP\_dat NP\_acc V\_3\_sg\_ditransitiv\\
S  & $\to$ NP\_1\_pl\_nom NP\_dat NP\_acc V\_1\_pl\_ditransitiv\\
S  & $\to$ NP\_2\_pl\_nom NP\_dat NP\_acc V\_2\_pl\_ditransitiv\\
S  & $\to$ NP\_3\_pl\_nom NP\_dat NP\_acc V\_3\_pl\_ditransitiv\\
\end{tabular}
\item 3 * 2 * 4 = 24 new categories for NP in total
\item 3 * 2 * x  categories for V (x = number of different valence classes)
\end{itemize}

}


\frame[shrink=15]{

\frametitle{Determiner noun agreement}

\begin{itemize}
\item Agreement in gender (fem, mas, neu), number (sg, pl) and
      case (nom, gen, dat, acc)
\eal
\ex
\gll der Mann, die Frau, das Buch (gender)\\
     the man   the woman the book\\
\ex 
\gll das Buch, die Bücher (number)\\
     the book  the books\\
\ex 
\gll des Buches, dem Buch (case)\\
     the book    the book\\
\zl
\pause
\item from NP $\to$ Det N we get:\\[2ex]
\resizebox{\linewidth}{!}{
\begin{tabular}{@{}l@{ }l@{\hspace{4mm}}l@{ }l}
NP\_3\_sg\_nom  & $\to$ Det\_fem\_sg\_nom N\_fem\_sg\_nom & NP\_gen  & $\to$ Det\_fem\_sg\_gen N\_fem\_sg\_gen\\
NP\_3\_sg\_nom  & $\to$ Det\_mas\_sg\_nom N\_mas\_sg\_nom & NP\_gen  & $\to$ Det\_mas\_sg\_gen N\_mas\_sg\_gen\\
NP\_3\_sg\_nom  & $\to$ Det\_neu\_sg\_nom N\_neu\_sg\_nom & NP\_gen  & $\to$ Det\_neu\_sg\_gen N\_neu\_sg\_gen\\
NP\_3\_pl\_nom  & $\to$ Det\_fem\_pl\_nom N\_fem\_pl\_nom & NP\_gen  & $\to$ Det\_fem\_pl\_gen N\_fem\_pl\_gen\\
NP\_3\_pl\_nom  & $\to$ Det\_mas\_pl\_nom N\_mas\_pl\_nom & NP\_gen  & $\to$ Det\_mas\_pl\_gen N\_mas\_pl\_gen\\
NP\_3\_pl\_nom  & $\to$ Det\_neu\_pl\_nom N\_neu\_pl\_nom & NP\_gen  & $\to$ Det\_neu\_pl\_gen N\_neu\_pl\_gen\\[2mm]


\ldots & \hphantom{$\to$} dative                                                             & \ldots & \hphantom{$\to$} accusative\\[2mm]
\end{tabular}
}
\item 24 symbols for determiners, 24 symbols for nouns
\item 24 rules instead of one
\end{itemize}
}

\subsubsection{Extension of PSG by features}


\frame{

\frametitle{Problems of this approach}

\begin{itemize}
\item Gernalizations not captured.
\item neither in the rules nor in category symbols
      \begin{itemize}
      \item Where can an NP or an NP\_nom be placed?\\
            Not: Where can an NP\_3\_sg\_nom be placed?
      \item Comonalities of rules are not obvious.
      \end{itemize}
\pause
\item Solution: Features with values and identity of values\\
      Category symbol: NP Feature: Per, Num, Kas, \ldots\\

We get rules like the following:\\

\begin{tabular}{@{}l@{ }l}
NP(3,sg,nom)  & $\to$ Det(fem,sg,nom) N(fem,sg,nom)\\
NP(3,sg,nom)  & $\to$ Det(mas,sg,nom) N(mas,sg,nom)\\
\end{tabular}
\end{itemize}
}


\frame{
\frametitle{Features and rule schemata (I)}

\begin{itemize}
\item Rules with special values are generalized to schemata:

\medskip

\begin{tabular}{@{}l@{ }l@{ }l}
NP(\blau<3>{3},\blau<2>{Num},\blau<2>{Cas}) & $\to$ & Det(\gruen<2>{Gen},\blau<2>{Num},\blau<2>{Cas}) N(\gruen<2>{Gen},\blau<2>{Num},\blau<2>{Cas})\\
\end{tabular}
\pause
\item Gen, Num and Cas values do not matter,\\
      as long as the values are identical
\pause
\item The value of the person feature (first slot in NP(3,Num,Kas))\\
 is fixed by the rule: 3.
\end{itemize}
}


\frame{
\frametitle{Features and rule schemata (II)}

\begin{itemize}
\item Rules with specific values are generalized into rule schemata:

\medskip
\begin{tabular}{@{}l@{ }l@{ }l}
NP({3},{Num},{Kas}) & $\to$ & Det(Gen,{Num},{Kas}) N(Gen,{Num},{Kas})\\
S  & $\to$ & NP(\blau<1>{Per1},\blau<1>{Num1},\blau<3>{nom})\\
   &       & NP(Per2,Num2,\blau<3>{dat})\\
   &       & NP(Per3,Num3,\blau<3>{acc})\\
   &       & V(\blau<1>{Per1},\blau<1>{Num1})\\\\
\end{tabular}
\item Per1 and Num1 are identical for verb and subject.
\pause
\item The values of other NPs do not matter.\\
      (notation for irrelevant values: `\_')
\pause
\item Case of the NPs are specified in the second rule.
\end{itemize}

}

%% Kommt dann in Theorie anders, deshalb hier raus
%% \frame{

%% \small
%% \frametitle{Bündelung von Merkmalen}

%% \begin{itemize}
%% \item Kann es Regeln geben, in denen nur der Per-Wert oder nur der Num-Wert identisch sein muß?\\[2ex]

%% \begin{tabular}{@{}l@{ }l@{ }l}
%% S  & $\to$ & NP(Per1,Num1,nom)\\
%%    &       & NP(Per2,Num2,dat)\\
%%    &       & NP(Per3,Num3,akk)\\
%%    &       & V(Per1,Num1)\\\\
%% \end{tabular}
%% \pause
%% \item Gruppierung von Information $\to$ stärkere Generalisierung, stärkere Aussage\\[2ex]

%% \begin{tabular}{@{}l@{ }l@{ }l}
%% S  & $\to$ & NP(Agr1,nom)\\
%%    &       & NP(Agr2,dat)\\
%%    &       & NP(Agr3,akk)\\
%%    &       & V(Agr1)\\\\
%% \end{tabular}

%% wobei Agr ein Merkmal mit komplexen Wert ist: \zb agr(1,sg)
%% \end{itemize}


%% }


\iftoggle{hpsgvorlesung}{
\subsubsection{Abstraktion über Regeln: \texorpdfstring{\xbar}{X-Bar}-Theorie}

\frame{
\frametitle{Abstraktion über Regeln}

\xbar"=Theorie \citep{Jackendoff77a}:

\medskip
\oneline{\(
\begin{array}{@{}l@{\hspace{1cm}}l@{\hspace{1cm}}l}
\xbar\mbox{-Regel} & \mbox{mit Kategorien} & \mbox{Beispiel}\\[2mm]
\overline{\overline{\mbox{X}}} \rightarrow \overline{\overline{\mbox{Spezifikator}}}~~\xbar & \overline{\overline{\mbox{N}}} \rightarrow \overline{\overline{\mbox{DET}}}~~\overline{\mbox{N}} & \mbox{das [Bild von Maria]} \\
\xbar \rightarrow \xbar~~\overline{\overline{\mbox{Adjunkt}}}             & \overline{\mbox{N}} \rightarrow \overline{\mbox{N}}~~\overline{\overline{\mbox{REL\_SATZ}}} & \mbox{[Bild von Maria] [das alle kennen]}\\
\xbar \rightarrow \overline{\overline{\mbox{Adjunkt}}}~~\xbar             & \overline{\mbox{N}} \rightarrow \overline{\overline{\mbox{ADJ}}}~~\overline{\mbox{N}} & \mbox{schöne [Bild von Maria]}\\
\xbar \rightarrow \mbox{X}~~\overline{\overline{\mbox{Komplement}}}*               & \overline{\mbox{N}} \rightarrow \mbox{N}~~\overline{\overline{\mbox{P}}} & \mbox{Bild [von Maria]}\\\\
\end{array}
\)}

X steht für beliebige Kategorie, `*' für beliebig viele Wiederholungen

}

\frame{
\frametitle{\xbar-Theorie}

\xbar-Theorie wird in vielen verschiedenen Frameworks angenommen:\\
\begin{itemize}
\item Government \& Binding (GB): \citew*{Chomsky81a}
\item Lexical Functional Grammar (LFG): \citew{Bresnan82a-ed,Bresnan2001a}
\item Generalized Phrase Structure Grammar (GPSG):\\
      \citew*{GKPS85a}
\end{itemize}

}



\subsection{Hausaufgabe}

\frame{
\frametitle{Hausaufgabe}

\begin{enumerate}
\item Schreiben Sie eine Phrasenstrukturgrammatik, mit der man u.\,a.\ die Sätze in (\mex{1})
      analysieren kann, die die Wortfolgen in (\mex{2}) aber nicht zulässt.
      \eal
      \ex[]{
      Der Mann hilft der Frau.
      }
      \ex[]{
      Er gibt ihr das Buch.
      }
      \ex[]{
      Er wartet auf ein Wunder.
      }
%       \ex[]{
%       Er wartet neben dem Bushäuschen auf ein Wunder.
%       }
      \zl
      \eal
      \ex[*]{
        Der Mann hilft er.
      }
      \ex[*]{
        Er gibt ihr den Buch.
      }
      \zl
      Dabei sollen Sie nicht für jeden Satz einzeln eigene Regeln für NP usw.\ aufstellen, sondern gemeinsame Regeln für
      alle aufgeführten Sätze entwickeln.

      Sie können für Ihre Arbeit auch Prolog benutzen: \url{https://swish.swi-prolog.org} zur Syntax
      für die Grammatiken siehe \url{https://en.wikipedia.org/wiki/Definite_clause_grammar}.
\end{enumerate}

}

} % if hpsgvorlesung




%      <!-- Local IspellDict: en_US-w_accents -->
