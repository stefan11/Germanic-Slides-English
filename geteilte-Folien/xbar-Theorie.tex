\subsection{\xbar syntax}

\subsubsection{Noun phrases}
\label{sec-psg-np}

\frame[shrink=5]{
\frametitle{Noun phrases}

\begin{itemize}
\item Until now NPs were Det + N, but NPs can be much more complex:

\eal
\label{Beispiele-NP-Adjunkte}
\ex 
\gll ein Buch\\
     a   book\\
\ex
\label{ex-ein-Buch-das-wir-kennen} 
\gll ein Buch, das  wir kennen\\
     a   book  that we  know\\
\ex 
\label{ex-ein-Buch-aus-Japan}
\gll ein Buch aus  Japan\\
     a   book from Japan\\
\ex 
\gll ein interessantes Buch\\
     an   interesting   book\\
\ex 
\gll ein Buch aus  Japan, das  wir kennen\\
     a   book from Japan  that we  know\\
\ex 
\gll ein interessantes Buch aus  Japan\\
     an  interesting   book from Japan\\
\ex 
\gll ein interessantes Buch, das  wir kennen\\
     an  interesting   book  that we  know\\
\ex 
\gll ein interessantes Buch aus  Japan, das  wir kennen\\
     an  interesting   book from Japan  that we  know\\
\zl

Additional material in (\mex{0}) are adjuncts.

\end{itemize}

}

\frame{
\frametitle{Adjectives in NPs}

\begin{itemize}
\item Suggestion:
\eal
\ex NP $\to$ Det N
\ex NP $\to$ Det A N
\zl
\pause
\item What about (\mex{1})?
\ea
\gll alle weiteren schlagkräftigen Argumente\\
     all further strong arguments\\
\glt `all other strong arguments'
\z
\pause
\item (\mex{1}) seems to be needed for the analysis of (\mex{0}):
\ea 
NP $\to$ Det A A N
\z
\pause
\item We do not want to stipulate a maximum number of adjectives in NPs: 
\ea 
NP $\to$ Det A* N
\z
\end{itemize}

}

\frame{
\frametitle{Adjectives in NPs (II)}

\begin{itemize}

\item Problem: Adjective and noun do not form a constituent if we assume (\mex{1}).
\ea 
NP $\to$ Det A* N
\z
Constituency tests suggest that A + N form a constituent:
\ea
\gll alle [[großen Seeelefanten] und [grauen Eichhörnchen]]\\
     all  \spacebr{}\spacebr{}big elephant.seals and  \spacebr{}gray squirrels\\
\glt `all big elephant seals and gray squirrels'	 
\z

\end{itemize}

}

\frame{
\frametitle{Adjective + noun as constituent}

\begin{itemize}
\item Better rules:
\eal
\ex NP $\to$ Det \nbar
\ex \nbar $\to$ A \nbar
\ex \nbar $\to$ N
\zl


\hfill%
\scalebox{.65}{%
\begin{forest}
sm edges
[NP
   [Det [ein;a] ]
   [\nbar
      [N [Eichhörnchen;squirrel] ] ] ]
\end{forest}}
\hfill
\scalebox{.65}{%
\begin{forest}
sm edges
[NP
   [Det [ein;a] ]
   [\nbar
      [A [graues;gray] ]
      [\nbar
        [N [Eichhörnchen;squirrel] ] ] ] ]
\end{forest}}
%
\hfill
\scalebox{.65}{%
\begin{forest}
sm edges
[NP
  [Det [ein;a] ]
    [\nbar
    [A [großes;big] ]
       [\nbar
       [A [graues;gray] ]
         [\nbar
         [N [Eichhörnchen;squirrel] ] ] ] ] ]
\end{forest}}
\hfill\mbox{}
%
\end{itemize}

}





% Das Adjektiv \emph{klug} schränkt die Menge der Bezugselemente der Nominalgruppe ein. Nimmt man ein
% weiteres Adjektiv wie \emph{glücklich} dazu, dann bezieht man sich nur auf die Frauen, die sowohl glücklich
% als auch klug sind. Solche Nominalphrasen können \zb in Kontexten wie dem folgenden verwendet
% werden:
% \ea
% \label{Beispiel-Iteration-Adjektive}
% A: Alle klugen Frauen sind unglücklich.\\
% B: Nein, ich kenne eine glückliche kluge Frau.
% \z
% Man kann sich nun überlegen, dass dieses schöne Gespräch mit Sätzen wie \emph{Aber alle glücklichen
%   klugen Frauen sind schön} und einer entsprechenden Antwort weitergehen kann. Die Möglichkeit,
% Nominalphrasen wie \emph{eine glückliche kluge Frau} weitere Adjektive hinzuzufügen, ist im
% Regelsystem in (\mex{-1}) angelegt.

% Wir haben jetzt eine wunderschöne kleine Grammatik entwickelt, die Nominalphrasen mit
% Adjektivmodifikatoren analysieren kann. Dabei wird der Kombination von Adjektiv und Nomen
% Konstituentenstatus zugesprochen. Der Leser wird sich jetzt vielleicht fragen, ob man nicht genauso
% gut die Kombination aus Determinator und Adjektiv als Konstituente behandeln könnte, denn es gibt
% auch Nominalgruppen der folgenden Art:
% \ea
% diese schlauen und diese neugierigen Frauen
% \z
% Hier liegt jedoch noch eine andere Struktur vor: Koordiniert sind zwei vollständige Nominalphrasen,
% wobei im ersten Konjunkt ein Teil gelöscht wurde:% \footnote{
% %   Außerdem gibt es bei entsprechendem Kontext natürlich noch die Lesart, in der sich \emph{diese schlauen} nicht auf Frauen
% %   sondern \zb auf Männer bezieht:
% % \ea
% % Hier stehen mehrere Gruppen von Männern
% \ea
% diese schlauen \st{Frauen} und diese neugierigen Frauen
% \z
% Man findet ähnliche Phänomene im Satzbereich und sogar bei Wortteilen:
% \eal
% \ex dass Peter dem Mann das Buch \st{gibt} und Maria der Frau die Schallplatte gibt
% \ex be- und entladen
% \zl
% % Dass in (\mex{-1}) wirklich keine normale symmetrische Koordination vorliegt, sieht man, wenn man
% % (\mex{-1}) mit (\mex{1}) vergleicht:
% % \ea
% % diese schlauen Frauen und klugen Männer
% % \z
% % Mit (\mex{0}) verweist man auf eine Gruppe, die aus schlauen Frauen und klugen Männern besteht,
% % wohingegen man mit (\mex{-2}) auf zwei Gruppen verweist, nämlich

\frame{
\frametitle{Other adjuncts}


\begin{itemize}
\item Other adjuncts are analogous:
\eal
\ex \nbar $\to$ \nbar PP
\ex \nbar $\to$ \nbar RelativeClause
\zl
\pause
\item With the rules introduced so far,\\
      we can anaylze all determiner-adjunct-noun combinations.

\end{itemize}

}

\frame{
\frametitle{Complements}


\begin{itemize}
\item \nbar $\to$ N contains one noun only,
      but some nouns allow further arguments:
\eal
\ex 
\gll der Vater von Peter\\
	 the father of Peter\\
\glt `Peter's father'
\ex 
\gll das Bild vom Gleimtunnel\\
	 the picture of.the Gleimtunnel\\
\glt `the picture of the Gleimtunnel'
\ex 
\gll das Kommen der Installateurin\\
	 the coming of.the plumber\\
\glt `the plumber's visit'
\zl

\pause
\item Therefore:
\ea
\nbar $\to$ N PP
\z

\end{itemize}

}

\frame{
\frametitle{Complements (and adjuncts)}

\hfill

\centerfit{%
\begin{forest}
sm edges
[NP
 [Det [das;the] ]
 [\nbar
   [N [Bild;picture] ]
   [PP [vom Gleimtunnel;of.the Gleimtunnel,roof ] ] ] ]
\end{forest}%
\hspace{2em}%
\begin{forest}
sm edges
[NP
  [Det [das;the] ]
  [\nbar
    [\nbar
      [N [Bild;picture] ]
      [PP [vom Gleimtunnel;of.the Gleimtunnel,roof ] ] ] 
    [PP [im Gropiusbau;in.the Gropiusbau,roof ] ] ] ]
\end{forest}}


}

\frame{
\frametitle{Missing nouns (I)}

\begin{itemize}
\item Noun is missing, but adjuncts are present:
\eal
\ex
\gll ein interessantes \_\\
     an  interesting\\
\glt `an interesting one'
\ex 
\gll ein neues interessantes \_\\
     a   new   interesting\\
\glt `a new interesting one'

\ex 
\gll ein interessantes \_ aus  Japan\\
     an  interesting   {} from Japan\\
\glt `an interesting one from Japan'
\ex 
\gll ein interessantes \_, das  wir kennen\\
     an  interesting   {}  that we  know\\
\glt `an interesting one that we know'
\zl
\end{itemize}
}

\frame[shrink=25]{
\frametitle{Missing nouns (II)}
\begin{itemize}
\item Noun is missing, but complement of the noun is present:
\eal
\ex
\gll (Nein, nicht der Vater von Klaus), der \_ von Peter war gemeint.\\
	\spacebr{}no not the father of Klaus the {} of Peter was meant\\
\glt `No, it wasn't the father of Klaus, but rather the one of Peter that was meant.'
%\itdopt{that statt the one?}
\ex 
\gll (Nein, nicht das Bild von der Stadtautobahn), das \_ vom Gleimtunnel war beeindruckend.\\
	 \spacebr{}no not the picture of the motorway the {} of.the Gleimtunnel was impressive\\
\glt `No, it wasn't the picture of the motorway, but rather the one of the Gleimtunnel that was impressive.'
\ex 
\gll (Nein, nicht das Kommen des Tischlers), das \_ der Installateurin ist wichtig.\\
	 \spacebr{}no not the coming of.the carpenter the {} of.the plumber is important\\
\glt `No, it isn't the visit of the carpenter, but rather the visit of the plumber that is important.'
\zl
\pause
\item PSG: \alert{Epsilon production}
\pause
\item Notational variants:
\eal
\ex N $\to$
\ex N $\to$ $\epsilon$
\zl 

\pause
\item Rules in (\mex{0}) = empty boxes with the same labels as boxes with normal nouns

\end{itemize}
}

\frame{
\frametitle{Analyses with empty nouns}

\hfill
\begin{forest}
sm edges
[NP
  [Det [ein;an] ]
  [\nbar
    [A [interessantes;interesting] ]
    [\nbar
      [N [\trace ] ] ] ] ]
\end{forest}
\hfill
\begin{forest}
sm edges
[NP
  [Det [das;the] ]
  [\nbar
    [N [\trace] ]
    [PP [vom Gleimtunnel;of.the Gleimtunnel, roof] ] ] ]
\end{forest}
\hfill%
\mbox{}

}


\frame{
\frametitle{Missing determiners -- plural}

\begin{itemize}
\item Determiners can be omitted as well:

Plural:
\eal
\ex 
\gll Bücher\\
     books\\
\ex 
\gll Bücher, die  wir kennen\\
     books   that we  know\\
\ex 
\gll interessante Bücher\\
     interesting  books\\
\ex 
\gll interessante Bücher, die  wir kennen\\
     interesting  books   that we know\\
\zl

\end{itemize}

}


\frame{
\frametitle{Missing determiners -- mass nouns}

\begin{itemize}
\item With mass nouns in the singular:
\eal
\ex 
\gll Getreide\\
	 grain\\
\ex 
\gll Getreide, das gerade gemahlen wurde\\
	 grain that just ground was\\
\glt `grain that has just been ground'
\ex 
\gll frisches Getreide\\
	 fresh grain\\
\ex 
\gll frisches Getreide, das gerade gemahlen wurde\\
	 fresh grain that just ground was\\
\glt `fresh grain that has just been ground'
\zl

\end{itemize}

}


\frame{
\frametitle{Missing Determiners}


\centering
\begin{forest}
sm edges
[NP
  [Det [\trace] ]
  [\nbar
    [N [Bücher] ] ] ]
\end{forest}

}

\frame{
\frametitle{Missing determiners and missing nouns}

Determiner and nouns can be dropped simultaneously:
\eal
\ex 
\gll Ich lese interessante.\\
     I   read interesting\\
\glt `I read interesting ones.'
\ex 
\gll Dort drüben steht frisches, das gerade gemahlen wurde.\\
	 there over stands fresh that just ground was\\
\glt `Over there is some fresh (grain) that has just been ground.'
\zl

\centerline{%
\scalebox{.55}{%
\begin{forest}
sm edges
[NP
  [Det [\trace] ]
  [\nbar
    [A [interessante;interesting] ]
    [\nbar
      [N [\trace] ] ] ] ]
\end{forest}}
}
}

% % Mit diesen wenigen Phrasenstrukturregeln sind die wesentlichen Aspekte der Nominalsyntax
% % abgedeckt. Zwei Probleme werden dem aufmerksamen Leser nicht entgangen sein: Erstens muss man, wenn
% % eine Nominalstruktur kein Adjektiv enthält, ausschließen, dass sowohl der Determinator als auch das
% % Nomen leer sind, denn ansonsten würde man eine leere Nominalphrase ableiten. Das kann man mit
% % zusätzlichen Merkmalen sicherstellen \citep{Netter98a}.\NOTE{genaue Quellenangabe} Zweitens darf das
% % Nomen nicht entfallen, wenn kein Adjektiv in der Nominalgruppe gibt:
% % \eal
% % \ex[]{
% % Wir helfen den Männern.
% % }
% % \ex[]{
% % Wir helfen den schlauen.
% % }
% % \ex[*]{
% % Wir helfen den.
% % }
% % \ex[]{
% % Wir helfen denen.
% % }
% % \ex[]{
% % Wir helfen denen mit Hut.
% % }
% % \ex[]{
% % Wir helfen denen, die wir kennen.
% % }
% % \zl
% % In den Fällen ohne Adjektiv muss das Demonstrativpronomen verwendet werden.
% %
% %
% % Da man in elliptischen Kontexten beim
% % Auspacken einer ähnlich beschrifteten Schachtel bereits etwas gefunden hat, muss man die leere
% % Schachtel nicht mehr auspacken und so fällt es nicht auf, dass sie nichts enthält.


% %%%%%%%%%%%%%%%%%%%%%%%%%%%%%%%%%%%%%%%%%%%%%%%%%%%%%%%%%%%%%%%%%%%%%%%%%%%%%%%%%%%%%%%%%%%%%%%%%%

\subsubsection{Adjective phrases}

\frame[shrink]{
\frametitle{Adjective phrases}

\begin{itemize}
\item Until now we only had simple adjective like \emph{interessant} `interesting'.
\pause
\item Adjective phrases can be very complex:
\eal
\ex 
\gll der seiner Frau treue Mann\\
     the his.\DAT{} wife faithful man\\
\glt `the man faithful to his wife'
\ex 
\gll der auf seinen Sohn stolze Mann\\
     the on his.\ACC{} son proud man\\
\glt `the man proud of his son'
\ex 
\gll der seine Frau liebende Mann\\
     the his.\ACC{} woman loving man\\
\glt `the man who loves his wife'
\ex 
\gll der von seiner Frau geliebte Mann\\
     the by his.\DAT{} wife loved man\\
\glt `the man loved by his wife'	 
\zl
\pause
\item Rule for attributive adjectives has to be adapted:
\ea
\nbar $\to$ AP \nbar
\z
\pause
\item
Rules for APs:
\eal
\ex AP $\to$ NP A
\ex AP $\to$ PP A
\ex AP $\to$ A
\zl

\end{itemize}

}

\if 0

In den bisher ausgearbeiteten Regeln gibt es zwei Unschönheiten. Das sind die Regeln für Adjektive
bzw.\ Nomina ohne Komplemente in (\mex{0}c) bzw.\ (\ref{NP-Regeln-Nbar-N}) --  hier als (\mex{1}) wiederholt:
\ea
\nbar $\to$ N
\z
Werden diese Regeln angewendet, ergeben sich Teilbäume mit unärer Verzweigung, \dash mit einer Mutter,
die nur eine Tochter hat. Für ein Beispiel siehe Abbildung~\ref{Abbildung-Adjektive-in-NP}. Wenn wir
bei unserem Gleichnis mit den Schachteln bleiben, heißt das, dass es eine Schachtel gibt, die eine
Schachtel enthält, in der dann der eigentlich relevante Inhalt steckt. 

Im Prinzip hindert uns aber nichts, den Inhalt gleich in die größere Schachtel zu tun. Statt der
Regeln in (\mex{1}) verwenden wir einfach die Regeln in (\mex{2}):
\eal
\ex A $\to$ kluge
\ex N $\to$ Mann
\zl
\eal
\label{Lexikon-Projektion}
\ex AP $\to$ kluge
\ex \nbar $\to$ Mann
\zl
Mit (\mex{0}a) wird ausgedrückt, dass \emph{kluge} dieselben Eigenschaften wie vollständige
Adjektivphrasen hat, insbesondere kann es nicht mehr mit einem Komplement kombiniert werden. Das ist
parallel zur Kategoriesierung des Pronomens \emph{er} als NP in den Grammatiken
(\ref{bsp-grammatik-psg}) und (\ref{psg-binaer}).


Die Einordnung von Nomina, die kein Komplement verlangen, als \nbar hat außerdem auch den Vorteil,
dass man nicht erklären muss, warum es neben (\mex{1}a) auch noch die Analyse (\mex{1}b) geben soll, obwohl es
keinen Bedeutungsunterschied gibt.
\eal
\ex {}[\sub{NP} einige [\sub{\nbar} kluge [\sub{\nbar} [\sub{\nbar} Frauen ] und [\sub{\nbar} Männer
]]]]
\ex {}[\sub{NP} einige [\sub{\nbar} kluge [\sub{\nbar} [\sub{N} [\sub{N} Frauen ] und [\sub{N} Männer
]]]]]
\zl
In (\mex{0}a) sind zwei Nomina der Kategorie \nbar koordinativ verknüpft worden. Das Ergebnis einer
Koordination zweier Konstituenten gleicher syntaktischer Kategorie ist immer einer neue Konstituente
derselben syntaktischen Kategorie, in (\mex{0}a) also ebenfalls eine \nbar. Diese wird dann mit dem
Adjektiv und dem Determinator kombiniert.
In (\mex{0}b)) wurden die Nomina kombiniert. Das Ergebnis ist wieder eine Konstituente, die dieselbe
Kategorie hat, wie ihre Teile, also ein N. Dieses N wird zur \nbar, die dann mit dem Adjektiv
verbunden wird. Wenn man Nomina, die kein Komplement verlangen, nicht als N sondern als \nbar
kategorisiert, ergibt sich das beschriebene Problem mit sogenannten unechten Mehrdeutigkeiten
nicht. 

\fi

\subsubsection{Prepositional phrases}

\frame[shrink=25]{
\frametitle{Prepositional phrases}

\savespace
\begin{itemize}
\item The syntax of PPs is rather simple. First suggestion:
\ea
PP $\to$ P NP
\z
\pause
\item But PPs can be extended by measurement phrases and other additions qualifying the semantic
  contribution of the preposition:
\eal
\ex\label{Beispiel-Schritt-vor-dem-Abgrund} 
\gll {}[[Einen Schritt] vor dem Abgrund] blieb er stehen.\\
	 {}\spacebr{}\spacebr{}one step before the abyss remained he stand\\
\glt `He stopped one step in front of the abyss.'
\ex 
\gll {}[[Kurz] nach dem Start] fiel die Klimaanlage aus.\\
	 {}\spacebr{}\spacebr{}shortly after the take.off fell the air.conditioning out\\
\glt `Shortly after take off, the air conditioning stopped working.'
\ex 
\gll {}[[Schräg] hinter der Scheune] ist ein Weiher.\\
	 {}\spacebr{}\spacebr{}diagonally behind the barn is a pond\\
\glt `There is a pond diagonally across from the barn.'
% \ex 
% \gll {}[[Mitten] im Urwald] stießen die Forscher auf einen alten Tempel.\\
% 	 {}\spacebr{}\spacebr{}middle in.the jungle stumbled the researchers on an old temple\\
% \glt `In the middle of the jungle, the researches came across an old temple.'
\zl

% Man könnte jetzt für die Analyse von (\mex{0}a,b) eine Regel wie in (\mex{1}) vorschlagen:
% \eal
% \ex PP $\to$ NP PP
% \ex PP $\to$ AP PP
% \zl
% Die Regeln kombinieren eine PP mit einer Maßangabe. Das Ergebnis ist wieder eine PP. Mit den Regeln
% könnte man zwar die Präpositionalphrasen in (\mex{-1}a,b) analysieren, aber leider auch die in
% (\mex{1}):
% \eal
% \ex[*]{
% einen Schritt kurz vor dem Abgrund
% }
% \ex[*]{
% kurz einen Schritt vor dem Abgrund
% }
% \zl
% In (\mex{0}) wurden jeweils beide Regeln aus (\mex{-1}) angewendet.

% Durch Umformulierung der bisherigen Regeln kann man diesen Nebeneffekt vermeiden:
\eal
\ex PP $\to$ NP \pbar
\ex PP $\to$ AP \pbar
\ex PP $\to$ \pbar\label{Regel-PP-P}
\ex \pbar $\to$ P NP
\zl
\end{itemize}

}

\frame{
\frametitle{Prepositional phrases}

\hfill
\hfill
\begin{forest}
sm edges
[PP
  [\pbar
    [P [vor;before] ]
    [NP [dem Abgrund;the abyss, roof] ] ] ]
\end{forest}
\hfill
\begin{forest}
sm edges
[PP
  [AP [kurz;shortly,roof] ]
  [\pbar
    [P [vor;before] ]
    [NP [dem Abgrund;the abyss,roof] ] ] ]
\end{forest}
\hfill
\mbox{}

}



\subsubsection{\xbar theory}
\label{sec-xbar}

\frame{
\frametitle{Generalization over rules}

\begin{itemize}
\item head + complement = intermediate level:
\eal
\ex \nbar $\to$ N PP
\ex \pbar $\to$ P NP
\zl
\pause
\item intermediate level + further constituent = maximal projection
\eal
\ex NP $\to$ Det \nbar
\ex PP $\to$ NP \pbar
\zl
\pause
\item parallel structures for AP and VP in English
\end{itemize}

}


\frame{
\frametitle{Adjective phrases in Englisch}

\eal
\ex They are proud.
\ex They are very proud.
\ex They are proud of their child.
\ex They are very proud of their child.
\zl

\pause

\eal
\ex AP $\to$ \abar
\ex AP $\to$ Adv \abar
\ex \abar $\to$ A PP
\ex \abar $\to$ A
\zl

}


\frame{
\frametitle{Adjective phrases in Englisch}

\eal
\ex AP $\to$ \abar
\ex AP $\to$ AdvP \abar
\ex \abar $\to$ A PP
\ex \abar $\to$ A
\zl


\hfill
\begin{forest}
sm edges
[AP
  [\abar
    [A [proud] ] ] ]
\end{forest}
\hfill
\begin{forest}
sm edges
[AP
  [AdvP [very] ]
  [\abar
    [A [proud] ] ] ]
\end{forest}
\hfill
\begin{forest}
sm edges
[AP
  [\abar
    [A [proud] ]
    [PP [of their child,roof] ] ] ]
\end{forest}
\hfill
\begin{forest}
sm edges
[AP
  [AdvP [very] ]
  [\abar
    [A [proud] ]
    [PP [of their child,roof] ] ] ]
\end{forest}
\hfill
\mbox{}
\hfill
\mbox{}

}

\frame{
\frametitle{Further abstraction}

\begin{itemize}
\item We saw how one can abstract over case, gender and so on (variables in rule schemata).

\ea
NP({3},{Num},{Cas}) $\to$ D(Gen,{Num},{Cas}), N(Gen,{Num},{Cas})
\z

\pause
\item In a similar way one can abstract over part of speech.\\
      Instead of AP, NP, PP, VP, one writes XP.
\pause
\item Instead of (\mex{1}), one writes (\mex{2}):
\eal
\ex PP $\to$ \pbar
\ex AP $\to$ \abar
\zl
\ea
XP $\to$ \xbar
\z
\end{itemize}


}


\frame{
\frametitle{\xbar Theory: Assumptions}

Phrases have at least three levels:
\begin{itemize}
\item X$^0$ = head
\item X$'$ = intermediate level (= \xbar, X-Bar $\to$ name of the theory) 
\item XP = highest level (=~X$''$ = $\overline{\overline{\mbox{X}}}$), called maximal projection
\end{itemize}
%Neuere Analysen $\to$ teilweise Verzicht auf nichtverzweigende X$'$-Knoten
\nocite{Muysken82a}


}

\frame[shrink]{
\frametitle{Minimal and maximal structure of phrases}

\bigskip

\small\hfill
\begin{forest}
sm edges
[XP
  [\xbar [X] ] ]
\end{forest}
\hfill
\begin{forest}
%where n children=0{}{},
%sm edges
%for tree={parent anchor=south, child anchor=north,align=center,base=bottom}
[XP
  [specifier]
  [\xbar
    [adjunct]
    [\xbar
      [complement] [X] ] ] ]
\end{forest}
\hfill\mbox{}


\begin{itemize}
\item Adjuncts are optional $\to$ there does not need to be an X$'$ with adjunct daughter.
\pause
\item Some categories do not have a specifier or it is optional (\eg A).\\
%(Zusätzliche Regel nötig $\overline{\overline{\mbox{X}}} \rightarrow \xbar$)
\pause
\item Sometimes adjuncts to XP or head adjuncts to X are assumed.
\end{itemize}

}

\frame{
\frametitle{\xbar Theory: Rules according to \citew{Jackendoff77a}}\nocite{KP90a}\nocite{Pullum85a}



\oneline{%
\begin{tabular}[t]{@{}l@{\hspace{5mm}}l@{\hspace{5mm}}l@{}}
\xbar\mbox{ rule} & \mbox{with specific categories} & \mbox{example strings}\\[2mm]
$\overline{\overline{\mbox{X}}} \rightarrow \overline{\overline{\mbox{specifier}}}$~~\xbar &
$\overline{\overline{\mbox{N}}} \rightarrow \overline{\overline{\mbox{DET}}}$~~\nbar & \mbox{the [picture of Paris]} \\
$\xbar \rightarrow$ \xbar~~$\overline{\overline{\mbox{adjunct}}}$            & \nbar $\rightarrow$ \nbar~~$\overline{\overline{\mbox{REL\_CLAUSE}}}$ & \mbox{[picture of Paris]}\\
                            &                                              & \mbox{[that everybody knows]}\\
\xbar $\rightarrow \overline{\overline{\mbox{adjunct}}}$~~\xbar            & \nbar $\rightarrow \overline{\overline{\mbox{A}}}$~~\nbar & \mbox{beautiful [picture of Paris]}\\
\xbar $\rightarrow$ \mbox{X}~~$\overline{\overline{\mbox{complement}}}*$   & \nbar $\rightarrow$ \mbox{N}~~$\overline{\overline{\mbox{P}}}$ & \mbox{picture [of Paris]}\\
\end{tabular}}

X stands for an arbitrary category, X is the head, `*' stands for arbitrarily many repretitions

\medskip
X can be placed left or right in rules

}


\frame{
\frametitle{NP structures with all projection levels}

\centerline{
\scalebox{.65}{%
\begin{forest}
sm edges
[NP
  [DetP
    [\detbar
      [Det [das;the] ] ] ]
  [\nbar
    [N [Bild;picture] ] ] ]
\end{forest}}
\hspace{5mm}
\scalebox{.65}{%
\begin{forest}
sm edges
[NP
  [DetP
    [\detbar
      [Det [das;the] ] ] ]
  [\nbar
    [AP
      [\abar
        [A [schöne;beautiful] ] ] ]
    [\nbar
      [N [Bild;picture] ]
      [PP 
        [\pbar
          [P [von;of] ]
          [NP
            [\nbar
              [N [Paris;Paris] ] ] ] ] ] ] ] ]
\end{forest}}}
%
%
%\hfill\mbox{}


}




%      <!-- Local IspellDict: en_US-w_accents -->
